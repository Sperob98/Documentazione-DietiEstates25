\section{Descrizione dell'architettura}

Il sistema è basata su un architettura \textbf{client-server}, questo stile architetturale agevola \textbf{scalabilità}, \textbf{manutenibilità} e \textbf{flessibilità}.
La comunicazione tra client e server è implementata secondo lo stile architetturale \textbf{REST}, mediante l’utilizzo di una \textbf{RESTful API}.
\\
Ricordiamo che un \textbf{API (application programming interface)}, sono \textbf{set
di definizioni e protocolli} con cui vengono realizzati e integrati software applicativi. Talvolta le si definisce come un contratto tra un fornitore di informazioni e l’utente destinatario
di tali dati: l’API stabilisce il contenuto richiesto dal consumatore (la chiamata) e il contenuto
richiesto dal produttore (la risposta). In un’architettura client-server, come quella adottata in
questo sistema, il Server funge da produttore di dati e servizi, mentre il Client agisce come
consumatore, richiedendo e utilizzando tali risorse tramite le API esposte dal Server.
\\
Una \textbf{REST (Representational state transfer)} è invece un insieme di limitazioni architetturali, non si tratta di un protocollo o uno standard. 
Esso segue i seguenti principi:

\begin{itemize}
	\item \textbf{Client-Server}: Le responsabilità devono essere separate. I server non si devono fare carico dell'interfaccia grafica o dello stato dell'utente e il client non si deve preoccupare del salvataggio delle informazioni, che rimangono all'interno dei singoli server.
	\item \textbf{Stateless}: La comunicazione client–server è vincolata in modo che nessun contesto client venga memorizzato sul server tra le richieste. Ciascuna richiesta dai vari client contiene tutte le informazioni necessarie per richiedere il servizio e lo stato della sessione è contenuto nel client.
	\item \textbf{Cacheable}: I client possono mettere in cache le risposte. Queste devono in ogni modo definirsi cacheable o no, in modo da prevenire che i client possano riutilizzare stati vecchi e dati errati. Tale caratteristica migliora scalabilità e prestazioni.
	\item \textbf{Layered system}: Il sistema può essere stratificati su più livelli, ad esempio, pubblicare le API in un server, memorizzare i dati in un secondo server e gestire l'autenticazione delle richieste in un terzo server.
	\item \textbf{Uniform interface}: Un'interfaccia di comunicazione omogenea tra client e server.
\end{itemize}

Un concetto importante in REST è l'esistenza di \textbf{risorse (fonti di informazioni)}, a cui si può accedere tramite un \textbf{identificatore globale (un URI)}. Per utilizzare le risorse, le componenti (client e server) comunicano attraverso un'interfaccia standard \textbf{(per esempio HTTP)} per scambiare rappresentazioni di queste risorse, ovvero il documento che trasmette le informazioni. 
L'applicazione deve conoscere il formato dell'informazione restituita, ovvero la sua rappresentazione. Tipicamente è un documento \textbf{HTML}, \textbf{XML} o \textbf{JSON}, ma possono essere anche immagini o altri contenuti.
\\
Una \textbf{RESTful API} Indica un’API che segue rigorosamente i principi REST quindi:

\begin{itemize}
	\item è \textbf{stateless}
	\item usa i metodi HTTP (\colorbox{lightgray}{GET}, \colorbox{lightgray}{POST}, \colorbox{lightgray}{PUT}, \colorbox{lightgray}{DELETE}, ecc.) secondo la loro semantica;
	\item espone risorse identificate da \textbf{URI} chiari (es. \colorbox{lightgray}{/users/1});
	\item usa rappresentazioni (JSON, XML, ecc.) per trasferire lo stato;
	\item impiega correttamente status code HTTP.
\end{itemize}