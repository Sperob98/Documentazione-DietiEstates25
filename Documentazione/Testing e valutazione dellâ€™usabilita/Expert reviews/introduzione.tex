\subsection{Expert Reviews / Inspections}

È stata pianificata una revisione sistematica dell’usabilità del prodotto, basata su una \textbf{checklist di valutazione euristica}.  
La revisione è stata condotta internamente, con l’obiettivo di individuare punti di forza e criticità dell’interfaccia utente, sia del \textit{prototipo realizzato in Figma} sia della \textit{versione web funzionante} del sito.

\subsection*{Obiettivi della Review}

Gli obiettivi principali dell’attività di \textit{Expert Review} sono stati:
\begin{itemize}
    \item Valutare la \textbf{usabilità} dei flussi principali del sistema.
    \item Analizzare la \textbf{coerenza visiva} e l’aderenza alle scelte di brand.
    \item Individuare problematiche legate a \textbf{chiarezza, navigabilità e feedback} del sistema.
    \item Confrontare il \textbf{prototipo Figma} con la \textbf{versione implementata}, evidenziando differenze o regressioni.
\end{itemize}

\subsection*{Motivazione della Checklist Scelta}

Per garantire una valutazione sistematica e comparabile tra diverse parti dell’interfaccia, è stata adottata una \textbf{Heuristic Evaluation} basata sui dieci principi di usabilità proposti da Nielsen \cite{nielsen1995}.  
Tale approccio è stato ritenuto adeguato in quanto:
\begin{itemize}
    \item è applicabile anche in assenza di utenti reali o revisori esterni, consentendo un’autovalutazione strutturata;  
    \item si adatta sia a \textbf{prototipi statici} (come quelli realizzati in Figma) sia a \textbf{interfacce funzionanti};  
    \item copre in modo ampio i principali aspetti di usabilità: visibilità dello stato del sistema, coerenza, prevenzione degli errori, efficienza e feedback.
\end{itemize}

\subsection*{Ambito di Valutazione}

\textbf{Prototipo Figma:} la checklist è stata applicata ai quattro casi d’uso principali, già trattati negli esami, più alla Home, in quanto rappresenta la prima interazione dell’utente con il sistema:
\begin{enumerate}
	\item  \textbf{Home} - prima schermata e punto di accesso alle principali funzionalità.
    \item \textbf{Creazione di un annuncio immobiliare} – flusso dell’agente per la pubblicazione di un nuovo immobile.
    \item \textbf{Registrazione di un nuovo impiegato} – flusso del manager per l’aggiunta di nuovi agenti o altri manager.
    \item \textbf{Contrattazione di una proposta} – flusso di gestione delle proposte inviate dai clienti.
    \item \textbf{Gestione delle categorie di notifiche} – flusso lato utente per l’attivazione o disattivazione delle diverse tipologie di notifiche.
\end{enumerate}

\textbf{Versione Web:} la checklist è stata applicata all’intero insieme di requisiti e funzionalità richiesti dal docente, permettendo una valutazione completa dell’interfaccia e dei flussi principali del sistema.

\subsection*{Definizione della Checklist di Usabilità}

La checklist elaborata si basa su una selezione adattata delle euristiche di Nielsen, ridotte e riformulate per meglio aderire al contesto del progetto.  
Ogni criterio è espresso come domanda di verifica utilizzata per valutare sia il prototipo sia il sito web.

\begin{table}[h!]
\centering
\begin{tabular}{p{0.8cm} p{4cm} p{7cm}}
\hline
\textbf{\#} & \textbf{Criterio} & \textbf{Domanda di verifica} \\
\hline
1 & Visibilità dello stato del sistema & L’utente riceve sempre un feedback visivo o testuale in seguito a un’azione (es. salvataggio, invio, errore)? \\
2 & Corrispondenza tra sistema e mondo reale & Terminologia, icone e messaggi sono coerenti con il linguaggio del dominio immobiliare? \\
3 & Controllo e libertà dell’utente & È possibile annullare o correggere facilmente un’azione (es. modifica o cancellazione di un annuncio)? \\
4 & Coerenza e standard & Colori, stili e pulsanti rispettano il design definito nel brand e nel prototipo Figma? \\
5 & Prevenzione e gestione degli errori & I form prevengono errori tramite validazioni e messaggi chiari? \\
6 & Efficienza e flessibilità d’uso & Le operazioni più frequenti possono essere completate rapidamente e da dispositivi mobili verticali? \\
7 & Chiarezza del contenuto & I testi e le etichette guidano l’utente nel flusso, senza ambiguità? \\
8 & Supporto al riconoscimento & Le opzioni principali sono sempre visibili senza richiedere memoria a breve termine all’utente? \\
9 & Feedback e conferme & Sono presenti messaggi di conferma dopo operazioni critiche (es. pubblicazione, eliminazione, invio)? \\
10 & Aiuto e documentazione & È disponibile una sezione informativa o un supporto base per l’utente? \\
\hline
\end{tabular}
\caption{Checklist di usabilità basata sulle euristiche di Nielsen, adattata al contesto del sito immobiliare multi-agenzia.}
\label{tab:checklist_usabilita}
\end{table}

\subsection*{Applicazione della Checklist}

La review è stata eseguita individualmente per ciascun caso d’uso su Figma e per l’intero insieme di funzionalità sulla versione web.  
I risultati sono stati successivamente confrontati per identificare le criticità più significative e le aree di miglioramento.

\paragraph{Osservazioni principali}
\begin{itemize}
    \item \textbf{Prototipo Figma:} i punti di forza e le criticità sono stati valutati solo per i quattro casi d’uso principali.  
        - Struttura chiara, buona coerenza visiva e corretto uso dei colori di brand.  
        - Mancano alcuni feedback utente (es. messaggi di conferma o errore).
    \item \textbf{Versione Web:} la valutazione copre tutti i requisiti richiesti dal docente, permettendo di individuare problemi di usabilità, coerenza visiva, feedback, validazioni e responsività.  
        - Interfaccia coerente con il design, ma con alcune carenze nella gestione dei feedback e nei messaggi di errore.  
        - Responsività su dispositivi mobili parziale ma funzionale.
    \item \textbf{Casi d’Uso Specifici (per Figma):} 
        \begin{itemize}
            \item \textit{Creazione di un annuncio:} assenza di un messaggio di conferma dopo la pubblicazione.  
            \item \textit{Registrazione di un impiegato:} i campi obbligatori non sempre evidenziati.  
            \item \textit{Contrattazione di una proposta:} le azioni accettazione/rifiuto non forniscono feedback immediato.  
            \item \textit{Gestione delle categorie di notifiche:} comportamento delle spunte non sempre intuitivo.
        \end{itemize}
\end{itemize}

\subsection*{Conclusioni e Raccomandazioni}

La revisione ha permesso di individuare punti di miglioramento chiave, pur confermando la solidità generale della struttura e del design.  
Si raccomanda di:
\begin{itemize}
    \item Introdurre \textbf{feedback visivi e testuali} chiari in tutti i flussi principali.
    \item Migliorare la \textbf{validazione dei form} e la segnalazione di errori.
    \item Ottimizzare la \textbf{responsività} per dispositivi mobili verticali.
    \item Integrare una sezione di \textbf{aiuto o guida utente} di base.
\end{itemize}

Nel complesso, l’interfaccia risulta coerente, intuitiva e in linea con gli obiettivi progettuali definiti.  
Le criticità emerse costituiranno la base per le successive iterazioni del design e del prototipo.
