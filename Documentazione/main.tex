\documentclass[a4paper,12pt]{report}

% Importazione pacchetti
\usepackage[utf8]{inputenc}
\usepackage{geometry}
\usepackage{tikz} 
\usetikzlibrary{positioning}
\usepackage{setspace} %gestione degli spazi
\geometry{
  a4paper,
  left=2.5cm,
  top=2.5cm,
  bottom=2.5cm,
  top=2.5cm
}
\onehalfspacing   % interlinea 1.5
\setlength{\parindent}{0pt} % toglie il rientro
\usepackage{tabularx}
\usepackage{verbatim}
\usepackage{float}
\usepackage[utf8]{inputenc}
\usepackage[T1]{fontenc}
\usepackage{amsmath}
\usepackage{lipsum}
\usepackage{hyperref}
\usepackage{enumitem}
\usepackage{longtable} % Per le tabelle che si estendono su più pagine, se necessario
\usepackage{graphicx} 
\usepackage{colortbl}
\usepackage[table,xcdraw]{xcolor}
\usepackage{booktabs}
\usepackage{multirow} % Pacchetto per celle multiple nelle tabelle
\usepackage{titlesec}
\titlespacing*{\section}{0pt}{1ex}{1ex} % Modifica per \section
\titlespacing*{\subsection}{0pt}{0.5ex}{0.5ex} % Modifica per \subsection
\titlespacing*{\subsubsection}{0pt}{0.5ex}{0.5ex} % Modifica per \subsubsection
\usepackage{array} % Per definire tabelle con colonne adattabili
\usepackage{graphicx} % Per gestire immagini e figure
\usepackage{adjustbox} % Per ridurre tabelle che escono dai margini
\usepackage{ragged2e} % Per la gestione del testo a capo nelle celle
\usepackage{makecell}
\usepackage{listings}
\usepackage{xcolor}



\newcommand{\NomeProgetto}{DietiEstetes25}

\begin{document}

% Copertina Università
\begin{titlepage}
    \centering
    \includegraphics[width=0.3\textwidth]{Immagini/logo_universita.png} % Inserisci il logo
    
    \vspace{1cm}
    
    {\LARGE UNIVERSITÀ DEGLI STUDI DI NAPOLI FEDERICO II \\}
    \vspace{0.3cm}
    {\Large SCUOLA POLITECNICA E DELLE SCIENZE DI BASE \\}
    \vspace{0.3cm}
    {\Large DIPARTIMENTO DI INGEGNERIA ELETTRICA E \\ TECNOLOGIE DELL'INFORMAZIONE \\}
    
    \vspace{1.5cm}
    
    {\Large \textbf{CORSO DI LAUREA IN INFORMATICA \\}}
    \vspace{0.3cm}
    {\Large INGEGNERIA DEL SOFTWARE \\}
    
    \vspace{1.5cm}
    
    {\Huge \textbf{\NomeProgetto} \\}
    \vspace{0.5cm}
    {\Large Piattaforma per la gestione di agenzie immobiliari \\}
    
    \vfill
    
    {\large Raimondo Morosini - N86003839 \\}
    {\large Roberto Spena - N86003552 \\}
    {\large Lorenzo Sepe - N86003622 \\}
    
    \vspace{1.5cm}
    
    {\large Anno Accademico 2024/2025 \\}\newpage
\end{titlepage}
\newpage
Questa pagina è stata lasciata intenzionalmente bianca

% Pagina del titolo
%\maketitle
\tableofcontents % Crea indice


% Include i capitoli direttamente

\chapter{Revisioni}
% Please add the following required packages to your document preamble:
% \usepackage[table,xcdraw]{xcolor}
% Beamer presentation requires \usepackage{colortbl} instead of \usepackage[table,xcdraw]{xcolor}
% \usepackage{longtable}
% Note: It may be necessary to compile the document several times to get a multi-page table to line up properly
\begin{longtable}{|l|l|l|l|}
\caption{Tabella Revisioni documento }
\label{tab:my-table}\\
\hline
\rowcolor[HTML]{10b981} 
{\color[HTML]{FFFFFF} Data} & {\color[HTML]{FFFFFF} Versione} & {\color[HTML]{FFFFFF} Autore/i} & {\color[HTML]{FFFFFF} Descrizione} \\ \hline
\endhead
%
22/01/25                    & 0.1                             & R. Spena, R. Morosini, L. Sepe  &  Prima Stesura                      \\ \hline
\rowcolor[HTML]{ecfdf5} 
  24/01/25                  & 0.2                             & R. Spena, R. Morosini, L. Sepe  & Aggiunte Personas        \\ \hline
   27/01/25                 & 0.3                             & R. Spena, R. Morosini, L. Sepe  & Aggiunte Tabelle Cockburn\\ \hline
\rowcolor[HTML]{ecfdf5} 
  29/01/25                  &  0.4                            & L. Sepe                         &\begin{tabular}{c}
        Completate Tabelle Cockburn \\
        Inizio Lavoro su MockUp
  \end{tabular}         \\ \hline
    31/01/25                & 0.5                             & R. Spena, R. Morosini, L. Sepe  &    Revisioni minori                          \\ \hline
\rowcolor[HTML]{ecfdf5} 
   19/02/25                 &    0.6                          &    L. Sepe                     &  Revisione Tabelle Cockburn                           \\ \hline
                            &                                 &                                 &                                    \\ \hline
\rowcolor[HTML]{ecfdf5} 
    21/02/25                           & 0.7                                &   R. Spena                              &   Inserita grafica tabelle dei requisiti                                 \\ \hline
                            &                                 &                                 &                                    \\ \hline
\rowcolor[HTML]{ecfdf5} 
    21/02/25                           & 0.7                                &   R. Morosini                              &   Inserita descrizione ai mockup delle estensioni                                 \\ \hline
                            &                                 &                                 &                                    \\ \hline
                            

\end{longtable}

%2.Documento dei requisiti del software
\chapter{Requisiti Software}
\section*{Introduzione}
In questa sezione verranno raccolti e descritti i vari requisiti del sistema in analisi, suddivisi in requisiti funzionali e non funzionali.\\
 I requisiti funzionali sono stati derivati dai casi d'uso identificati nella fase di analisi, mentre i requisiti non funzionali riguardano aspetti quali prestazioni, 
 sicurezza e usabilità del sistema. 

\section*{Obiettivo}
Il sistema deve fornire un'area amministrativa e un'area Cliente, con funzionalità specifiche per i diversi ruoli utente e per la gestione degli Immobili. Le funzionalità sono state riformulate in un linguaggio tecnico per favorire chiarezza e comprensione.

\section*{Area Amministrativa}
L'area amministrativa consente la gestione degli account e degli Immobili. I ruoli previsti sono:
\begin{itemize}
    \item Manager
    \item Manager di supporto
    \item Agenti Immobiliari Immobiliari
\end{itemize}

\subsection*{Accesso all'Area Amministrativa}
\begin{description}[style=nextline]
    \item[Manager:] L'accesso avviene dopo la registrazione di un'agenzia Immobiliare tramite richiesta al sistema. Una volta approvata, vengono fornite credenziali predefinite per l'accesso.
    \item[Manager di supporto:] Creati dall'Manager principale e associati alla stessa agenzia Immobiliare. Le credenziali sono generate dall'Manager.
    \item[Agenti Immobiliari Immobiliari:] Creati dall'Manager o dall'Manager di supporto. Le credenziali sono anch'esse generate dall'Manager.
\end{description}

\section*{Area Cliente}
L'area Cliente è destinata agli utenti registrati che possono:

\begin{itemize}
    \item \textbf{Ricercare Immobili:}
    \begin{itemize}
        \item Ricerca avanzata con filtro per posizione geografica (selezionando un punto e un raggio su una mappa).
        \item Visualizzazione degli Immobili in lista o su mappa.
        \item Filtri aggiuntivi per caratteristiche dell'immobile e del contratto (es. metratura, numero di stanze, ecc.).
    \end{itemize}

    \item \textbf{Visualizzare Immobili completi di dettagli:}
    \begin{itemize}
        \item Inclusi dati come metratura, stanze, servizi, ecc.
        \item Informazioni sui luoghi di interesse vicini (es. scuole, parchi, metro) utilizzando l'integrazione con Geopify.
    \end{itemize}

    \item \textbf{Proporre offerte:}
    \begin{itemize}
        \item L'utente registrato può inviare una proposta sull'immobile, che verrà gestita dall'Agente Immobiliare.
    \end{itemize}

    \item \textbf{Gestire notifiche push:}
    \begin{itemize}
        \item Notifiche categorizzate (promozionali, private, ecc.).
        \item Possibilità di disattivare categorie specifiche.
    \end{itemize}
\end{itemize}

\section*{Funzionalità dei Ruoli}
\subsection*{Manager}
\begin{itemize}
    \item Creare account per Manager di supporto e Agenti Immobiliari Immobiliari.
    \item Modificare e cancellare definitivamente Immobili creati dagli Agenti Immobiliari.
    \item Modificare la propria password.
\end{itemize}

\subsection*{Manager di supporto}
\textbf{Nota:} Non sono stati definiti requisiti specifici per questo ruolo. Si propone di assegnare le stesse funzionalità dell'Manager, con eventuali limitazioni future da specificare.

\subsection*{Agenti Immobiliari Immobiliari}
\begin{itemize}
    \item Creare, modificare e cancellare Immobili da loro creati.
    \item Visualizzare le proposte ricevute sugli Immobili.
    \item Registrare proposte esterne.
    \item Accettare o rifiutare proposte.
    \item Modificare la propria password.
\end{itemize}

% punto a)
\newpage
\section{Glossario}

\textbf{Amministratore}: Ruolo all'interno del sistema che gode di accesso a tutti gli Endpoint, autorizzazioni  e metodi.\\
\textbf{Agente}: Ruolo all'interno del sistema che è autorizzato a creare, modificare ed eliminare gli Annunci per gli immobili gestiti dall'Agenzia.\\
\textbf{Agenzia}: Entità caratterizzata dal suo Amministratore e Agenti in  impiego. Alla sua creazione nella base dati verrà automaticamente creato un Amministratore.\\
\textbf{Annunci}: Entità caratterizzata dal un Immobile e il tipo di Contratto.\\

\newpage
% punto b)
\newpage
\section{Modellazione dei Casi d'Uso}
Dopo aver definito i requisiti funzionali e i vincoli del sistema, questa sezione illustra i casi
d'uso (use case), che descrivono in dettaglio le principali interazioni tra gli utenti e il sistema
per il raggiungimento degli obiettivi. Nella nostra analisi preliminare, abbiamo individuato quattro
attori principali che interagiscono con il sistema: Guest, Utente, Agente e Manager.\\
\begin{itemize}
    \item Guest: rappresenta l'utente non autenticato, che accede al sito senza effettuare il login. Questo attore ha accesso limitato, ma può comunque compiere alcune azioni di base, come
effettuare ricerche sugli Immobili e visualizzare informazioni pubbliche su di essi.
\item User: questo attore è un utente autenticato con un account registrato. Rispetto al Guest, ha funzionalità aggiuntive, come fare
offerte sugli Immobili, monitorare il proprio storico delle ricerche effettuate e ricevere notifiche in base alle preferenze che ha esibito. Queste notifiche possono esse ulteriormente personalizzata nelle impostazioni con dei filtri.
\item Agente: rappresenta l'utente che può creare e modificare gli Annunci e gli Immobili.
\item Manager: l'attore che gestisce l'agenzia Immobiliare nella sua interezza, avendo il potere di appuntare sia Agenti che altri Manager. Inoltre il Manager può curare il catalogo di Annunci, modificando o eliminando elementi dalla lista.
\end{itemize}
Per ciascun attore, sono stati definiti i casi d'uso relativi, illustrati nei diagrammi seguenti. Questi schemi mostrano le modalità principali d'interazione con il sistema e forniscono una
visione chiara, delineando le possibilità di interazione per ciascun tipo di utente. In questo
modo, è possibile comprendere come le funzionalità definite nei requisiti trovano applicazione
pratica all'interno delle azioni eseguibili da ciascun attore, evidenziando i passaggi chiave delle interazioni.
\newpage
\subsection*{Diagrammi Casi d'uso}

\begin{figure}[H]
\caption{Casi d'uso Guest}
\centering
\includegraphics[width=0.8\textwidth]{Immagini/Diagrammi Casi D'uso/UseCase-Utente Non Registrato.drawio.png}
\end{figure}

\begin{figure}[H]
\centering
\caption{Casi d'uso Utente}
\includegraphics[width=0.8\textwidth]{Immagini/Diagrammi Casi D'uso/UseCase-Utente registrato.drawio.png}
\end{figure}

\begin{figure}[H]
\centering
\caption{Casi d'uso Agente}
\includegraphics[width=0.8\textwidth]{Immagini/Diagrammi Casi D'uso/UseCase-Agente.drawio.png}
\end{figure}

\begin{figure}[H]
\centering
\caption{Casi d'uso Manager}
\includegraphics[width=0.8\textwidth]{Immagini/Diagrammi Casi D'uso/UseCase-Admin.drawio.png}
\end{figure}
\newpage

% punto c)
\section{Analisi Target Utenti
}
Quando si crea un software, è importante ricordare che non è necessario soddisfare ogni singolo utente possibile, ma solo la categoria di utenti che usufruiranno del nostra sistema più frequentemente.\\
Questa decisione ci permette di concentrare la nostra analisi e ridurre funzioni che, al momento del rilascio del software al committente, sono superflue.\\
Uno strumento utile per fare questa analisi è la 
Persona, un modello di utente ideale che può essere basato su ricerche di mercato oppure da input dai committenti stessi.\\
\subsection*{Personas Individuate}
Per creare Personas è stato adottato un approccio basato su ipotesi e ricerche secondarie.\\
Abbiamo così definito il contesto, identificando il pubblico target, i loro obiettivi e le situazioni in cui potrebbero utilizzare il prodotto. Si possono utilizzare archetipi comuni del dominio di riferimento per immaginare utenti tipo, discutendo possibili bisogni, obiettivi e preferenze degli utenti.\\
Abbiamo creato le Personas a partire da una template predefinita e usando Figma per il design e l'aggiornamento di esse in corso d'opera.
\\

\begin{figure}[H]
\centering
\caption{Proprietario Agenzia Immobiliare}
\includegraphics[width=1\textwidth]{{Immagini/Personas/Personas-Propretario di un agenzia Immobiliare.png}}
\end{figure}

\begin{figure}[H]
\centering
\caption{Agente Immobiliare}
\includegraphics[width=1\textwidth]{{Immagini/Personas/Personas-Agente Immobiliare.png}}
\end{figure}

\begin{figure}[H]
\centering
\caption{Manager di azienda tecnologica}
\includegraphics[width=1\textwidth]{Immagini/Personas/Personas-manager di azienda tecnologica.png}
\end{figure}

\begin{figure}[H]
\centering
\caption{Padre di famiglia}
\includegraphics[width=1\textwidth]{Immagini/Personas/Personas-Padre di famiglia.png}
\end{figure}
\newpage
\subsection*{Tratti Caratteristici}
Analizzando queste Personas possiamo notare dei tratti che possiamo usare nella nostra applicazione:
\begin{itemize}
    \item  L'Agente- Aldo Imparato: questa categoria di utenti è parte dello staff dell'Agenzia Immobiliare e quindi ha accesso al lato di amministrazione degli Immobili, e potrebbe volere una sezione nella sa pagina personale che gli mostra tutti gli Immobili a suo carico.
    \item  Il Padre di Famiglia - Giovanni Rossi: l'archetipo che incarna il Utente medio che ha vaghe idee su cosa vuole e si affida all'Agente Immobiliare. Dobbiamo quindi tener conto che le informazioni visibili a questo tipo di utenti deve essere chiara e concisa, come ad esempio le etichette che indicando punti di interesse come parchi e scuole.
    \item Il Manager di Azienda Tecnologica - Giorgia Esposito: l'utente che sta cercando qualcosa di ben definito, ma l'agenzia non offre Immobili con le caratteristiche cercate. Per questo tipo di persone offriremo un servizio di iscrizione ai tag di interesse per ricevere notifiche quando un immobile che rispecchia tali desideri viene messo in vendita.
    \item Il Proprietario dell'Agenzia Immobiliare - Marco Bianchi: gli Manager del sistema che gestiscono dipendenti e il catalogo di Immobili, questo tipo di utente deve gestire multipli dati personali e sensibili. Quando crea un Agente o Manager, le nuove informazioni verranno generate dal sistema e inviate tramite un servizio di posta elettronica.
\end{itemize}
\newpage
% punto d)
\clearpage
\section{Descrizione Requisiti}

\subsection{Requisiti non funzionali}

\subsection*{Sicurezza}

\begin{table}[H]
    \centering
    \renewcommand{\arraystretch}{1.3} % Aumenta leggermente l'altezza delle righe
    
    \begin{tabular}{|p{3cm}|p{10cm}|} 
        \hline
        \textbf{ID} & \textbf{Descrizione} \\  
        \hline
        NF-S1 &  Implementazione di protocolli standard di sicurezza (es. HTTPS per il trasporto sicuro
        dei dati). \\ 
        \hline
        NF-S2 &  Le credenziali devono essere salvate in modo sicuro, utilizzando tecniche di hashing sicuro. \\ 
        \hline
        NF-S3 &  Le richieste alle REST API devono essere autenticate utilizzando JWT (JSON Web Tokens), per garantire che solo gli utenti autenticati possano accedere alle risorse protette. \\ 
        \hline
    \end{tabular}
    
\end{table}
\subsection*{Interfaccia utente}

\begin{table}[H]
    \centering
    \renewcommand{\arraystretch}{1.3} % Aumenta leggermente l'altezza delle righe
    \begin{tabular}{|p{3cm}|p{10cm}|} 
        \hline
        \textbf{ID} & \textbf{Descrizione} \\  
        \hline
        NF-UI11 & L'interfaccia utente deve adattarsi automaticamente alla risoluzione dello schermo dell’utente, garantendo un’esperienza di utilizzo ottimale su dispositivi mobili e desktop. \\ 
        \hline
    \end{tabular}
\end{table}
\subsection*{Prestazioni}

\begin{table}[H]
    \centering
    \renewcommand{\arraystretch}{1.3} % Aumenta leggermente l'altezza delle righe
    \begin{tabular}{|p{3cm}|p{10cm}|} 
        \hline
        \textbf{ID} & \textbf{Descrizione} \\  
        \hline
        NF-P1 & La ricerca e la visualizzazione degli Immobili devono avere un tempo di risposta inferiore
        a 2 secondi, con un database contenente almeno 10.000 record. \\ 
        \hline
    \end{tabular}
\end{table}
\subsection*{Manutenibilità e scalabilità}

\begin{table}[H]
    \centering
    \renewcommand{\arraystretch}{1.3} % Aumenta leggermente l'altezza delle righe
    \begin{tabular}{|p{3cm}|p{10cm}|} 
        \hline
        \textbf{ID} & \textbf{Descrizione} \\  
        \hline
        NF-MS1 & Il sistema deve essere progettato per supportare l’aggiunta di nuove tipologie di Immobili
        e contratti (ad esempio, affitti brevi) senza modificare l’architettura esistente. \\ 
        \hline
    \end{tabular}
\end{table}

\newpage

\subsection{Requisiti di dominio}

\subsection*{Gestione dell'agenzia Immobiliare}

\begin{table}[H]
    \centering
    \renewcommand{\arraystretch}{1.3} % Aumenta leggermente l'altezza delle righe
    
    \begin{tabular}{|p{3cm}|p{10cm}|} 
        \hline
        \textbf{ID} & \textbf{Descrizione} \\  
        \hline
        D-A1 &  Un’agenzia Immobiliare è composta da un fondatore e da una lista di dipendenti, ciascuno
        dei quali ricopre il ruolo di Manager o Agente Immobiliare. \\ 
        \hline
        D-A2 &  La registrazione di un’agenzia Immobiliare comporta la generazione automatica di un
        Manager con il ruolo di fondatore. \\ 
        \hline
        D-A3 &  Alla creazione di un account Manager vengono assegnate credenziali predefinite
        che possono essere modificate in seguito. \\ 
        \hline
        D-A4 &   Le credenziali di un Manager includono uno username, costruito seguendo il formato [nome][cognome]@DIETI25.com, e una password scelta dall’utente. \\ 
        \hline
    \end{tabular}
    
\end{table}
\subsection*{Gestione degli utenti}

\begin{table}[H]
    \centering
    \renewcommand{\arraystretch}{1.3} % Aumenta leggermente l'altezza delle righe
    
    \begin{tabular}{|p{3cm}|p{10cm}|} 
        \hline
        \textbf{ID} & \textbf{Descrizione} \\  
        \hline
        D-U1 & Un guest può registrarsi al sistema fornendo un’email valida e una password, diventando così un utente. \\ 
        \hline
        D-U2 & Un guest può registrarsi o accedere tramite l’uso di API di terze parti. \\ 
        \hline
    \end{tabular}
    
\end{table}
\subsection*{Gestione degli Immobili}

\begin{table}[H]
    \centering
    \renewcommand{\arraystretch}{1.3} % Aumenta leggermente l'altezza delle righe
    
    \begin{tabular}{|p{3cm}|p{10cm}|} 
        \hline
        \textbf{ID} & \textbf{Descrizione} \\  
        \hline
        D-G1 &  Ogni immobile è descritto da una serie di dettagli obbligatori, tra cui: foto, descrizione, prezzo, dimensioni, indirizzo, numero di stanze, piano, presenza di ascensore, classe energetica e ulteriori servizi (es. portineria, climatizzazione). \\ 
        \hline
        D-G2 &  Ogni immobile è associato a una delle seguenti tipologie contrattuali: "vendita" o "affitto". \\ 
        \hline
        D-G3 &  Ogni immobile, al momento della creazione, ha associata una lista di punti di riferimento generata automaticamente utilizzando il servizio GEOAPIFY. \\ 
        \hline
    \end{tabular}
    
\end{table}
\subsection*{Ricerca e visualizzazione}

\begin{table}[H]
    \centering
    \renewcommand{\arraystretch}{1.3} % Aumenta leggermente l'altezza delle righe
    
    \begin{tabular}{|p{3cm}|p{10cm}|} 
        \hline
        \textbf{ID} & \textbf{Descrizione} \\  
        \hline
        D-R1 &  La ricerca degli immobili consente di effettuare una selezione geografica basata su un punto centrale e un raggio, garantendo una precisione pari o superiore al 95\%. \\ 
        \hline
        D-R2 &  La ricerca avanzata degli immobili permette di utilizzare parametri multipli, tra cui tipologia di inserzione, prezzo minimo e massimo, numero di stanze, classe energetica e area geografica tramite mappa interattiva. \\ 
        \hline
        D-R3 &  Gli immobili ricercati possono essere visualizzati su una mappa interattiva. \\ 
        \hline
    \end{tabular}
    
\end{table}
\subsection*{Gestione delle offerte}

\begin{table}[H]
    \centering
    \renewcommand{\arraystretch}{1.3} % Aumenta leggermente l'altezza delle righe
    
    \begin{tabular}{|p{3cm}|p{10cm}|} 
        \hline
        \textbf{ID} & \textbf{Descrizione} \\  
        \hline
        D-O1 & Le inserzioni degli Immobili possono ricevere offerte da parte degli utenti, composte da un prezzo proposto e dalle credenziali dell’offerente. Le offerte sono visibili a tutti gli utenti. \\ 
        \hline
    \end{tabular}
    
\end{table}
\subsection*{Gestione delle notifiche}

\begin{table}[H]
    \centering
    \renewcommand{\arraystretch}{1.3} % Aumenta leggermente l'altezza delle righe
    
    \begin{tabular}{|p{3cm}|p{10cm}|} 
        \hline
        \textbf{ID} & \textbf{Descrizione} \\  
        \hline
        D-N1 & Il sistema di notifiche consente di inviare avvisi personalizzati agli utenti, categorizzati in base a eventi come: nuove proprietà in linea con ricerche precedenti, conferme o rifiuti di proposte, e messaggi promozionali. \\ 
        \hline
        D-N2 & Le notifiche promozionali sono composte da un contenuto in formato RICH TEXT, definito dagli amministratori, e sono visibili esclusivamente agli utenti iscritti alla relativa newsletter. \\ 
        \hline
    \end{tabular}
    
\end{table}


% punto e. I)
\section{Descrizione Testuale strutturata del Sistema}
In questa sezione faremo un analisi più approfondita di alcuni dei casi d'uso campione del sistema, per esporre il nostro ragionamento per modellare e sviluppare il Software .\\
Questa descrizione ci servirà per pensare a che tipo di interazioni con l'Utente sono necessarie, i passaggi da effettuare  dal sistema in risposta alle richieste ricevute e 
I casi d'uso scelti per questo scopo sono la Creazione di un nuovo Annuncio di un Immobile e Modifica notifiche in Arrivo.
  
    \subsection*{Creazione Nuovo Annuncio per Immobile}
% Please add the following required packages to your document preamble:
% \usepackage{multirow}
% \usepackage[table,xcdraw]{xcolor}
% Beamer presentation requires \usepackage{colortbl} instead of \usepackage[table,xcdraw]{xcolor}
% \usepackage{longtable}
% Note: It may be necessary to compile the document several times to get a multi-page table to line up properly

\subsection{Cockburn: nuovo annuncio}

\begin{longtable}{|l|lll|}
\caption{Creazione nuovo Annuncio}
\label{tab:my-table}\\
\hline
\rowcolor[HTML]{B2C9AB} 
\textbf{Use Case 1}                                                                                                                                                                              & \multicolumn{3}{l|}{\cellcolor[HTML]{B2C9AB}\textbf{Creazione Nuovo Annuncio per Immobile}}                                                                                                                                                                                                                                                                                                                   \\ \hline
\endhead
%
\cellcolor[HTML]{B2C9AB}\textbf{Goal In Context}                                                                                                                                                 & \multicolumn{3}{l|}{\begin{tabular}[c]{@{}l@{}}Inserimento nel Catalogo dell'Agenzia Immobiliare\\ un nuovo Annuncio per Immobile.\end{tabular}}                                                                                                                                                                                                                                                              \\ \hline
\cellcolor[HTML]{B2C9AB}\textbf{Preconditions}                                                                                                                                                   & \multicolumn{3}{l|}{\begin{tabular}[c]{@{}l@{}}- Login effettuato con account con\\ ruolo Agente o Amministratore. \\ - Si trova nella pagina di Creazione Annunci.\end{tabular}}                                                                                                                                                                                                                             \\ \hline
\cellcolor[HTML]{B2C9AB}\textbf{\begin{tabular}[c]{@{}l@{}}Success End\\ Conditions\end{tabular}}                                                                                                & \multicolumn{3}{l|}{\begin{tabular}[c]{@{}l@{}}Aggiunta in Catalogo dell'Immobile e Salvataggio\\ nel Database.\end{tabular}}                                                                                                                                                                                                                                                                                 \\ \hline
\cellcolor[HTML]{B2C9AB}\textbf{Primary Actor}                                                                                                                                                   & \multicolumn{3}{l|}{Agente}                                                                                                                                                                                                                                                                                                                                                                                   \\ \hline
\cellcolor[HTML]{B2C9AB}\textbf{Trigger}                                                                                                                                                         & \multicolumn{3}{l|}{\begin{tabular}[c]{@{}l@{}}Agente preme il bottone Aggiungi \\ Immobile nella Schermata del Catalogo.\end{tabular}}                                                                                                                                                                                                                                                                       \\ \hline
\rowcolor[HTML]{B2C9AB} 
\textbf{Descrizione}                                                                                                                                                                             & \multicolumn{1}{l|}{\cellcolor[HTML]{B2C9AB}\textbf{Step n.}} & \multicolumn{1}{l|}{\cellcolor[HTML]{B2C9AB}\textbf{Agente}}                                                                                      & \textbf{Sistema}                                                                                                                                                                          \\ \hline
\cellcolor[HTML]{B2C9AB}{\color[HTML]{000000} }                                                                                                                                                  & \multicolumn{1}{l|}{1}                                        & \multicolumn{1}{l|}{}                                                                                                                             & \textit{\begin{tabular}[c]{@{}l@{}}Controlla che non ci siano\\ operazioni in sospeso\end{tabular}}                                                                                       \\ \cline{2-4} 
\cellcolor[HTML]{B2C9AB}{\color[HTML]{000000} }                                                                                                                                                  & \multicolumn{1}{l|}{2}                                        & \multicolumn{1}{l|}{}                                                                                                                             & \textit{\begin{tabular}[c]{@{}l@{}}Mostra la\\ pagina di creazione \\ Annunci.\end{tabular}}                                                                                              \\ \cline{2-4} 
\cellcolor[HTML]{B2C9AB}{\color[HTML]{000000} }                                                                                                                                                  & \multicolumn{1}{l|}{3}                                        & \multicolumn{1}{l|}{\begin{tabular}[c]{@{}l@{}}Compila 3 campi:\\ - Titolo Annuncio.\\ - Tipologia contratto.\\ Tipologia Immobile.\end{tabular}} &                                                                                                                                                                                           \\ \cline{2-4} 
\cellcolor[HTML]{B2C9AB}{\color[HTML]{000000} }                                                                                                                                                  & \multicolumn{1}{l|}{4}                                        & \multicolumn{1}{l|}{}                                                                                                                             & \textit{\begin{tabular}[c]{@{}l@{}}In base ai dati inseriti \\ mostra il form adatto \\ per la creazione.\end{tabular}}                                                                   \\ \cline{2-4} 
\cellcolor[HTML]{B2C9AB}{\color[HTML]{000000} }                                                                                                                                                  & \multicolumn{1}{l|}{5}                                        & \multicolumn{1}{l|}{\begin{tabular}[c]{@{}l@{}}Compila Fom\\ Annuncio\\ \\ \end{tabular}}                                                                                               & \textit{}                                                                                                                                                                                 \\ \cline{2-4} 
\cellcolor[HTML]{B2C9AB}{\color[HTML]{000000} }                                                                                                                                                  & \multicolumn{1}{l|}{6}                                        & \multicolumn{1}{l|}{Click Bottone Conferma.}                                                                                                      & \textit{}                                                                                                                                                                                 \\ \cline{2-4} 
\cellcolor[HTML]{B2C9AB}{\color[HTML]{000000} }                                                                                                                                                  & \multicolumn{1}{l|}{7}                                        & \multicolumn{1}{l|}{}                                                                                                                             & \textit{\begin{tabular}[c]{@{}l@{}}Mostra Riepilogo Dati\\ Inseriti.\end{tabular}}                                                                                                        \\ \cline{2-4} 
\cellcolor[HTML]{B2C9AB}{\color[HTML]{000000} }                                                                                                                                                  & \multicolumn{1}{l|}{8}                                        & \multicolumn{1}{l|}{Click Bottone Pubblica.}                                                                                                      & \textit{}                                                                                                                                                                                 \\ \cline{2-4} 
\cellcolor[HTML]{B2C9AB}{\color[HTML]{000000} }                                                                                                                                                  & \multicolumn{1}{l|}{9}                                        & \multicolumn{1}{l|}{}                                                                                                                             & \textit{Validazione Dei Dati}                                                                                                                                                             \\ \cline{2-4} 
\cellcolor[HTML]{B2C9AB}{\color[HTML]{000000} }                                                                                                                                                  & \multicolumn{1}{l|}{10}                                       & \multicolumn{1}{l|}{}                                                                                                                             & \textit{Salvataggio in Database}                                                                                                                                                          \\ \cline{2-4} 
\cellcolor[HTML]{B2C9AB}{\color[HTML]{000000} }                                                                                                                                                  & \multicolumn{1}{l|}{11}                                       & \multicolumn{1}{l|}{}                                                                                                                             & \textit{\begin{tabular}[c]{@{}l@{}}Mostra Popup\\ Operazione completata\\ con successo.\end{tabular}}                                                                                     \\ \cline{2-4} 
\multirow{-43}{*}{\cellcolor[HTML]{B2C9AB}{\color[HTML]{000000} \begin{tabular}[c]{@{}l@{}}Scenario \\ Principale\end{tabular}}}                                                                 & \multicolumn{1}{l|}{12}                                       & \multicolumn{1}{l|}{}                                                                                                                             & \textit{\begin{tabular}[c]{@{}l@{}}Use Case terminato con \\ Successo.\end{tabular}}                                                                                                      \\ \hline
\newpage
\rowcolor[HTML]{B2C9AB} 
\textbf{Extension}                                                                                                                                                                               & \multicolumn{1}{l|}{\cellcolor[HTML]{B2C9AB}\textbf{Step n.}} & \multicolumn{1}{l|}{\cellcolor[HTML]{B2C9AB}\textbf{Utente}}                                                                                      & \textbf{Sistema}                                                                                                                                                                          \\ \hline
\cellcolor[HTML]{B2C9AB}                                                                                                                                                                         & \multicolumn{1}{l|}{A.2}                                      & \multicolumn{1}{l|}{}                                                                                                                             & \textit{\begin{tabular}[c]{@{}l@{}}Vede che c'è un operazione\\ di creazione in sospeso\end{tabular}}                                                                                     \\ \cline{2-4} 
\cellcolor[HTML]{B2C9AB}                                                                                                                                                                         & \multicolumn{1}{l|}{A.3}                                      & \multicolumn{1}{l|}{}                                                                                                                             & \textit{\begin{tabular}[c]{@{}l@{}}Mostra Popup chiedendo se \\ vuole continuare \\ l'operazione o creare un \\ nuovo Annuncio\end{tabular}}                                              \\ \cline{2-4} 
\cellcolor[HTML]{B2C9AB}                                                                                                                                                                         & \multicolumn{1}{l|}{A.4}                                      & \multicolumn{1}{l|}{Click su Nuovo Annuncio}                                                                                                      & \textit{}                                                                                                                                                                                 \\ \cline{2-4} 
\cellcolor[HTML]{B2C9AB}                                                                                                                                                                         & \multicolumn{1}{l|}{A.5}                                      & \multicolumn{1}{l|}{}                                                                                                                             & \textit{\begin{tabular}[c]{@{}l@{}}Mostra Dialog per avvisare \\ che i dati dell Annuncio \\ in sospeso verranno persi\end{tabular}}                                                      \\ \cline{2-4} 
\cellcolor[HTML]{B2C9AB}                                                                                                                                                                         & \multicolumn{1}{l|}{A.6}                                      & \multicolumn{1}{l|}{Click Conferma}                                                                                                               & \textit{}                                                                                                                                                                                 \\ \cline{2-4} 
\multirow{-18}{*}{\cellcolor[HTML]{B2C9AB}\begin{tabular}[c]{@{}l@{}}Sistema riconosce\\ che c’è un Annuncio\\ in creazione sospeso.\\ Utente vuole creare\\ un Nuovo Annuncio\end{tabular}}      & \multicolumn{1}{l|}{A.7}                                      & \multicolumn{1}{l|}{}                                                                                                                             & \textit{\begin{tabular}[c]{@{}l@{}}Vai al passo 2 dello \\ Scenario Principale\end{tabular}}                                                                                              \\ \hline
\cellcolor[HTML]{B2C9AB}                                                                                                                                                                         & \multicolumn{1}{l|}{B.2}                                      & \multicolumn{1}{l|}{}                                                                                                                             & \textit{\begin{tabular}[c]{@{}l@{}}Vede che c'è un \\ operazione di\\ creazione in sospeso\end{tabular}}                                                                                  \\ \cline{2-4} 
\cellcolor[HTML]{B2C9AB}                                                                                                                                                                         & \multicolumn{1}{l|}{B.3}                                      & \multicolumn{1}{l|}{}                                                                                                                             & \textit{\begin{tabular}[c]{@{}l@{}}Mostra Popup chiedendo se \\ vuole continuare \\ l'operazione o creare un \\ nuovo Annuncio\end{tabular}}                                              \\ \cline{2-4} 
\cellcolor[HTML]{B2C9AB}                                                                                                                                                                         & \multicolumn{1}{l|}{B.4}                                      & \multicolumn{1}{l|}{\begin{tabular}[c]{@{}l@{}}Click su Continua\\ Operazione\\ \\ \end{tabular}}                                                                                                & \textit{}                                                                                                                                                                                 \\ \cline{2-4} 
\cellcolor[HTML]{B2C9AB}                                                                                                                                                                         & \multicolumn{1}{l|}{B.5}                                      & \multicolumn{1}{l|}{\textit{}}                                                                                                                    & \textit{\begin{tabular}[c]{@{}l@{}}Mostra la pagina\\ di creazione caricando i\\ dati inseriti nell’annuncio\\ sospeso\end{tabular}}                                                      \\ \cline{2-4} 
\multirow{-19}{*}{\cellcolor[HTML]{B2C9AB}\begin{tabular}[c]{@{}l@{}}Sistema riconosce\\ che c’è un Annuncio\\ in creazione sospeso.\\ Utente vuole continuare\\ operazione sospesa\end{tabular}} & \multicolumn{1}{l|}{B.6}                                      & \multicolumn{1}{l|}{}                                                                                                                             & \textit{\begin{tabular}[c]{@{}l@{}}Vai al passo 4 dello \\ Scenario Principale\end{tabular}}                                                                                              \\ \hline
\cellcolor[HTML]{B2C9AB}                                                                                                                                                                         & \multicolumn{1}{l|}{C.9}                                      & \multicolumn{1}{l|}{}                                                                                                                             & \textit{\begin{tabular}[c]{@{}l@{}}Mostra Messaggio di Errore \\ che avverte che l'operazione \\ non è andata buon fine e \\ che i dati sono stati \\ preservati in locale.\end{tabular}} \\ \cline{2-4} 
\cellcolor[HTML]{B2C9AB}                                                                                                                                                                         & \multicolumn{1}{l|}{C.10}                                     & \multicolumn{1}{l|}{}                                                                                                                             & \textit{Salva dati in locale.}                                                                                                                                                            \\ \cline{2-4} 
\multirow{-8}{*}{\cellcolor[HTML]{B2C9AB}\begin{tabular}[c]{@{}l@{}}Il Sistema è Offline \\ al momento della\\ Pubblicazione oppure\\ non è Collegato a\\ Internet\end{tabular}}                 & \multicolumn{1}{l|}{C.11}                                     & \multicolumn{1}{l|}{}                                                                                                                             & \textit{\begin{tabular}[c]{@{}l@{}}Chiude la schermata.\\ Use Case Terminato in \\ Fallimento.\end{tabular}}                                                                              \\ \hline
\cellcolor[HTML]{B2C9AB}                                                                                                                                                                         & \multicolumn{1}{l|}{D.10}                                     & \multicolumn{1}{l|}{}                                                                                                                             & \begin{tabular}[c]{@{}l@{}}Evidenzia campi del form\\ non riempiti/validi\end{tabular}                                                                                                    \\ \cline{2-4} 
\multirow{-4}{*}{\cellcolor[HTML]{B2C9AB}Dati Inseriti non validi}                                                                                                                               & \multicolumn{1}{l|}{D.11}                                     & \multicolumn{1}{l|}{}                                                                                                                             & \begin{tabular}[c]{@{}l@{}}Vai al passo 5 dello\\ Scenario Principale\end{tabular}                                                                                                        \\ \hline
\end{longtable}
\newpage
\subsection*{Modifica Stato Notifiche in arrivo}
% Please add the following required packages to your document preamble:
% \usepackage{multirow}
% \usepackage[table,xcdraw]{xcolor}
% Beamer presentation requires \usepackage{colortbl} instead of \usepackage[table,xcdraw]{xcolor}
% \usepackage{longtable}
% Note: It may be necessary to compile the document several times to get a multi-page table to line up properly

\subsection{Cockburn: Attivazione e Disattivazione categorie notifiche}

\begin{longtable}{|l|lll|}
\caption{}
\label{tab:my-table}\\
\hline
\rowcolor[HTML]{E1D5E7} 
\textbf{Use Case 2}                                                                                                                                                                 & \multicolumn{3}{l|}{\cellcolor[HTML]{E1D5E7}\textbf{Modifica notifiche in Arrivo}}                                                                                                                                                                                                                                                                                                                        \\ \hline
\endhead
%
\cellcolor[HTML]{E1D5E7}\textbf{Goal In Context}                                                                                                                                    & \multicolumn{3}{l|}{Cambia lo stato di notiche a scelta}                                                                                                                                                                                                                                                                                                                                                  \\ \hline
\cellcolor[HTML]{E1D5E7}\textbf{Preconditions}                                                                                                                                      & \multicolumn{3}{l|}{\begin{tabular}[c]{@{}l@{}}- Login effettuato con account con\\ ruolo Utente. \\ - Si trova nella pagina di Visualizazione notifiche\end{tabular}}                                                                                                                                                                                                                                    \\ \hline
\cellcolor[HTML]{E1D5E7}\textbf{\begin{tabular}[c]{@{}l@{}}Success End\\ Conditions\end{tabular}}                                                                                   & \multicolumn{3}{l|}{Lo stato di abilitazione di notifica viene modificato}                                                                                                                                                                                                                                                                                                                                \\ \hline
\cellcolor[HTML]{E1D5E7}\textbf{Primary Actor}                                                                                                                                      & \multicolumn{3}{l|}{Utente}                                                                                                                                                                                                                                                                                                                                                                               \\ \hline
\cellcolor[HTML]{E1D5E7}\textbf{Trigger}                                                                                                                                            & \multicolumn{3}{l|}{Utente preme il bottone Gestisci Notifiche}                                                                                                                                                                                                                                                                                                                                           \\ \hline
\rowcolor[HTML]{E1D5E7} 
\textbf{Descrizione}                                                                                                                                                                & \multicolumn{1}{l|}{\cellcolor[HTML]{E1D5E7}\textbf{Step n.}} & \multicolumn{1}{l|}{\cellcolor[HTML]{E1D5E7}\textbf{Utente}}                                                                            & \textbf{Sistema}                                                                                                                                                                                \\ \hline
\cellcolor[HTML]{E1D5E7}                                                                                                                                                            & \multicolumn{1}{l|}{1}                                        & \multicolumn{1}{l|}{}                                                                                                                   & \textit{\begin{tabular}[c]{@{}l@{}}Mostra pagina delle\\ notifiche\end{tabular}}                                                                                                                \\ \cline{2-4} 
\cellcolor[HTML]{E1D5E7}                                                                                                                                                            & \multicolumn{1}{l|}{2}                                        & \multicolumn{1}{l|}{\begin{tabular}[c]{@{}l@{}}Click Bottone per\\ modificare stato notifica\end{tabular}}                              &                                                                                                                                                                                                 \\ \cline{2-4} 
\cellcolor[HTML]{E1D5E7}                                                                                                                                                            & \multicolumn{1}{l|}{3}                                        & \multicolumn{1}{l|}{}                                                                                                                   & \textit{\begin{tabular}[c]{@{}l@{}}Mosta una Popup con \\ elenco di tutte le \\ categorie che possono \\ essere Modificate\end{tabular}}                                                        \\ \cline{2-4} 
\cellcolor[HTML]{E1D5E7}                                                                                                                                                            & \multicolumn{1}{l|}{4}                                        & \multicolumn{1}{l|}{\begin{tabular}[c]{@{}l@{}}Click su la casella/e \\ relativa/e alla categoria/e \\ da Modificare\end{tabular}}      & \textit{}                                                                                                                                                                                       \\ \cline{2-4} 
\cellcolor[HTML]{E1D5E7}                                                                                                                                                            & \multicolumn{1}{l|}{5}                                        & \multicolumn{1}{l|}{Click Bottone Conferma}                                                                                             & \textit{}                                                                                                                                                                                       \\ \cline{2-4} 
\cellcolor[HTML]{E1D5E7}                                                                                                                                                            & \multicolumn{1}{l|}{6}                                        & \multicolumn{1}{l|}{}                                                                                                                   & \textit{\begin{tabular}[c]{@{}l@{}}Mostra Dialog con \\ un messaggio che \\ avverte l'Utente \\ che non riceverà \\ più nuove Notifiche \\ relative alla \\ categoria selezionata\end{tabular}} \\ \cline{2-4} 
\cellcolor[HTML]{E1D5E7}                                                                                                                                                            & \multicolumn{1}{l|}{7}                                        & \multicolumn{1}{l|}{\begin{tabular}[c]{@{}l@{}}Click Bottone Conferma \\ nella Dialog\end{tabular}}                                     & \textit{}                                                                                                                                                                                       \\ \cline{2-4} 
\cellcolor[HTML]{E1D5E7}                                                                                                                                                            & \multicolumn{1}{l|}{8}                                        & \multicolumn{1}{l|}{}                                                                                                                   & \textit{\begin{tabular}[c]{@{}l@{}}Modifica Stato \\ Notifica in Database\end{tabular}}                                                                                                         \\ \cline{2-4} 
\cellcolor[HTML]{E1D5E7}                                                                                                                                                            & \multicolumn{1}{l|}{9}                                        & \multicolumn{1}{l|}{}                                                                                                                   & \textit{\begin{tabular}[c]{@{}l@{}}Feedback visivo del \\ cambio di visibilità\end{tabular}}                                                                                                    \\ \cline{2-4} 
\multirow{-50}{*}{\cellcolor[HTML]{E1D5E7}\textbf{\begin{tabular}[c]{@{}l@{}}Scenario \\ Principale\end{tabular}}}                                                                  & \multicolumn{1}{l|}{10}                                       & \multicolumn{1}{l|}{}                                                                                                                   & \textit{\begin{tabular}[c]{@{}l@{}}Use Case terminato \\ con Successo\end{tabular}}                                                                                                             \\ \hline
\newpage
\rowcolor[HTML]{E1D5E7} 
\textbf{Extension}                                                                                                                                                                  & \multicolumn{1}{l|}{\cellcolor[HTML]{E1D5E7}\textbf{Step n.}} & \multicolumn{1}{l|}{\cellcolor[HTML]{E1D5E7}\textbf{Utente}}                                                                            & \textbf{Sistema}                                                                                                                                                                                \\ \hline
\cellcolor[HTML]{E1D5E7}                                                                                                                                                            & \multicolumn{1}{l|}{A.2}                                      & \multicolumn{1}{l|}{click su notifica ricevuta}                                                                                         & \textit{}                                                                                                                                                                                       \\ \cline{2-4} 
\cellcolor[HTML]{E1D5E7}                                                                                                                                                            & \multicolumn{1}{l|}{A.3}                                      & \multicolumn{1}{l|}{}                                                                                                                   & \textit{\begin{tabular}[c]{@{}l@{}}Mostra il contenuto \\ della notifica\end{tabular}}                                                                                                          \\ \cline{2-4} 
\cellcolor[HTML]{E1D5E7}                                                                                                                                                            & \multicolumn{1}{l|}{A.4}                                      & \multicolumn{1}{l|}{\begin{tabular}[c]{@{}l@{}}Click Bottone Disattiva\\ Notifica\end{tabular}}                                         & \textit{}                                                                                                                                                                                       \\ \cline{2-4} 
\cellcolor[HTML]{E1D5E7}                                                                                                                                                            & \multicolumn{1}{l|}{A.5}                                      & \multicolumn{1}{l|}{}                                                                                                                   & \textit{\begin{tabular}[c]{@{}l@{}}Bottone\\ Disattiva Notifica \\ diventa Bottone \\ Attiva Notifica\end{tabular}}                                                                             \\ \cline{2-4} 

\multirow{-15}{*}{\cellcolor[HTML]{E1D5E7}\begin{tabular}[c]{@{}l@{}}Utente vuole \\ disabilitare una categoria \\ di notifiche a partire da \\ una notifica ricevuta\end{tabular}}    & \multicolumn{1}{l|}{A.6}                                      & \multicolumn{1}{l|}{}                                                                                                                   & \textit{\begin{tabular}[c]{@{}l@{}}Vai allo step 6 dello\\ Scenario Principale\end{tabular}}                                                                                                    \\ \hline
\cellcolor[HTML]{E1D5E7}                                                                                                                                                            & \multicolumn{1}{l|}{B.6}                                      & \multicolumn{1}{l|}{}                                                                                                                   & \textit{\begin{tabular}[c]{@{}l@{}}Il sistema nota che \\ non ci sono state \\ modifiche\end{tabular}}                                                                                          \\ \cline{2-4} 
\multirow{-4}{*}{\cellcolor[HTML]{E1D5E7}\begin{tabular}[c]{@{}l@{}}Utente non modifica \\ nessuna categoria \\ durante lo \\ Scenario Principale\end{tabular}}                     & \multicolumn{1}{l|}{B.7}                                      & \multicolumn{1}{l|}{}                                                                                                                   & \textit{\begin{tabular}[c]{@{}l@{}}Use case Terminato \\ con Successo\end{tabular}}                                                                                                             \\ \hline
\cellcolor[HTML]{E1D5E7}                                                                                                                                                            & \multicolumn{1}{l|}{C.2}                                      & \multicolumn{1}{l|}{\begin{tabular}[c]{@{}l@{}}Utente preme Bottone\\ Disabilita dalla lista delle\\ categorie Visibili\end{tabular}}   &                                                                                                                                                                                                 \\ \cline{2-4} 
\cellcolor[HTML]{E1D5E7}                                                                                                                                                            & \multicolumn{1}{l|}{C.3}                                      & \multicolumn{1}{l|}{}                                                                                                                   & \textit{\begin{tabular}[c]{@{}l@{}}Bottone Disabilita \\ diventa\\ Bottone Abilita\end{tabular}}                                                                                                \\ \cline{2-4} 
\multirow{-6}{*}{\cellcolor[HTML]{E1D5E7}\begin{tabular}[c]{@{}l@{}}Utente vuole disabilitare \\ categoria dalla lista \\ delle Categorie Attive\end{tabular}}                      & \multicolumn{1}{l|}{C.4}                                      & \multicolumn{1}{l|}{\textit{}}                                                                                                          & \textit{\begin{tabular}[c]{@{}l@{}}Vai al Passo 6 dello \\ Scenario Principale\end{tabular}}                                                                                                    \\ \hline
\cellcolor[HTML]{E1D5E7}                                                                                                                                                            & \multicolumn{1}{l|}{D.2}                                      & \multicolumn{1}{l|}{\begin{tabular}[c]{@{}l@{}}Utente Preme Bottone\\ Abilita dalla lista delle \\ Categorie disabilitate\end{tabular}} &                                                                                                                                                                                                 \\ \cline{2-4} 
\cellcolor[HTML]{E1D5E7}                                                                                                                                                            & \multicolumn{1}{l|}{D.3}                                      & \multicolumn{1}{l|}{}                                                                                                                   & \textit{\begin{tabular}[c]{@{}l@{}}Bottone Abilita \\ diventa\\ Bottone Disabilita\end{tabular}}                                                                                                \\ \cline{2-4} 
\multirow{-6}{*}{\cellcolor[HTML]{E1D5E7}\begin{tabular}[c]{@{}l@{}}Utente vuole abilitare\\ categoria dalla lista delle\\ Categorie Disabilitate\end{tabular}}                     & \multicolumn{1}{l|}{D.4}                                      & \multicolumn{1}{l|}{\textit{}}                                                                                                          & \textit{\begin{tabular}[c]{@{}l@{}}Vai al Passo 8 dello \\ Scenario Principale\end{tabular}}                                                                                                    \\ \hline
\cellcolor[HTML]{E1D5E7}                                                                                                                                                            & \multicolumn{1}{l|}{E.7}                                      & \multicolumn{1}{l|}{}                                                                                                                   & \textit{\begin{tabular}[c]{@{}l@{}}Mostra messaggio di \\ Errore il quale \\ avverte che \\ l'operazione non è \\ andata a buon fine.\end{tabular}}                                             \\ \cline{2-4} 
\cellcolor[HTML]{E1D5E7}                                                                                                                                                            & \multicolumn{1}{l|}{}                                         & \multicolumn{1}{l|}{}                                                                                                                   &                                                                                                                                                                                                 \\
\cellcolor[HTML]{E1D5E7}                                                                                                                                                            & \multicolumn{1}{l|}{}                                         & \multicolumn{1}{l|}{}                                                                                                                   &                                                                                                                                                                                                 \\
\multirow{-7}{*}{\cellcolor[HTML]{E1D5E7}\begin{tabular}[c]{@{}l@{}}Il sistema è Offline al \\ click del bottone \\ dell’Utente oppure \\ non è collegato a Internet.\end{tabular}} & \multicolumn{1}{l|}{\multirow{-3}{*}{E.8}}                    & \multicolumn{1}{l|}{\multirow{-3}{*}{}}                                                                                                 & \multirow{-3}{*}{\textit{\begin{tabular}[c]{@{}l@{}}Use Case Terminato \\ in Fallimento.\end{tabular}}}                                                                                         \\ \hline
\newpage
\cellcolor[HTML]{E1D5E7}                                                                                                                                                            & \multicolumn{1}{l|}{\cellcolor[HTML]{FFFFFF}F.2}              & \multicolumn{1}{l|}{Click su notifica ricevuta}                                                                                         & \textit{}                                                                                                                                                                                       \\ \cline{2-4} 
\cellcolor[HTML]{E1D5E7}                                                                                                                                                            & \multicolumn{1}{l|}{\cellcolor[HTML]{FFFFFF}F.3}              & \multicolumn{1}{l|}{}                                                                                                                   & \textit{\begin{tabular}[c]{@{}l@{}}Mostra il contenuto \\ della notifica\end{tabular}}                                                                                                          \\ \cline{2-4} 
\cellcolor[HTML]{E1D5E7}                                                                                                                                                            & \multicolumn{1}{l|}{\cellcolor[HTML]{FFFFFF}F.4}              & \multicolumn{1}{l|}{\begin{tabular}[c]{@{}l@{}}Click Bottone Attiva\\ Notifica\end{tabular}}                                            &                                                                                                                                                                                                 \\ \cline{2-4} 
\cellcolor[HTML]{E1D5E7}                                                                                                                                                            & \multicolumn{1}{l|}{F.5}                                      & \multicolumn{1}{l|}{}                                                                                                                   & \textit{\begin{tabular}[c]{@{}l@{}}Bottone \\ Attiva Notifiche\\ diventa Bottone \\ Disabilita Notifiche\end{tabular}}                                                                          \\ 

\cline{2-4} 
\multirow{-15}{*}{\cellcolor[HTML]{E1D5E7}\begin{tabular}[c]{@{}l@{}}Utente vuole abilitare \\ una categoria di \\ notifiche a partire \\ da una notica ricevuta\end{tabular}}       & \multicolumn{1}{l|}{\cellcolor[HTML]{FFFFFF}F.6}              & \multicolumn{1}{l|}{\cellcolor[HTML]{FFFFFF}\textbf{}}                                                                                  & \textit{\begin{tabular}[c]{@{}l@{}}Vai al Passo 8 dello \\ Scenario Principale\end{tabular}}                                                                                                    \\ \hline
\end{longtable}
% punto e.II)
\newpage
\section{Mock up}
\subsection*{Introduzione}
Dopo aver definito gli use case utilizzando il metodo Cockburn, siamo passati alla fase di progettazione visiva realizzando una serie di mockup interattivi in Figma. Questi mockup hanno lo scopo di rappresentare graficamente le interazioni dell'utente con il sistema, traducendo le specifiche funzionali in un'interfaccia visibile e navigabile. Ogni mockup è stato sviluppato tenendo conto delle diverse casistiche previste negli use case e nelle loro estensioni, in modo da garantire un’esperienza utente coerente e intuitiva.

Nei paragrafi seguenti verranno presentati i vari mockup, suddivisi per use case. Ogni sezione illustrerà le scelte progettuali effettuate, evidenziando come la UI si adatti alle diverse situazioni previste dai casi d'uso.

\subsection{Caso d'Uso: Aggiungi un Annuncio Immobiliare}

Per garantire un’esperienza utente ottimale, abbiamo progettato i mockup del caso d'uso \textit{Aggiungi un Annuncio Immobiliare} seguendo i principi della user experience (UX) e dell'usabilità. Il percorso utente è stato studiato attentamente per minimizzare il carico cognitivo e semplificare l’interazione, in linea con i modelli proposti da Nielsen e Norman \cite{nielsen1995,norman1988}.

\subsection*{Schermata Iniziale: Gestione Annunci Immobiliari}
L’azione di aggiungere un nuovo annuncio parte dalla schermata di Gestione Annunci Immobiliari, dove l’utente trova:
\begin{itemize}
    \item \textbf{Una lista degli Immobili a lui associati}, che consente una rapida contestualizzazione.
    \item \textbf{Un pulsante unico ed evidente}, etichettato “Aggiungi Annuncio Immobiliare”, che centralizza l’azione primaria.
\end{itemize}

\subsection*{Flusso di Navigazione e Segmentazione del Form}
Al click del pulsante, l’utente viene indirizzato a una schermata dedicata alla compilazione di un form articolato in più step. Questa segmentazione si basa sul principio del \textbf{chunking dell’informazione} \cite{miller1956}, riducendo il carico di memoria e rendendo il processo meno gravoso. In particolare:
\begin{itemize}
    \item \textbf{Step 1:} Richiede le informazioni basilari, quali il titolo, il tipo di contratto e il tipo di immobile.
    \item \textbf{Step 2 e successivi:} Raccolgono i dati principali e le caratteristiche secondarie dell’immobile, organizzati in modo logico e intuitivo.
\end{itemize}

\subsection*{Navigazione Sticky e Accessibilità degli Step}
I pulsanti di navigazione, posizionati in alto con comportamento \textit{sticky}, consentono all’utente di passare agevolmente da uno step all’altro, anche in presenza di schermate particolarmente lunghe. Tale scelta:
\begin{itemize}
    \item Riduce il tempo necessario per individuare i controlli di navigazione.
    \item Favorisce un’interazione continua e fluida, in linea con i principi di design centrato sull’utente e le best practice in ambito HCI (Human-Computer Interaction) \cite{shneiderman2004}.
\end{itemize}

\subsection*{Integrazione della Mappa Interattiva}
Per quanto riguarda l’inserimento dell’indirizzo:
\begin{itemize}
    \item \textbf{La mappa interattiva} permette di visualizzare immediatamente la posizione indicata, offrendo un feedback visivo diretto.
    \item L’utente ha la possibilità di regolare la posizione direttamente sulla mappa, migliorando la precisione del dato inserito \cite{wickens2008}.
\end{itemize}

\subsection*{Gestione delle Immagini e Feedback Visivo}
La sezione dedicata alle immagini è progettata per:
\begin{itemize}
    \item \textbf{Consentire l’aggiunta di foto} tramite pulsante dedicato o drag and drop, facilitando l’upload in maniera intuitiva.
    \item \textbf{Permettere l’inserimento di una descrizione} per ogni immagine, migliorando la contestualizzazione visiva dell’annuncio \cite{pieters2004}.
\end{itemize}

\subsection*{Anteprima e Conferma Finale}
Infine, viene presentato uno schema riepilogativo che funge da anteprima dell’annuncio così come apparirà agli utenti finali. Una volta verificata la correttezza dei dati:
\begin{itemize}
    \item L’utente può cliccare il pulsante \textbf{“Pubblica”}, che innesca un processo di caricamento e validazione.
    \item Al termine del processo, viene mostrato un messaggio di conferma che attesta il successo dell’operazione, riducendo l’ansia da incertezza e rafforzando la fiducia nel sistema \cite{nielsen1995}.
\end{itemize}

\newpage

\input{Requisiti Del Software/Analisi dei Requisiti/Mockup/aggiungi annuncio/scenario principale}

\input{Requisiti Del Software/Analisi dei Requisiti/Mockup/aggiungi annuncio/scenario principale Parte II}

\input{Requisiti Del Software/Analisi dei Requisiti/Mockup/aggiungi annuncio/estensione A}

\input{Requisiti Del Software/Analisi dei Requisiti/Mockup/aggiungi annuncio/estensione B}

\input{Requisiti Del Software/Analisi dei Requisiti/Mockup/aggiungi annuncio/estensione C}

\input{Requisiti Del Software/Analisi dei Requisiti/Mockup/aggiungi annuncio/estensione D}

\clearpage
\newpage

\subsection{Caso d'Uso: Attivazione e Disattivazione Notifiche}

Il sistema prevede un'interfaccia dedicata alla gestione delle preferenze di notifica, pensata per permettere all'utente di curare la propria esperienza d'uso dell'applicazione, ricevendo news solo sugli argomenti di interesse.\\
In questa sezione andremo a esaminare le scelte effettuate durante la progettazione, l'impatto sull'esperienza utente e i prototipi generati alla fine dell'analisi. 

\vspace{0.5cm}
\subsubsection{Tipologie di Notifiche e Controllo Utente}
Le notifiche sono suddivise in diverse categorie per permettere una personalizzazione granulare:
\begin{itemize}
    \item \textbf{Annunci di nuovi Immobili}: notifiche basate sulle ricerche dell’utente.
    \item \textbf{Risposte alle offerte}: aggiornamenti sulle interazioni con gli annunci pubblicati.
    \item \textbf{Messaggi promozionali}: comunicazioni di marketing e offerte esclusive.
\end{itemize}
L’utente può, in qualsiasi momento, disattivare le notifiche per una o più categorie, mantenendo un controllo totale sulla propria esperienza \cite{shneiderman2004}.

\vspace{0.5cm}
\subsubsection{Gestione delle Notifiche e Conferma delle Modifiche}
La gestione delle notifiche avviene principalmente attraverso la schermata delle notifiche, composta da:
\begin{itemize}
    \item \textbf{Lista delle notifiche ricevute}, ognuna cliccabile per visualizzare i dettagli.
    \item \textbf{Barra laterale con le categorie di notifiche}, suddivise in:
    \begin{itemize}
        \item \textbf{Categorie attive}, con notifiche attualmente abilitate.
        \item \textbf{Categorie disattivate}, che non inviano più notifiche.
    \end{itemize}
\end{itemize}

In cima alla barra laterale è presente un’icona che, se cliccata, apre una schermata popup intitolata “Attiva e Disattiva Notifiche”. All’interno, ogni categoria è rappresentata da un toggle switch che indica lo stato attuale delle notifiche.

\vspace{0.5cm}
\subsubsection{Feedback Visivo e Animazioni Intuitive}
Per garantire un’interazione chiara e immediata, il sistema utilizza diverse tecniche di UX design:
\begin{itemize}
    \item \textbf{Animazione di transizione}: quando un toggle viene modificato, la categoria si sposta visivamente tra la sezione attiva e quella disattiva, sfruttando il principio di \textbf{gestalt della continuità} \cite{miller1956} per rendere il cambiamento intuitivo.
    \item \textbf{Feedback visivo immediato}: l’utente percepisce immediatamente l’effetto dell’azione senza necessità di un testo esplicativo eccessivo.
\end{itemize}

\newpage
\subsubsection{Conferma e Implicazioni della Disattivazione}
Per evitare errori accidentali e garantire consapevolezza delle conseguenze, la disattivazione di una categoria di notifiche è accompagnata da:
\begin{itemize}
    \item Un popup di conferma che informa l’utente che, durante il periodo in cui le notifiche sono disattivate, le notifiche non potranno essere recuperate \cite{wickens2008}.
    \item Un ulteriore messaggio di avviso prima della conferma definitiva, in linea con le \textbf{heuristiche di usabilità di Nielsen} \cite{nielsen1995} per la prevenzione degli errori.
\end{itemize}

Solo dopo la conferma finale, il sistema applica le modifiche alle preferenze dell'utente, garantendo un'interazione consapevole e trasparente.\\
Nel prototipo questi comportamenti sono stati modellati con un pulsante, tuttavia nell'applicazione finale è stato deciso di sostituirlo con un menù contestuale.



\input{Requisiti Del Software/Analisi dei Requisiti/Mockup/disattivazione notifiche/Scenario principale}

\input{Requisiti Del Software/Analisi dei Requisiti/Mockup/disattivazione notifiche/estensione A}
\input{Requisiti Del Software/Analisi dei Requisiti/Mockup/disattivazione notifiche/estensione B}

\input{Requisiti Del Software/Analisi dei Requisiti/Mockup/disattivazione notifiche/estensione c}

\input{Requisiti Del Software/Analisi dei Requisiti/Mockup/disattivazione notifiche/estensione D}

\input{Requisiti Del Software/Analisi dei Requisiti/Mockup/disattivazione notifiche/estensione E}

\input{Requisiti Del Software/Analisi dei Requisiti/Mockup/disattivazione notifiche/estensione F}

\clearpage
\newpage

\section{Test di usabilità}

I \textbf{test di usabilità} rappresentano una fase essenziale nel processo di sviluppo di un’interfaccia utente, consentendo di valutare l'efficacia, l’efficienza e la soddisfazione dell’utente nell’interazione con il sistema. In particolare, l'uso di mockup interattivi offre la possibilità di raccogliere feedback sulle scelte di design prima ancora della fase di sviluppo, riducendo i costi di eventuali revisioni e migliorando la qualità dell’esperienza utente.
\newline
L'obiettivo principale di questi test è identificare eventuali problemi di navigazione, ambiguità nelle interazioni o difficoltà nella comprensione delle funzionalità, al fine di ottimizzare l'interfaccia prima del rilascio definitivo. 

\vspace{0.5cm} % Aggiunge spazio prima della sezione

\textbf{1. Tempo medio per completare un Task}
\begin{equation}
T_{\text{medio}} = \frac{\sum_{i=1}^{n} T_i}{n}
\end{equation}
dove \( T_i \) è il tempo impiegato dall'utente \( i \) per completare il task e \( n \) è il numero totale di utenti.

\vspace{0.5cm} % Aggiunge spazio prima della sezione
\textbf{2. Tasso di completamento di un Task}
\begin{equation}
T_{\text{completamento}} = \frac{U_{\text{completati}}}{U_{\text{totali}}} \times 100
\end{equation}
dove \( U_{\text{completati}} \) è il numero di utenti che hanno completato il task e \( U_{\text{totali}} \) è il numero totale di utenti che hanno provato il task.

\vspace{0.5cm} % Aggiunge spazio prima della sezione
\textbf{3. Errore Medio per Task}
\begin{equation}
E_{\text{medio}} = \frac{\sum_{i=1}^{n} E_i}{n}
\end{equation}
dove \( E_i \) è il numero di errori commessi dall'utente \( i \) durante il task.

\vspace{0.5cm} % Aggiunge spazio prima della sezione
\textbf{4. Customer Satisfaction Score}
\begin{equation}
CSAT = \frac{\sum_{i=1}^{n} S_i}{n} \times 100
\end{equation}
dove \( S_i \) è il punteggio di soddisfazione dato dall'utente \( i \) e \( n \) è il numero totale di risposte raccolte.

\vspace{0.5cm} % Aggiunge spazio prima della sezione
\textbf{5. System Usability Scale (SUS)}
\newline
\newline
Il \textbf{System Usability Scale (SUS)} è un metodo standardizzato, introdotto da \textbf{John Brooke} nel 1986, utilizzato per misurare l’usabilità di un prodotto attraverso un questionario composto da \textbf{10 affermazioni}. Gli utenti rispondono utilizzando una \textbf{scala Likert a 5 punti}, esprimendo il loro grado di accordo o disaccordo. Il punteggio complessivo, che varia da 0 a 100, fornisce una misura quantitativa dell’usabilità percepita, consentendo di confrontare i risultati con benchmark consolidati.

\vspace{0.5cm} % Aggiunge spazio prima della sezione
\subsubsection{Questionario relativo al mockup "Creazione nuovo annuncio"}

Di seguito presentiamo il questionario utilizzato per valutare il mockup del processo di creazione di un annuncio immobiliare. Ogni domanda prevede cinque opzioni di risposta:

\begin{itemize}
    \item Per niente d’accordo
    \item Poco d’accordo
    \item Né d’accordo né in disaccordo
    \item Abbastanza d’accordo
    \item Molto d’accordo
\end{itemize}
Ai fini del calcolo del punteggio \textbf{SUS}, alle risposte viene assegnato un valore da \textbf{1} a \textbf{5}, dove la prima opzione corrisponde a 1 punto e l’ultima a 5 punti.

\vspace{0.5cm} % Aggiunge spazio prima della sezione

\begin{enumerate}
    \item \textbf{La compilazione dei campi divisi in step ha reso il processo di creazione più intuitivo e meno stancante.}
    \newline
    \texttt{Scopo}: Verificare se la suddivisione del form in più passaggi migliora l’esperienza utente, evitando un sovraccarico cognitivo e rendendo la compilazione più fluida.

    \item \textbf{Il bottone “Avanti” è posizionato in modo poco visibile e difficile da individuare.}
    \newline
    \texttt{Scopo}: Valutare la visibilità e l’intuitività del pulsante che consente di procedere nella compilazione, elemento essenziale per garantire una navigazione chiara e senza interruzioni.

    \item \textbf{I campi a scelta (non liberi) mi hanno aiutato a capire meglio il tipo di informazioni richieste e a velocizzare la compilazione.}
    \newline
    \texttt{Scopo}: Analizzare se l’uso di menu a tendina o opzioni predefinite aiuta a ridurre incertezze, errori e tempi di compilazione rispetto ai campi di testo libero.

    
    \item \textbf{Il numero di campi richiesti è eccessivo o insufficiente, rendendo il form poco bilanciato.}
    \newline
    \texttt{Scopo}: Ottenere un feedback sulla quantità di informazioni richieste, bilanciando completezza e semplicità d’uso, evitando di rendere il processo troppo lungo o complesso.

    \item \textbf{I colori utilizzati sono gradevoli e non affaticano la lettura.}
    \newline
    \texttt{Scopo}: Valutare se la combinazione di colori scelta favorisce una buona leggibilità e un’esperienza visiva piacevole, senza creare affaticamento visivo.

    \item \textbf{L’importazione e la gestione delle foto dell’annuncio risultano poco intuitive.}
    \newline
    \texttt{Scopo}: esaminare la semplicità e l’efficacia del meccanismo di caricamento delle immagini, una funzionalità chiave nella creazione di annunci immobiliari.

    \item \textbf{Trovo utile l’anteprima dell’annuncio prima di confermare la pubblicazione.}
    \newline
    \texttt{Scopo}: Capire se l’anteprima fornisce valore aggiunto agli utenti, permettendo loro di controllare e correggere eventuali errori prima della pubblicazione definitiva.

    \item \textbf{L’avviso in caso di un annuncio in sospeso è poco chiaro o inutile.}
    \newline
    \texttt{Scopo}: Testare la comprensibilità e l’efficacia del messaggio di avviso per evitare che l’utente perda dati o crei annunci duplicati.

    \item \textbf{In caso di errore (compilazione errata o parziale), i messaggi di errore sono posizionati bene e mi aiutano a capire rapidamente dove ho sbagliato.}
    \newline
    \texttt{Scopo}: Valutare la chiarezza dei messaggi di errore e il loro posizionamento, per garantire che l’utente possa correggere facilmente eventuali problemi.

    \item \textbf{Nel complesso, non sono soddisfatto dell’esperienza di creazione di un annuncio.}
    \newline
    \texttt{Scopo}: Raccogliere un’indicazione generale sul livello di soddisfazione dell’utente rispetto all’intero processo, fornendo una misura qualitativa dell’usabilità percepita.
    
\end{enumerate}

\vspace{0.5cm} % Aggiunge spazio prima della sezione
\subsubsection{Risultato SUS ottenuto attraverso i feedback raccolti}
Il System Usability Scale (SUS) viene calcolato seguendo questi passaggi:

\begin{enumerate}

    \item \textbf{Assegnazione punteggio}
    \newline 
    Ogni domanda viene valutata su una scala da 1 a 5:

    \begin{itemize}
        \item 1 = Per niente d’accordo
        \item 2 = Poco d’accordo
        \item 3 = Né d’accordo né in disaccordo
        \item 4 = Abbastanza d’accordo
        \item 5 = Molto d’accordo
    \end{itemize}
    Le 10 domande del questionario SUS sono di due tipi:
    \begin{itemize}
        \item \textbf{Domande dispari (1, 3, 5, 7, 9):} indicano usabilità positiva
        
        \item \textbf{Domande pari (2, 4, 6, 8, 10):} indicano usabilità negativa
    \end{itemize}

    \item \textbf{Calcolo del punteggio per ogni domanda}
    
    \begin{itemize}
        \item \textbf{Per le domande dispari:} Punteggio=(Risposta-1)
        
        \item \textbf{Per le domande pari:} Punteggio=(5-Risposta)
    \end{itemize}

    \item \textbf{Sommiamo tutti i punteggi ottenuti}
    \newline
    Otteniamo un punteggio complessivo che va da 0 a 40.

    \item \textbf{Moltiplichiamo per 2,5}: SUS=(somma dei punteggi)×2.5.
    \newline
    Il punteggio finale sarà compreso tra 0 e 100, ma non rappresenta una percentuale.
    
\end{enumerate}

\textbf{Interpretazione del punteggio SUS}

\begin{itemize}
    \item \textbf{Sopra 80} → Ottima usabilità
    \item \textbf{Tra 70 e 80} → Buona usabilità
    \item \textbf{Tra 50 e 70} → Accettabile, ma migliorabile
    \item \textbf{Sotto 50} → Problemi di usabilità significativi
\end{itemize}
Il questionario è stato compilato da quattro agenti immobiliari, i principali attori di questo caso d’uso. Ciascun agente proviene da un’agenzia immobiliare diversa e appartiene a una fascia d’età differente, al fine di garantire un campione più eterogeneo e realistico. 

\vspace{0.5cm}

\textbf{Utente 1:}
\begin{itemize}
    \item \textbf{Risposte a domande dispari}
    \begin{itemize}
        \item Domanda 1: molto d'accordo = 5-1 = 4
        \item Domanda 3: molto d'accordo = 5-1 = 4
        \item Domanda 5: molto d'accordo = 5-1 = 4
        \item Domanda 7: molto d'accordo = 5-1 = 4
        \item Domanda 9: molto d'accordo = 5-1 = 4
    \end{itemize}
    \item \textbf{Risposte a domande pari}
    \begin{itemize}
        \item Domanda 2: per niente d'accordo = 5-1 = 4
        \item Domanda 4: poco d'accordo = 5-2 = 3
        \item Domanda 6: poco d'accordo = 5-2 = 3
        \item Domanda 8: per niente d'accordo = 5-1 = 4
        \item Domanda 10:per niente d'accordo = 5-1 = 4
    \end{itemize}

    \item \textbf{Totale punteggio utente 1: } 38*2.5 = \textbf{95}
    
\end{itemize}
\textbf{Utente 2:}
\begin{itemize}
    \item \textbf{Risposte a domande dispari}
    \begin{itemize}
        \item Domanda 1: molto d'accordo = 5-1 = 4
        \item Domanda 3: molto d'accordo = 5-1 = 4
        \item Domanda 5: molto d'accordo = 5-1 = 4
        \item Domanda 7: abbastanza d'accordo = 4-1 = 3
        \item Domanda 9: molto d'accordo = 5-1 = 4
    \end{itemize}
    \item \textbf{Risposte a domande pari}
    \begin{itemize}
        \item Domanda 2: per niente d'accordo = 5-1 = 4
        \item Domanda 4: per niente d'accordo = 5-1 = 4
        \item Domanda 6: poco d'accordo = 5-2 = 3
        \item Domanda 8:né d'accordo né in disaccordo = 5-3 = 2
        \item Domanda 10:per niente d'accordo = 5-1 = 4
    \end{itemize}

    \item \textbf{Totale punteggio utente 2: } 36*2.5 = \textbf{90}
    
\end{itemize}
\textbf{Utente 3:}
\begin{itemize}
    \item \textbf{Risposte a domande dispari}
    \begin{itemize}
        \item Domanda 1:abbastanza d'accordo = 4-1 = 3
        \item Domanda 3: molto d'accordo = 5-1 = 4
        \item Domanda 5: molto d'accordo = 5-1 = 4
        \item Domanda 7: molto d'accordo = 5-1 = 4
        \item Domanda 9: molto d'accordo = 5-1 = 4
    \end{itemize}
    \item \textbf{Risposte a domande pari}
    \begin{itemize}
        \item Domanda 2: poco d'accordo = 5-2 = 3
        \item Domanda 4: poco d'accordo= 5-2 = 3
        \item Domanda 6: per niente d'accordo = 5-1 = 4
        \item Domanda 8: poco d'accordo = 5-2 = 3
        \item Domanda 10: per niente d'accordo = 5-1 = 4
    \end{itemize}

    \item \textbf{Totale punteggio utente 3: } 36*2.5 = \textbf{90}
    
\end{itemize}
\textbf{Utente 4:}
\begin{itemize}
    \item \textbf{Risposte a domande dispari}
    \begin{itemize}
        \item Domanda 1: molto d'accordo = 5-1 = 4
        \item Domanda 3: molto d'accordo = 5-1 = 4
        \item Domanda 5: né d'accordo né in dissacordo = 3-1 = 2
        \item Domanda 7: molto d'accordo = 5-1 = 4
        \item Domanda 9: molto d'accordo = 5-1 = 4
    \end{itemize}
    \item \textbf{Risposte a domande pari}
    \begin{itemize}
        \item Domanda 2: poco d'accordo = 5-2 = 3
        \item Domanda 4: né 'accordo né in disaccordo = 5-3 = 2
        \item Domanda 6:per niente d'accordo = 5-1 = 4
        \item Domanda 8: per niente d'accordo = 5-1 = 4
        \item Domanda 10:per niente d'accordo = 5-1 = 4
    \end{itemize}

    \item \textbf{Totale punteggio utente 4: } 35*2.5= \textbf{87.5}
    
\end{itemize}
Media punteggio SUS = 95+90+90+87.5/4 = \textbf{90.62}. Il punteggio è nettamente superiore a 80 il che indica un \textbf{ottima usabilità}.




% 3.Documento di Design del sistema
\chapter{Design del Sistema}
\section{Descrizione dell'architettura}

Il sistema è basata su un architettura \textbf{client-server}, questo stile architetturale agevola \textbf{scalabilità}, \textbf{manutenibilità} e \textbf{flessibilità}.
La comunicazione tra client e server è implementata secondo lo stile architetturale \textbf{REST}, mediante l’utilizzo di una \textbf{RESTful API}.
\\
Ricordiamo che un \textbf{API (application programming interface)}, sono \textbf{set
di definizioni e protocolli} con cui vengono realizzati e integrati software applicativi. Talvolta le si definisce come un contratto tra un fornitore di informazioni e l’utente destinatario
di tali dati: l’API stabilisce il contenuto richiesto dal consumatore (la chiamata) e il contenuto
richiesto dal produttore (la risposta). In un’architettura client-server, come quella adottata in
questo sistema, il Server funge da produttore di dati e servizi, mentre il Client agisce come
consumatore, richiedendo e utilizzando tali risorse tramite le API esposte dal Server.
\\
Una \textbf{REST (Representational state transfer)} è invece un insieme di limitazioni architetturali, non si tratta di un protocollo o uno standard. 
Esso segue i seguenti principi:

\begin{itemize}
	\item \textbf{Client-Server}: Le responsabilità devono essere separate. I server non si devono fare carico dell'interfaccia grafica o dello stato dell'utente e il client non si deve preoccupare del salvataggio delle informazioni, che rimangono all'interno dei singoli server.
	\item \textbf{Stateless}: La comunicazione client–server è vincolata in modo che nessun contesto client venga memorizzato sul server tra le richieste. Ciascuna richiesta dai vari client contiene tutte le informazioni necessarie per richiedere il servizio e lo stato della sessione è contenuto nel client.
	\item \textbf{Cacheable}: I client possono mettere in cache le risposte. Queste devono in ogni modo definirsi cacheable o no, in modo da prevenire che i client possano riutilizzare stati vecchi e dati errati. Tale caratteristica migliora scalabilità e prestazioni.
	\item \textbf{Layered system}: Il sistema può essere stratificati su più livelli, ad esempio, pubblicare le API in un server, memorizzare i dati in un secondo server e gestire l'autenticazione delle richieste in un terzo server.
	\item \textbf{Uniform interface}: Un'interfaccia di comunicazione omogenea tra client e server.
\end{itemize}

Un concetto importante in REST è l'esistenza di \textbf{risorse (fonti di informazioni)}, a cui si può accedere tramite un \textbf{identificatore globale (un URI)}. Per utilizzare le risorse, le componenti (client e server) comunicano attraverso un'interfaccia standard \textbf{(per esempio HTTP)} per scambiare rappresentazioni di queste risorse, ovvero il documento che trasmette le informazioni. 
L'applicazione deve conoscere il formato dell'informazione restituita, ovvero la sua rappresentazione. Tipicamente è un documento \textbf{HTML}, \textbf{XML} o \textbf{JSON}, ma possono essere anche immagini o altri contenuti.
\\
Una \textbf{RESTful API} Indica un’API che segue rigorosamente i principi REST quindi:

\begin{itemize}
	\item è \textbf{stateless}
	\item usa i metodi HTTP (\colorbox{lightgray}{GET}, \colorbox{lightgray}{POST}, \colorbox{lightgray}{PUT}, \colorbox{lightgray}{DELETE}, ecc.) secondo la loro semantica;
	\item espone risorse identificate da \textbf{URI} chiari (es. \colorbox{lightgray}{/users/1});
	\item usa rappresentazioni (JSON, XML, ecc.) per trasferire lo stato;
	\item impiega correttamente status code HTTP.
\end{itemize}
\section{Descrizione e motivazioni delle tecnologie adottate }
In questa sezione vengono presentate le tecnologie adottate per l’implementazione delle componenti client e server del sistema.
Inoltre, vengono descritte le architetture specifiche utilizzate nel back-end, in relazione alle tecnologie impiegate, e quelle adottate nel front-end.

\subsection{Tecnologie e architettura backend}
L'architettura del nostro server è stata progettata per rispondere alle esigenze richieste dal cliente, dove è stata prioritizzata la velocità di sviluppo e deployment senza sacrificare robustezza e sicurezza. Abbiamo adottato un'architettura \textbf{monolitica} basata su una struttura a livelli, che facilita la separazione delle responsabilità e garantisce una maggiore manutenibilità nel tempo. La scelta di un'architettura monolitica si giustifica con la necessità di ridurre la complessità iniziale e non rallentare lo sviluppo, mantenendo comunque una struttura modulare che potrà essere facilmente evoluta nel tempo, ad esempio, trasformando parti del sistema in microservizi qualora il progetto dovesse espandersi in futuro.
\\
Le tecnologie selezionate per lo sviluppo del backend sono state scelte in base a una combinazione di familiarità e facilità d'uso, con l'obiettivo di accelerare il processo di implementazione e garantire al contempo solidità e qualità del codice.
\\
Il back-end è stato sviluppato utilizzando \textbf{Spring Boot}, abbiamo optato per questo framework grazie alle sue numerose configurazioni predefinite e utili estensioni come \textbf{Lombok}, che ci permettono di concentrarci sulla logica di business anziché su complesse configurazioni di sistema o codice ripetuto. Spring Boot consente di avviare rapidamente un'applicazione web robusta e scalabile.
esso sono state integrate diverse tecnologie complementari, tra cui:

\begin{itemize}
	
	\item \textbf{Hibernate}: Per la gestione della persistenza, Hibernate si rivela una scelta efficace, in quanto astrae le operazioni di accesso al database e riduce significativamente il lavoro manuale nella scrittura di query SQL. Questo approccio rende il codice più leggibile e facilmente mantenibile.
	
	\item \textbf{Spring Security e JWT (JSON Web Token)}: Per garantire un'autenticazione stateless e sicura, abbiamo deciso di utilizzare i JWT. Questa soluzione permette di gestire le sessioni degli utenti in maniera efficiente, differenziando eventuali ruoli o permessi e riducendo il carico sul server.
	
	\item \textbf{Swagger}: L’adozione di Swagger per la documentazione delle REST API agevola notevolmente il testing e la verifica degli endpoint, creando al contempo una documentazione ricca di informazioni utili per condividere informazioni critiche.
	
	\item \textbf{Azure Blob Storage}: È stato utilizzato il servizio Azure Blob Storage per l’archiviazione delle immagini associate agli annunci. Questo approccio consente di memorizzare file di grandi dimensioni all’esterno del database e del server applicativo, migliorando così le prestazioni e l’efficienza complessiva del sistema.
	
	\item \textbf{GEO-API}: Per recuperare informazioni sui punti di interesse situati nei pressi di un immobile come, scuole, parcheggi o trasporti pubblici, è stata utilizzata un’API geografica gratuita. Essa consente, a partire dalle coordinate geografiche dell’annuncio e da una parola chiave rappresentante il tipo di luogo desiderato, di ottenere i relativi risultati in modo semplice ed efficient
	
\end{itemize}

Il back-end del sistema segue il pattern architetturale \textbf{MVC (Model-View-Controller)}, nativamente supportato dal framework \textbf{Spring Boot}.
Nel nostro caso, il \textbf{Model} è rappresentato dalle entità e dalla logica di business implementata nei servizi; il \textbf{Controller} è costituito dai componenti RESTful che gestiscono e mappano le richieste HTTP.
La \textbf{View} non è gestita all’interno del back-end, poiché l’applicazione svolge il ruolo di \textbf{API REST}, restituendo risposte in formato JSON al client, che si occupa della rappresentazione dei dati.
Questo approccio garantisce una chiara \textbf{separazione delle responsabilità}, migliorando la \textbf{modularità} e la \textbf{manutenibilità} del sistema.
\\ \\
Di seguito viene riportato uno schema del design utilizzato del backend

\begin{figure}[H]
	\centering
	\includegraphics[width=1\linewidth]{Immagini/Schema backend.png}
	\caption[schema backend]{Schema sintetico dell'architettura del backend}
\end{figure}

\begin{itemize}
	
	\item \textbf{MODEL}: Il Modello rappresenta i dati e la logica di business dell’applicazione, indipendentemente da come questi dati vengano visualizzati. In un’architettura MVC, il modello non deve dipendere dalle viste né dai controller, ma deve fornire metodi che consentano ai \textbf{controller} di manipolare i dati in risposta alle richieste dell’utente. Nel contesto del
	nostro sistema, il package \textbf{Service} all’interno del Model offre i metodi necessari per la 22 manipolazione dei dati relativi alle entità. Queste entità, grazie alle \textbf{funzionalità di Spring Boot}, sono mappate automaticamente alle relative tabelle nel database, facilitando l’interazione con i dati persistenti. L’accesso ai dati avviene tramite il \textbf{repository}, che permette di recuperare le informazioni direttamente dal database.
	
	\item \textbf{CONTROLLER}:  Il Controller svolge un ruolo fondamentale nell’architettura MVC come \textbf{ponte tra il Model e la View}. In particolare, gestisce le interazioni dell’utente, elabora le richieste e aggiorna sia il Model che la View in base alle azioni dell’utente.
	Quando un utente interagisce con l’interfaccia (la View), il Controller riceve l’input, elabora i dati (eventualmente tramite il Model, che contiene la logica di business) e quindi
	aggiorna la View per riflettere i cambiamenti dello stato. Nel nostro sistema, che segue l’architettura \textbf{RESTful API}, l’input dell’utente è rappresentato dalle richieste \textbf{HTTP (come POST, GET, PUT, DELETE)}. Ogni Controller gestisce le richieste relative a una specifica entità (ad esempio, UserController, AgenziaImmobiliareController, ecc.) e quindi si occupa di ricevere e rispondere a queste richieste. Una particolarità del nostro sistema è l’uso di \textbf{DTO (Data Transfer Object)}, che funge da \textbf{strato di filtraggio} tra la logica di business (Model) e le risposte al client. Le DTO vengono utilizzate per limitare i dati che vengono passati nelle richieste e nelle risposte. In altre parole, quando un client invia una richiesta al Controller o quando il Controller restituisce una risposta, la DTO permette di selezionare solo le informazioni necessarie, evitando di esporre direttamente l’intera entità. Questo approccio migliora la \textbf{sicurezza} e l’\textbf{efficienza}, poiché impedisce di passare dati non necessari e riduce il rischio di esposizione di informazioni sensibili. Per esempio, se un’entità Utente ha campi come id, nome, email, password, il Controller potrebbe rispondere con una DTO che include solo i campi nome ed email, evitando di restituire campi sensibili come la password. In questo modo, solo le informazioni realmente necessarie per il client vengono trasmesse.
	
	\item \textbf{VIEW}: La View è responsabile della \textbf{rappresentazione grafica dei dati} e dell’interfaccia utente. Poiché il nostro back-end è un’architettura \textbf{API RESTful}, la View viene gestita interamente dal client.
	
\end{itemize}

Il sistema include ulteriori pacchetti che contengono classi di configurazioni o classi di ausilio: 

\begin{itemize}
	
	\item In \textbf{config} troviamo le classi necessari per la configurazione del framework e la gestione della sicurezza, come l’autenticazione, essenziale per garantire l’accesso controllato agli end-point e il corretto esito delle risposte
	HTTP. Ad esempio, gli end-point che iniziano con \colorbox{lightgray}{/pb} sono configurati come accessibili liberamente, senza necessità di autenticazione. Questa impostazione viene definita nel file \textbf{SecurityConfiguration}, situato all’interno.
	
	\item In \textbf{exception} ci sono le classi che estendono RunTimeException.
	Ogni eccezione personalizzata rappresenta uno specifico stato di errore HTTP restituito nella risposta. La classe \colorbox{lightgray}{ExceptionManagement} gestisce tutte le eccezioni personalizzate, mappandole sul corretto codice di errore HTTP al momento del loro lancio.
	
	\item In \textbf{utils} contiene classi che offrono metodi statici di utilità riutilizzabili dai componenti del livello service. 
	In particolare \colorbox{lightgray}{getUserCurrent()} della classe UserContex, ampiamente utilizzato per ottenere l’utente che ha effettuato una richiesta HTTP tramite il contesto di sicurezza di Spring (SecurityContext).
	
	\item Infine rimangono i package \textbf{factory} e \textbf{strategy} contengono le classi di supporto ai service responsabili dell’invio di notifiche agli utenti, generando dinamicamente il contenuto in base alla categoria della notifica. Tale esigenza è stata risolta applicando il \textbf{pattern Factory Method}, che prevede la definizione di una \textbf{interfaccia comune} e di \textbf{classi factory specifiche} in grado di restituire l’istanza più appropriata in base al contesto.
	In combinazione, il \textbf{pattern Strategy} consente di definire comportamenti diversi per la generazione del contenuto della notifica, migliorando flessibilità e manutenibilità del sistema.
	
\end{itemize}
 
\subsection{Tecnologie e architettura frontend}

Poiché il sistema adotta un’architettura client-server, lo sviluppo del client è completamente indipendente dal server.
In questa prospettiva, abbiamo scelto di realizzare un’applicazione web-based, utilizzando tecnologie front-end moderne per garantire un’esperienza utente fluida e reattiva.
Tale approccio consente di mantenere una netta separazione tra logica di presentazione e logica applicativa, e permette, in futuro, di sviluppare ulteriori client, come un’applicazione mobile, che potranno interagire con lo stesso server senza necessità di modificarne l’implementazione.
\\ \\
È stato adottato il \textbf{framework Vue.js}, una libreria basata su JavaScript progettata per la costruzione di interfacce utente dinamiche e reattive. Vue.js è particolarmente
indicato per lo sviluppo di applicazioni \textbf{Single-Page Application (SPA)}, ossia applicazioni web che offrono un’esperienza utente fluida, simile a quella delle applicazioni desktop. \\
Le SPA permettono agli utenti di interagire con un sito o un’applicazione senza dover ricaricare l’intera pagina, grazie alla gestione dinamica dei contenuti attraverso JavaScript. Questo
approccio riduce i tempi di caricamento e migliora la responsività dell’interfaccia, risultando
particolarmente utile per sistemi complessi come quello sviluppato nel nostro progetto.
\\ \\

Di seguito viene presentato uno schema del fronend, seguito da una discussione approfondita delle strategie progettuali e delle tecnologie utilizzate.

\begin{figure}[H]
	\centering
	\includegraphics[width=1\linewidth]{Immagini/Schema frontend.png}
	\caption[Schema frontend]{Schema sintetico dell'architettura del frontend}
\end{figure}

Il codice sorgente del client è organizzato nel package src, che contiene i file principali dell’applicazione
e la configurazione del framework Vue.js. Le dipendenze del progetto, installate tramite npm, si trovano nella directory node\_modules. Di seguito illustriamo le componenti presenti in src:

\begin{itemize}
	
	\item \textbf{App.vue}: L’inizializzazione del framework parte dal file \colorbox{lightgray}{App.vue}, che funge da punto di ingresso per l’applicazione. All’interno di questo file vengono configurate le \textbf{componenti principali} e definito il sistema di routing, che sarà descritto in dettaglio nel punto successivo.
	In particolare, sono presenti le componenti \colorbox{lightgray}{Header} e \colorbox{lightgray}{Footer}, posizionate rispettivamente nella parte superiore e inferiore dell’interfaccia.
	Queste componenti sono incluse nel file principale poiché risultano \textbf{sempre visibili}, indipendentemente dalla pagina in cui si trova l’utente. Inoltre, la direttiva \colorbox{lightgray}{<router-view>} consente di visualizzare dinamicamente la vista corretta in base all’URL richiesto.
	In questa fase viene anche verificato se l’utente autenticato appartiene al ruolo di \textbf{membro} o di {staff}: in quest’ultimo caso, l’applicazione reindirizza automaticamente l’utente alla sezione riservata al personale.
	
	\item \textbf{Directory View}: La directory views contiene tutti i file \textbf{vue} che rappresentano le pagine principali del sito e sono associati a specifiche rotte tramite il sistema di routing.
	Come accennato in precedenza, Vue.js permette la creazione di \textbf{Single-Page Application (SPA)}, ma queste applicazioni presentano lo svantaggio di non avere una suddivisione naturale in pagine, compromettendo la navigabilità. Per risolvere questo problema, utilizziamo la libreria ufficiale \textbf{Vue Router}, che consente di associare un URL (o rotta) a una specifica view.
	
	\item \textbf{Directory router}: La directory router contiene un unico file index.js, in cui viene configurato il sistema di routing dell’applicazione. In questa configurazione, ogni rotta è associata alla rispettiva view descritta sopra, garantendo la navigazione fluida e dinamica tipica delle SPA.
	
	\item \textbf{Directory components}: La directory components contiene tutti i file .vue che rappresentano le componenti riutilizzabili dell’applicazione. A differenza della directory views,
	dove le view corrispondono alle pagine principali e sono direttamente associate alle rotte
	tramite il router, qui troviamo una suddivisione logica di elementi più granulari che vengono richiamati all’interno delle view. Le componenti in questa directory servono a fornire
	modularità al codice, favorendo la riutilizzabilità e una manutenzione più semplice. Inoltre
	implementano funzionalità specifiche che possono essere combinate per costruire view più
	complesse. Le componenti in questa directory non sono associate direttamente alle rotte
	ma vengono importate e utilizzate nelle rispettive view, garantendo un’organizzazione
	logica e semplificando la struttura complessiva del progetto.
	
	\item \textbf{Directory Service}: La directory service contiene file JavaScript che implementano funzioni e logiche riutilizzabili, progettate per essere utilizzate nelle diverse componenti dell’applicazione. Le operazioni ripetitive o complesse, come il recupero di dati da un server o la formattazione di informazioni, sono centralizzate in questa directory.
	
	\item \textbf{Directory stores}: La directory stores contiene file JavaScript che utilizzano \textbf{Pinia}, una libreria ufficiale di Vue.js per la gestione dello stato. Questa struttura permette di salvare informazioni a livello di sessione e di renderle facilmente accessibili in diverse parti dell’applicazione. Grazie a Pinia, le informazioni archiviate possono essere recuperate localmente, riducendo il tempo necessario per caricare dati critici.
	
Per l’implementazione delle funzionalità legate alla mappa, come la ricerca geografica e la visualizzazione degli immobili, è stata utilizzata la libreria \textbf{Vue Leaflet}, una soluzione open source completamente gratuita. Tale libreria si è rivelata un’ottima alternativa a Google Maps, la cui versione gratuita presenta alcune limitazioni in termini di numero di richieste API giornaliere e funzionalità avanzate disponibili solo a pagamento.
	
\end{itemize}
\section{Descrizione dello schema per la persistenza dati}
La gestione della persistenza dei dati è centrale per il nostro backend e si basa su un modello che sfrutta le potenzialità di Hibernate.  PostgreSQL, per la loro affidabilità e diffusione nell’ecosistema Java.

Hibernate si occupa di:

Mappare le entità: Le classi Java rappresentano le entità del dominio, le quali vengono automaticamente correlate alle tabelle del database.
Gestire le relazioni: Le associazioni fra le entità (uno-a-uno, uno-a-molti, molti-a-molti) sono gestite in modo trasparente, riducendo la complessità nella scrittura delle query.
Ottimizzare le operazioni di lettura e scrittura: Grazie al supporto per caching e gestione delle transazioni, Hibernate semplifica l’interazione con il DBMS e contribuisce a mantenere il sistema performante e affidabile.
\newpage
\begin{figure}
	\centering
	\includegraphics[width=1
	\linewidth]{Immagini/diagramma delle classi.drawio.png}
	\caption{Diagramma delle classi usate}
	\label{fig:enter-label}
\end{figure}

\input{Design del Sistema/Diagramma delle classi di design}
\section{Sequence diagram}

\subsection{Caso d'uso: Aggiungi un Annuncio Immobiliare}

\begin{figure}[H]
	\centering
	\includegraphics[width=\textwidth,height=\textheight,keepaspectratio]{Immagini/Sequence diagram/SequenceDiagram inserimentoAnnuncio.png}
	\caption[Sequence diagram 1]{Sequence Diagram del caso d'uso: Inserimento annuncio immobiliare}
\end{figure}

\subsection{Caso d'uso: Attivazione e Disattivazione Notifiche}

\begin{figure}[H]
	\centering
	\includegraphics[width=\textwidth,height=\textheight,keepaspectratio]{Immagini/Sequence diagram/Sequence Diagram Gestione Notifiche.png}
	\caption[Sequence diagram 2]{Sequence Diagram del caso d'uso: Attivazione e Disattivazione Notifiche}
\end{figure}

\subsection{Caso d'uso: Controproposta di un'offerta}

\begin{figure}[H]
	\centering
	\includegraphics[width=\textwidth,height=\textheight,keepaspectratio]{Immagini/Sequence diagram/Sequence Diagram Effettua Controproposta.png}
	\caption[Sequence diagram 3]{Sequence Diagram del caso d'uso: Controproposta di un'offerta}
\end{figure}

\subsection{Caso d'uso: Registrazione di un Nuovo Agente}

\begin{figure}[H]
	\centering
	\includegraphics[width=\textwidth,height=\textheight,keepaspectratio]{Immagini/Sequence diagram/Sequence Diagram Aggiungi Dipendente.png}
	\caption[Sequence diagram 4]{Sequence Diagram del caso d'uso: Aggiungi Dipendente}
\end{figure}



% 4.Artefatti Software e documentazione del processo di sviluppo
\chapter{Processo di sviluppo}
\section{Introduzione}
Quando si sviluppa un software complesso, diventa necessario pianificare il lavoro da svolgere per creare un prodotto robusto e affidabile.\\
In questa sezione discuteremo dei metodi utilizzati per migliorare lo sviluppo del software, come il sistema di versionamento utilizzato e gli strumenti di pulizia del codice usati per ottenere un prodotto di alta qualità.\\
\section{Versionamento del codice}
È deciso di adottare il sistema di versionamento Git, usando la piattaforma GitHub per ospitare il repository del progetto. Essendo l'applicazione divisa in due blocchi precisi e distinti, il Front end Vue e il Server Backend Spring Boot, sono stati creati due repository per gestirli in modo separato ed evitare maggiori conflitti durante lo sviluppo.\\
Il Team di sviluppo si è coordinato per lavorare su funzionalità separate, in modo da ridurre al minimo i conflitti di merge e facilitare l'integrazione del codice, seguendo le seguenti pratiche:\\
\begin{itemize}
    \item \textbf{Commit Frequency:} Si è deciso di effettuare commit frequenti con messaggi chiari e descrittivi, in modo da tracciare facilmente le modifiche apportate al codice.
    \item \textbf{Code Reviews:} Prima di unire una pull request al branch principale, un altro membro del team esegue una revisione del codice per garantire la qualità e la coerenza con gli standard di codifica stabiliti.
\end{itemize}
\newpage
Nelle figure \ref{fig:resoconto-commit-backend} e \ref{fig:resoconto-commit-frontend} sono riportati i resoconti dei commit effettuati su entrambi i repository durante il periodo di sviluppo.
\begin{figure}[H]
	\centering
	\includegraphics[width=1\linewidth]{"Immagini/Resoconto Commit BE.png"}
	\caption[Resoconto Commit Backend]{}
	\label{fig:resoconto-commit-backend}
\end{figure}
\begin{figure}[H]
	\centering
	\includegraphics[width=1\linewidth]{"Immagini/Resoconto Commit FE.png"}
	\caption[Resoconto Commit Frontend]{}
	\label{fig:resoconto-commit-frontend}
\end{figure}
\newpage
\section{Strumenti di pulizia del codice}
Per garantire la qualità del codice e mantenere uno standard elevato, é stato utilizzato uno strumento di pulizia del codice chiamato SonarQube. 	\\
Facendo analisi statiche del codice a intervalli regolari, abbiamo mantenuto un alto rigore qualitativo negli standard forniti dal cliente.\\
Si é fatto uso sia della versione Comunity di SonarQube, istallata in un container Docker locale, che di plugin che si interfacciavano automaticamente con l'IDE per agire in tempo reale durante la scrittura del codice.\\
\begin{figure}[H]
	\centering
	\includegraphics[width=1\linewidth]{"Immagini/Sonarqube-overview.png"}
	\caption[Resoconto SonarQube]{}
	\label{fig:sonarqube-overview}
\end{figure}
\begin{figure}[H]
	\centering
	\includegraphics[width=1\linewidth]{"Immagini/Sonarqube-registo-attivita.png"}
	\caption[Attivitá sul codice SonarQube]{}
	\label{fig:sonarqube-registo-attivita}
\end{figure}

% 5.Testing e valutazione dell’usabilità.
\chapter{Testing unitari}
\section{Introduzione}
Quando si sviluppa un software complesso, diventa necessario pianificare il lavoro da svolgere per creare un prodotto robusto e affidabile.\\
In questa sezione discuteremo dei metodi utilizzati per migliorare lo sviluppo del software, come il sistema di versionamento utilizzato e gli strumenti di pulizia del codice usati per ottenere un prodotto di alta qualità.\\
\section{Versionamento del codice}
È deciso di adottare il sistema di versionamento Git, usando la piattaforma GitHub per ospitare il repository del progetto. Essendo l'applicazione divisa in due blocchi precisi e distinti, il Front end Vue e il Server Backend Spring Boot, sono stati creati due repository per gestirli in modo separato ed evitare maggiori conflitti durante lo sviluppo.\\
Il Team di sviluppo si è coordinato per lavorare su funzionalità separate, in modo da ridurre al minimo i conflitti di merge e facilitare l'integrazione del codice, seguendo le seguenti pratiche:\\
\begin{itemize}
    \item \textbf{Commit Frequency:} Si è deciso di effettuare commit frequenti con messaggi chiari e descrittivi, in modo da tracciare facilmente le modifiche apportate al codice.
    \item \textbf{Code Reviews:} Prima di unire una pull request al branch principale, un altro membro del team esegue una revisione del codice per garantire la qualità e la coerenza con gli standard di codifica stabiliti.
\end{itemize}
\newpage
Nelle figure \ref{fig:resoconto-commit-backend} e \ref{fig:resoconto-commit-frontend} sono riportati i resoconti dei commit effettuati su entrambi i repository durante il periodo di sviluppo.
\begin{figure}[H]
	\centering
	\includegraphics[width=1\linewidth]{"Immagini/Resoconto Commit BE.png"}
	\caption[Resoconto Commit Backend]{}
	\label{fig:resoconto-commit-backend}
\end{figure}
\begin{figure}[H]
	\centering
	\includegraphics[width=1\linewidth]{"Immagini/Resoconto Commit FE.png"}
	\caption[Resoconto Commit Frontend]{}
	\label{fig:resoconto-commit-frontend}
\end{figure}
\newpage
\section{Strumenti di pulizia del codice}
Per garantire la qualità del codice e mantenere uno standard elevato, é stato utilizzato uno strumento di pulizia del codice chiamato SonarQube. 	\\
Facendo analisi statiche del codice a intervalli regolari, abbiamo mantenuto un alto rigore qualitativo negli standard forniti dal cliente.\\
Si é fatto uso sia della versione Comunity di SonarQube, istallata in un container Docker locale, che di plugin che si interfacciavano automaticamente con l'IDE per agire in tempo reale durante la scrittura del codice.\\
\begin{figure}[H]
	\centering
	\includegraphics[width=1\linewidth]{"Immagini/Sonarqube-overview.png"}
	\caption[Resoconto SonarQube]{}
	\label{fig:sonarqube-overview}
\end{figure}
\begin{figure}[H]
	\centering
	\includegraphics[width=1\linewidth]{"Immagini/Sonarqube-registo-attivita.png"}
	\caption[Attivitá sul codice SonarQube]{}
	\label{fig:sonarqube-registo-attivita}
\end{figure}

% 6.Valutazione dell’usabilità.
\chapter{Valutazione dell'usabilità}
\section{Test di usabilità}

I \textbf{test di usabilità} rappresentano una fase essenziale nel processo di sviluppo di un’interfaccia utente, consentendo di valutare l'efficacia, l’efficienza e la soddisfazione dell’utente nell’interazione con il sistema. In particolare, l'uso di mockup interattivi offre la possibilità di raccogliere feedback sulle scelte di design prima ancora della fase di sviluppo, riducendo i costi di eventuali revisioni e migliorando la qualità dell’esperienza utente.
\newline
L'obiettivo principale di questi test è identificare eventuali problemi di navigazione, ambiguità nelle interazioni o difficoltà nella comprensione delle funzionalità, al fine di ottimizzare l'interfaccia prima del rilascio definitivo. 

\vspace{0.5cm} % Aggiunge spazio prima della sezione

\textbf{1. Tempo medio per completare un Task}
\begin{equation}
T_{\text{medio}} = \frac{\sum_{i=1}^{n} T_i}{n}
\end{equation}
dove \( T_i \) è il tempo impiegato dall'utente \( i \) per completare il task e \( n \) è il numero totale di utenti.

\vspace{0.5cm} % Aggiunge spazio prima della sezione
\textbf{2. Tasso di completamento di un Task}
\begin{equation}
T_{\text{completamento}} = \frac{U_{\text{completati}}}{U_{\text{totali}}} \times 100
\end{equation}
dove \( U_{\text{completati}} \) è il numero di utenti che hanno completato il task e \( U_{\text{totali}} \) è il numero totale di utenti che hanno provato il task.

\vspace{0.5cm} % Aggiunge spazio prima della sezione
\textbf{3. Errore Medio per Task}
\begin{equation}
E_{\text{medio}} = \frac{\sum_{i=1}^{n} E_i}{n}
\end{equation}
dove \( E_i \) è il numero di errori commessi dall'utente \( i \) durante il task.

\vspace{0.5cm} % Aggiunge spazio prima della sezione
\textbf{4. Customer Satisfaction Score}
\begin{equation}
CSAT = \frac{\sum_{i=1}^{n} S_i}{n} \times 100
\end{equation}
dove \( S_i \) è il punteggio di soddisfazione dato dall'utente \( i \) e \( n \) è il numero totale di risposte raccolte.

\vspace{0.5cm} % Aggiunge spazio prima della sezione
\textbf{5. System Usability Scale (SUS)}
\newline
\newline
Il \textbf{System Usability Scale (SUS)} è un metodo standardizzato, introdotto da \textbf{John Brooke} nel 1986, utilizzato per misurare l’usabilità di un prodotto attraverso un questionario composto da \textbf{10 affermazioni}. Gli utenti rispondono utilizzando una \textbf{scala Likert a 5 punti}, esprimendo il loro grado di accordo o disaccordo. Il punteggio complessivo, che varia da 0 a 100, fornisce una misura quantitativa dell’usabilità percepita, consentendo di confrontare i risultati con benchmark consolidati.

\vspace{0.5cm} % Aggiunge spazio prima della sezione
\subsubsection{Questionario relativo al mockup "Creazione nuovo annuncio"}

Di seguito presentiamo il questionario utilizzato per valutare il mockup del processo di creazione di un annuncio immobiliare. Ogni domanda prevede cinque opzioni di risposta:

\begin{itemize}
    \item Per niente d’accordo
    \item Poco d’accordo
    \item Né d’accordo né in disaccordo
    \item Abbastanza d’accordo
    \item Molto d’accordo
\end{itemize}
Ai fini del calcolo del punteggio \textbf{SUS}, alle risposte viene assegnato un valore da \textbf{1} a \textbf{5}, dove la prima opzione corrisponde a 1 punto e l’ultima a 5 punti.

\vspace{0.5cm} % Aggiunge spazio prima della sezione

\begin{enumerate}
    \item \textbf{La compilazione dei campi divisi in step ha reso il processo di creazione più intuitivo e meno stancante.}
    \newline
    \texttt{Scopo}: Verificare se la suddivisione del form in più passaggi migliora l’esperienza utente, evitando un sovraccarico cognitivo e rendendo la compilazione più fluida.

    \item \textbf{Il bottone “Avanti” è posizionato in modo poco visibile e difficile da individuare.}
    \newline
    \texttt{Scopo}: Valutare la visibilità e l’intuitività del pulsante che consente di procedere nella compilazione, elemento essenziale per garantire una navigazione chiara e senza interruzioni.

    \item \textbf{I campi a scelta (non liberi) mi hanno aiutato a capire meglio il tipo di informazioni richieste e a velocizzare la compilazione.}
    \newline
    \texttt{Scopo}: Analizzare se l’uso di menu a tendina o opzioni predefinite aiuta a ridurre incertezze, errori e tempi di compilazione rispetto ai campi di testo libero.

    
    \item \textbf{Il numero di campi richiesti è eccessivo o insufficiente, rendendo il form poco bilanciato.}
    \newline
    \texttt{Scopo}: Ottenere un feedback sulla quantità di informazioni richieste, bilanciando completezza e semplicità d’uso, evitando di rendere il processo troppo lungo o complesso.

    \item \textbf{I colori utilizzati sono gradevoli e non affaticano la lettura.}
    \newline
    \texttt{Scopo}: Valutare se la combinazione di colori scelta favorisce una buona leggibilità e un’esperienza visiva piacevole, senza creare affaticamento visivo.

    \item \textbf{L’importazione e la gestione delle foto dell’annuncio risultano poco intuitive.}
    \newline
    \texttt{Scopo}: esaminare la semplicità e l’efficacia del meccanismo di caricamento delle immagini, una funzionalità chiave nella creazione di annunci immobiliari.

    \item \textbf{Trovo utile l’anteprima dell’annuncio prima di confermare la pubblicazione.}
    \newline
    \texttt{Scopo}: Capire se l’anteprima fornisce valore aggiunto agli utenti, permettendo loro di controllare e correggere eventuali errori prima della pubblicazione definitiva.

    \item \textbf{L’avviso in caso di un annuncio in sospeso è poco chiaro o inutile.}
    \newline
    \texttt{Scopo}: Testare la comprensibilità e l’efficacia del messaggio di avviso per evitare che l’utente perda dati o crei annunci duplicati.

    \item \textbf{In caso di errore (compilazione errata o parziale), i messaggi di errore sono posizionati bene e mi aiutano a capire rapidamente dove ho sbagliato.}
    \newline
    \texttt{Scopo}: Valutare la chiarezza dei messaggi di errore e il loro posizionamento, per garantire che l’utente possa correggere facilmente eventuali problemi.

    \item \textbf{Nel complesso, non sono soddisfatto dell’esperienza di creazione di un annuncio.}
    \newline
    \texttt{Scopo}: Raccogliere un’indicazione generale sul livello di soddisfazione dell’utente rispetto all’intero processo, fornendo una misura qualitativa dell’usabilità percepita.
    
\end{enumerate}

\vspace{0.5cm} % Aggiunge spazio prima della sezione
\subsubsection{Risultato SUS ottenuto attraverso i feedback raccolti}
Il System Usability Scale (SUS) viene calcolato seguendo questi passaggi:

\begin{enumerate}

    \item \textbf{Assegnazione punteggio}
    \newline 
    Ogni domanda viene valutata su una scala da 1 a 5:

    \begin{itemize}
        \item 1 = Per niente d’accordo
        \item 2 = Poco d’accordo
        \item 3 = Né d’accordo né in disaccordo
        \item 4 = Abbastanza d’accordo
        \item 5 = Molto d’accordo
    \end{itemize}
    Le 10 domande del questionario SUS sono di due tipi:
    \begin{itemize}
        \item \textbf{Domande dispari (1, 3, 5, 7, 9):} indicano usabilità positiva
        
        \item \textbf{Domande pari (2, 4, 6, 8, 10):} indicano usabilità negativa
    \end{itemize}

    \item \textbf{Calcolo del punteggio per ogni domanda}
    
    \begin{itemize}
        \item \textbf{Per le domande dispari:} Punteggio=(Risposta-1)
        
        \item \textbf{Per le domande pari:} Punteggio=(5-Risposta)
    \end{itemize}

    \item \textbf{Sommiamo tutti i punteggi ottenuti}
    \newline
    Otteniamo un punteggio complessivo che va da 0 a 40.

    \item \textbf{Moltiplichiamo per 2,5}: SUS=(somma dei punteggi)×2.5.
    \newline
    Il punteggio finale sarà compreso tra 0 e 100, ma non rappresenta una percentuale.
    
\end{enumerate}

\textbf{Interpretazione del punteggio SUS}

\begin{itemize}
    \item \textbf{Sopra 80} → Ottima usabilità
    \item \textbf{Tra 70 e 80} → Buona usabilità
    \item \textbf{Tra 50 e 70} → Accettabile, ma migliorabile
    \item \textbf{Sotto 50} → Problemi di usabilità significativi
\end{itemize}
Il questionario è stato compilato da quattro agenti immobiliari, i principali attori di questo caso d’uso. Ciascun agente proviene da un’agenzia immobiliare diversa e appartiene a una fascia d’età differente, al fine di garantire un campione più eterogeneo e realistico. 

\vspace{0.5cm}

\textbf{Utente 1:}
\begin{itemize}
    \item \textbf{Risposte a domande dispari}
    \begin{itemize}
        \item Domanda 1: molto d'accordo = 5-1 = 4
        \item Domanda 3: molto d'accordo = 5-1 = 4
        \item Domanda 5: molto d'accordo = 5-1 = 4
        \item Domanda 7: molto d'accordo = 5-1 = 4
        \item Domanda 9: molto d'accordo = 5-1 = 4
    \end{itemize}
    \item \textbf{Risposte a domande pari}
    \begin{itemize}
        \item Domanda 2: per niente d'accordo = 5-1 = 4
        \item Domanda 4: poco d'accordo = 5-2 = 3
        \item Domanda 6: poco d'accordo = 5-2 = 3
        \item Domanda 8: per niente d'accordo = 5-1 = 4
        \item Domanda 10:per niente d'accordo = 5-1 = 4
    \end{itemize}

    \item \textbf{Totale punteggio utente 1: } 38*2.5 = \textbf{95}
    
\end{itemize}
\textbf{Utente 2:}
\begin{itemize}
    \item \textbf{Risposte a domande dispari}
    \begin{itemize}
        \item Domanda 1: molto d'accordo = 5-1 = 4
        \item Domanda 3: molto d'accordo = 5-1 = 4
        \item Domanda 5: molto d'accordo = 5-1 = 4
        \item Domanda 7: abbastanza d'accordo = 4-1 = 3
        \item Domanda 9: molto d'accordo = 5-1 = 4
    \end{itemize}
    \item \textbf{Risposte a domande pari}
    \begin{itemize}
        \item Domanda 2: per niente d'accordo = 5-1 = 4
        \item Domanda 4: per niente d'accordo = 5-1 = 4
        \item Domanda 6: poco d'accordo = 5-2 = 3
        \item Domanda 8:né d'accordo né in disaccordo = 5-3 = 2
        \item Domanda 10:per niente d'accordo = 5-1 = 4
    \end{itemize}

    \item \textbf{Totale punteggio utente 2: } 36*2.5 = \textbf{90}
    
\end{itemize}
\textbf{Utente 3:}
\begin{itemize}
    \item \textbf{Risposte a domande dispari}
    \begin{itemize}
        \item Domanda 1:abbastanza d'accordo = 4-1 = 3
        \item Domanda 3: molto d'accordo = 5-1 = 4
        \item Domanda 5: molto d'accordo = 5-1 = 4
        \item Domanda 7: molto d'accordo = 5-1 = 4
        \item Domanda 9: molto d'accordo = 5-1 = 4
    \end{itemize}
    \item \textbf{Risposte a domande pari}
    \begin{itemize}
        \item Domanda 2: poco d'accordo = 5-2 = 3
        \item Domanda 4: poco d'accordo= 5-2 = 3
        \item Domanda 6: per niente d'accordo = 5-1 = 4
        \item Domanda 8: poco d'accordo = 5-2 = 3
        \item Domanda 10: per niente d'accordo = 5-1 = 4
    \end{itemize}

    \item \textbf{Totale punteggio utente 3: } 36*2.5 = \textbf{90}
    
\end{itemize}
\textbf{Utente 4:}
\begin{itemize}
    \item \textbf{Risposte a domande dispari}
    \begin{itemize}
        \item Domanda 1: molto d'accordo = 5-1 = 4
        \item Domanda 3: molto d'accordo = 5-1 = 4
        \item Domanda 5: né d'accordo né in dissacordo = 3-1 = 2
        \item Domanda 7: molto d'accordo = 5-1 = 4
        \item Domanda 9: molto d'accordo = 5-1 = 4
    \end{itemize}
    \item \textbf{Risposte a domande pari}
    \begin{itemize}
        \item Domanda 2: poco d'accordo = 5-2 = 3
        \item Domanda 4: né 'accordo né in disaccordo = 5-3 = 2
        \item Domanda 6:per niente d'accordo = 5-1 = 4
        \item Domanda 8: per niente d'accordo = 5-1 = 4
        \item Domanda 10:per niente d'accordo = 5-1 = 4
    \end{itemize}

    \item \textbf{Totale punteggio utente 4: } 35*2.5= \textbf{87.5}
    
\end{itemize}
Media punteggio SUS = 95+90+90+87.5/4 = \textbf{90.62}. Il punteggio è nettamente superiore a 80 il che indica un \textbf{ottima usabilità}.

% X. Fonti
\chapter{Fonti}
\begin{thebibliography}{9}
    \bibitem{nielsen1995} Nielsen, J. (1995). \textit{10 Heuristics for User Interface Design}. Nielsen Norman Group. Disponibile su: \url{https://www.nngroup.com/articles/ten-usability-heuristics/}.
    \bibitem{norman1988} Norman, D. A. (1988). \textit{The Psychology of Everyday Things}. Basic Books. (Ri-edizione: \textit{The Design of Everyday Things}, 2013).
    \bibitem{miller1956} Miller, G. A. (1956). \textit{The Magical Number Seven, Plus or Minus Two: Some Limits on Our Capacity for Processing Information}. Psychological Review, 63(2), 81-97.
    \bibitem{shneiderman2004} Shneiderman, B., \& Plaisant, C. (2004). \textit{Designing the User Interface: Strategies for Effective Human-Computer Interaction} (4ª ed.). Pearson.
    \bibitem{wickens2008} Wickens, C. D. (2008). \textit{Multiple Resources and Mental Workload}. Human Factors, 50(3), 449-455.
    \bibitem{pieters2004} Pieters, R., \& Wedel, M. (2004). \textit{Attention Capture in Advertising: Brand, Pictorial, and Text-Size Effects}. Journal of Marketing, 68(2), 36-50.
\end{thebibliography}

\end{document}
