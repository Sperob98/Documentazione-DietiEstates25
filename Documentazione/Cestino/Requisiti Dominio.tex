\subsection{Requisiti di Dominio}

\subsection*{Gestione dell'Agenzia Immobiliare}
\begin{itemize}
    \item [D\textminus A1] Un'agenzia immobiliare è composta da un fondatore e da una lista di dipendenti, ciascuno dei quali ricopre il ruolo di amministratore o agente immobiliare.
    \item [D\textminus A2] La registrazione di un’agenzia immobiliare comporta la generazione automatica di un amministratore con il ruolo di fondatore.
    \item [D\textminus A3] Alla creazione di un account amministratore vengono assegnate credenziali predefinite che possono essere modificate in seguito.
    \item [D\textminus A4] Le credenziali di un amministratore includono uno username, costruito seguendo il formato \texttt{[nome][cognome]@DIETI25.com}, e una password scelta dall’utente.
\end{itemize}

\subsection*{Gestione degli Utenti}
\begin{itemize}
    \item [D\textminus U1] Un guest può registrarsi al sistema fornendo un’email valida e una password, diventando così un utente.
    \item [D\textminus U2] Un guest può registrarsi o accedere tramite l’uso di API di terze parti.
\end{itemize}

\subsection*{Gestione degli Immobili}
\begin{itemize}
    \item [D\textminus G1] Ogni immobile è descritto da una serie di dettagli obbligatori, tra cui: foto, descrizione, prezzo, dimensioni, indirizzo, numero di stanze, piano, presenza di ascensore, classe energetica e ulteriori servizi (es. portineria, climatizzazione).
    \item [D\textminus G2] Ogni immobile è associato a una delle seguenti tipologie contrattuali: "vendita" o "affitto".
    \item [D\textminus G3] Ogni immobile, al momento della creazione, ha associata una lista di punti di riferimento generata automaticamente utilizzando il servizio GEOAPIFY.
\end{itemize}

\subsection*{Ricerca e Visualizzazione}
\begin{itemize}
    \item [D\textminus R1] La ricerca degli immobili consente di effettuare una selezione geografica basata su un punto centrale e un raggio, garantendo una precisione pari o superiore al 95\%.
    \item [D\textminus R2] La ricerca avanzata degli immobili permette di utilizzare parametri multipli, tra cui tipologia di inserzione, prezzo minimo e massimo, numero di stanze, classe energetica e area geografica tramite mappa interattiva.
    \item [D\textminus R3] Gli immobili ricercati possono essere visualizzati su una mappa interattiva.
\end{itemize}

\subsection*{Gestione delle Offerte}
\begin{itemize}
    \item [D\textminus O1] Le inserzioni degli immobili possono ricevere offerte da parte degli utenti, composte da un prezzo proposto e dalle credenziali dell’offerente. Le offerte sono visibili a tutti gli utenti.
\end{itemize}

\subsection*{Gestione delle Notifiche}
\begin{itemize}
    \item [D\textminus N1] Il sistema di notifiche consente di inviare avvisi personalizzati agli utenti, categorizzati in base a eventi come: nuove proprietà in linea con ricerche precedenti, conferme o rifiuti di proposte, e messaggi promozionali.
    \item [D\textminus N2] Le notifiche promozionali sono composte da un contenuto in formato RICH TEXT, definito dagli amministratori, e sono visibili esclusivamente agli utenti iscritti alla relativa newsletter.
\end{itemize}