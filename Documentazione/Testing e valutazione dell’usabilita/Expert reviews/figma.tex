\subsection{Expert Review sul Prototipo Figma}

L’attività di revisione euristica è stata condotta sul prototipo sviluppato in \textit{Figma} prima della realizzazione della versione web.  
L’obiettivo è stato valutare il livello di usabilità e coerenza delle principali interfacce progettate, verificando la presenza di feedback, la chiarezza dei flussi e la consistenza grafica secondo i dieci principi di Nielsen.  

\subsubsection*{1. Visibilità dello stato del sistema}  
Nel prototipo sono stati previsti diversi meccanismi di feedback.  
Nei flussi \textit{Fai controproposta} e \textit{Registra dipendente} sono stati inseriti \textbf{indicatori di caricamento} per mostrare lo stato del sistema durante le operazioni.  
In tutti i form è presente un \textbf{sistema di validazione immediata}: i campi non validi vengono evidenziati in rosso e, nel caso della \textit{creazione di un annuncio}, il processo è suddiviso in step.  
Quando uno step contiene errori, il suo indicatore cambia colore e un box iniziale riepiloga i campi non validi.  
Tutti i flussi analizzati includono un \textbf{doppio messaggio di conferma} prima dell’esecuzione definitiva, ad eccezione della creazione di un annuncio.  
Nel complesso, il prototipo offre una buona visibilità dello stato del sistema, con l’unica criticità relativa all’assenza di conferma dopo la pubblicazione.

\subsubsection*{2. Corrispondenza tra sistema e mondo reale}  
Il linguaggio utilizzato è \textbf{semplice e diretto}, adatto agli utenti finali.  
Le sezioni con terminologia più specifica (legate alle agenzie Immobiliari) risultano coerenti con il dominio e adatte a un pubblico professionale.  
Sono state impiegate \textbf{icone standard e universalmente riconoscibili}, affiancate da \textbf{tooltip descrittivi} che ne chiariscono il significato.  
L’unico aspetto ancora da migliorare riguarda la \textbf{scelta delle icone nello stepper} di creazione annuncio, dove sarebbe utile uno studio mirato per rendere più intuitivi i passaggi.

\subsubsection*{3. Controllo e libertà dell’utente}  
Il prototipo garantisce buone possibilità di annullamento e controllo.  
Tutti i popup includono una \textbf{“X” per chiudere o annullare}, e i messaggi di conferma prevedono esplicitamente un pulsante \textit{Annulla}.  
Le azioni potenzialmente distruttive, come la \textbf{disattivazione delle notifiche}, richiedono conferma; viceversa, l’attivazione non la richiede poiché non comporta rischi di perdita di dati.  
La progettazione su questo punto è stata ritenuta soddisfacente e coerente con le aspettative di usabilità.

\subsubsection*{4. Coerenza e standard}  
Il design mantiene nel complesso una \textbf{coerenza visiva} nei colori e nello stile dei pulsanti, in linea con il concept minimalista del progetto.  
Tuttavia, si è rilevata una \textbf{mancanza di uniformità nei popup}: nei diversi casi d’uso sono stati utilizzati modelli leggermente differenti, frutto di sperimentazioni grafiche ancora non consolidate.  
In fase di sviluppo sarà necessario \textbf{uniformare componenti e modali}, garantendo una coerenza piena tra tutti i flussi.

\subsubsection*{5. Prevenzione e gestione degli errori}  
La prevenzione degli errori è uno degli aspetti più curati nel prototipo.  
I form mostrano in tempo reale i campi non validi e forniscono \textbf{indicazioni visive chiare (rosso)} insieme a messaggi testuali.  
La combinazione di validazione immediata e conferme d’azione riduce significativamente il rischio di errori da parte dell’utente.  
Questo punto risulta pienamente soddisfatto.

\subsubsection*{6. Efficienza e flessibilità d’uso}  
Essendo un prototipo statico, l’efficienza d’uso è valutabile solo in parte.  
L’interfaccia è stata progettata per un utilizzo anche da \textbf{dispositivi verticali}, ma la resa migliore si ottiene in \textbf{orientamento orizzontale (landscape)}.  
Non sono state ancora considerate \textbf{scorciatoie o funzionalità avanzate} per utenti esperti (es. salvataggio bozza, importazione di dati da file), che potrebbero rappresentare un’evoluzione futura del progetto.  
Si riconosce quindi questo come un punto da sviluppare ulteriormente.

\subsubsection*{7. Chiarezza del contenuto}  
I testi sono \textbf{brevi, diretti e privi di tecnicismi}, con un linguaggio coerente e immediato.  
Quando una sola etichetta non è sufficiente, è previsto l’uso di \textbf{tooltip esplicativi}.  
Questo approccio consente una buona comprensione del flusso anche senza documentazione aggiuntiva.

\subsubsection*{8. Supporto al riconoscimento}
Le principali funzioni sono \textbf{raggiungibili in pochi clic}.
La \textit{header bar} consente di passare rapidamente tra ricerca, storico annunci e notifiche; inoltre, il passaggio tra l’area cliente e quella agenzia è reso accessibile dal \textit{footer}.
La disposizione degli elementi è stata progettata per garantire una \textbf{chiara collocazione visiva} delle funzioni più rilevanti, favorendo l’efficacia della navigazione.

\subsubsection*{9. Feedback e conferme}  
Le operazioni critiche, come eliminazione o modifica, utilizzano lo stesso colore principale del sito, il che può generare confusione con azioni neutre.  
Si suggerisce di differenziare \textbf{visivamente le azioni distruttive} (es. usando toni di rosso o arancione) e di inserire messaggi di conferma più evidenti per i processi più delicati.  
Nonostante ciò, il comportamento rimane accettabile per utenti esperti.

\subsubsection*{10. Aiuto e documentazione}  
Non è stato previsto un sistema di aiuto, documentazione o FAQ, poiché il prototipo nasce in un contesto accademico.  
Tuttavia, in una versione completa sarebbe opportuno introdurre:  
\begin{itemize}
    \item una \textbf{sezione FAQ} o assistenza per i problemi più comuni;  
    \item \textbf{overlay o onboarding guidato} per i nuovi utenti;  
    \item pulsanti contestuali “Come funziona” nelle pagine più complesse.  
\end{itemize}

\subsubsection*{Autovalutazione e sintesi}  
La tabella seguente riassume il livello di soddisfacimento delle euristiche sul prototipo Figma (0 = non soddisfatto, 1 = parzialmente soddisfatto, 2 = soddisfatto).

\begin{table}[h!]
\centering
\begin{tabular}{p{0.4cm} p{4cm} p{1.5cm} p{7cm}}
\hline
\textbf{\#} & \textbf{Criterio} & \textbf{Valutazione} & \textbf{Motivazione sintetica} \\ \hline
1 & Visibilità stato & 2 & Feedback e validazioni chiare, manca solo conferma finale in creazione annuncio \\
2 & Corrispondenza mondo reale & 2 & Linguaggio semplice e coerente, icone chiare \\
3 & Controllo e libertà & 2 & Annulla e conferme sempre presenti \\
4 & Coerenza e standard & 1 & Popup non uniformi tra i flussi \\
5 & Prevenzione errori & 2 & Validazione in tempo reale e segnalazioni visive \\
6 & Efficienza/flessibilità & 1 & Assenza di scorciatoie e ottimizzazione solo parziale per mobile \\
7 & Chiarezza del contenuto & 2 & Testi sintetici e coerenti \\
8 & Supporto al riconoscimento & 2 & Navigazione semplice e confermata da test visivi \\
9 & Feedback/Conferme & 1 & Colori azioni critiche da differenziare \\
10 & Aiuto/Documentazione & 0 & Non prevista sezione di supporto \\ \hline
\end{tabular}
\caption{Autovalutazione euristica del prototipo Figma secondo i principi di Nielsen.}
\end{table}

\subsubsection*{Conclusioni}  
Il prototipo Figma risulta \textbf{completo, coerente e ben strutturato}, con una gestione accurata dei form, validazioni efficaci e buona leggibilità generale.  
I principali margini di miglioramento riguardano la \textbf{coerenza visiva dei popup}, la \textbf{differenziazione dei feedback critici} e l’assenza di un \textbf{sistema di aiuto} o onboarding.  
Questi aspetti costituiranno la base per le iterazioni successive e per l’ottimizzazione della versione web finale.
