\subsection{Risultati dell'esperimento}

Presentiamo i risultati dell'esperimento


\begin{figure}[H]
	\centering
	\includegraphics[width=\linewidth]{Immagini/esperimento finale/tabella risultati.png}
	\caption[Risultati esperimento]{Risultati esperimento}
\end{figure}

\textbf{1. Tempo medio per completare l'esperimento}
\begin{equation}
	T_{\text{medio}} = \frac{9+6+6+4+10}{5} = 7 min
\end{equation}

\vspace{0.5cm} % Aggiunge spazio prima della sezione
\textbf{2. Tasso di completamento dell'esperimento}
\begin{equation}
	T_{\text{completamento}} = \frac{5}{5} \times 100 = 100\%
\end{equation}
dove \( U_{\text{completati}} \) 

\vspace{0.5cm} % Aggiunge spazio prima della sezione
\textbf{3. Errore Commesso Medio durante l'esperimento}
\begin{equation}
	E_{\text{medio}} = \frac{1}{5} = 0,2
\end{equation}

\vspace{0.5cm} % Aggiunge spazio prima della sezione
\textbf{4. Punteggio medio SUS}
\begin{equation}
	SUS_{\text{medio}} = \frac{95+90+90+87.5+97.5}{5} = 92
\end{equation}
	

La valutazione è stata condotta su un campione di cinque utenti, ciascuno rappresentativo di una diversa categoria di potenziali utilizzatori del sistema. Nonostante la dimensione ridotta del campione, la scelta mirata dei partecipanti consente di considerare i risultati ottenuti come sufficientemente affidabili per una prima valutazione dell’usabilità del sistema.
\\ \\
L’obiettivo degli esperimenti era misurare l’efficacia, l’efficienza e la soddisfazione degli utenti durante l’interazione con l’applicazione, verificando in particolare se l’interfaccia risultasse intuitiva e le funzionalità facilmente accessibili.
Dall’analisi dei dati emerge che le scelte progettuali adottate, in particolare la realizzazione di un’interfaccia semplice, minimalista e coerente con le linee guida di usabilità, si sono rivelate appropriate ed efficaci. Gli utenti sono riusciti a completare i compiti assegnati senza particolari difficoltà, con un numero limitato di errori e tempi di esecuzione mediamente contenuti. Inoltre, i punteggi ottenuti dal questionario di soddisfazione (SUS) indicano un buon livello di apprezzamento generale, confermando che l’esperienza d’uso risulta complessivamente positiva.
\\ \\
Questi risultati suggeriscono che il sistema è in grado di offrire un’interazione fluida e intuitiva, rispondendo alle esigenze dei diversi profili di utenti previsti. Eventuali miglioramenti futuri potranno concentrarsi sull’ottimizzazione di alcune funzionalità secondarie e sull’introduzione di elementi di supporto che rendano l’esperienza ancora più accessibile per gli utenti meno esperti.
