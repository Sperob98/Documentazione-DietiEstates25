\section*{Introduzione}
In questa sezione verranno raccolti e descritti i vari requisiti del sistema in analisi, suddivisi in requisiti funzionali e non funzionali.\\
 I requisiti funzionali sono stati derivati dai casi d'uso identificati nella fase di analisi, mentre i requisiti non funzionali riguardano aspetti quali prestazioni, 
 sicurezza e usabilità del sistema. 

\section*{Obiettivo}
Il sistema deve fornire un'area amministrativa e un'area Cliente, con funzionalità specifiche per i diversi ruoli utente e per la gestione degli immobili. Le funzionalità sono state riformulate in un linguaggio tecnico per favorire chiarezza e comprensione.

\section*{Area Amministrativa}
L'area amministrativa consente la gestione degli account e degli immobili. I ruoli previsti sono:
\begin{itemize}
    \item Amministratore
    \item Amministratore di supporto
    \item Agenti immobiliari
\end{itemize}

\subsection*{Accesso all'Area Amministrativa}
\begin{description}[style=nextline]
    \item[Amministratore:] L'accesso avviene dopo la registrazione di un'agenzia immobiliare tramite richiesta al sistema. Una volta approvata, vengono fornite credenziali predefinite per l'accesso.
    \item[Amministratore di supporto:] Creati dall'Amministratore principale e associati alla stessa agenzia immobiliare. Le credenziali sono generate dall'Amministratore.
    \item[Agenti immobiliari:] Creati dall'Amministratore o dall'Amministratore di supporto. Le credenziali sono anch'esse generate dall'Amministratore.
\end{description}

\section*{Area Cliente}
L'area Cliente è destinata agli utenti registrati che possono:

\begin{itemize}
    \item \textbf{Ricercare immobili:}
    \begin{itemize}
        \item Ricerca avanzata con filtro per posizione geografica (selezionando un punto e un raggio su una mappa).
        \item Visualizzazione degli immobili in lista o su mappa.
        \item Filtri aggiuntivi per caratteristiche dell'immobile e del contratto (es. metratura, numero di stanze, ecc.).
    \end{itemize}

    \item \textbf{Visualizzare immobili completi di dettagli:}
    \begin{itemize}
        \item Inclusi dati come metratura, stanze, servizi, ecc.
        \item Informazioni sui luoghi di interesse vicini (es. scuole, parchi, metro) utilizzando l'integrazione con Geopify.
    \end{itemize}

    \item \textbf{Proporre offerte:}
    \begin{itemize}
        \item L'utente registrato può inviare una proposta sull'immobile, che verrà gestita dall'agente immobiliare.
    \end{itemize}

    \item \textbf{Gestire notifiche push:}
    \begin{itemize}
        \item Notifiche categorizzate (promozionali, private, ecc.).
        \item Possibilità di disattivare categorie specifiche.
    \end{itemize}
\end{itemize}

\section*{Funzionalità dei Ruoli}
\subsection*{Amministratore}
\begin{itemize}
    \item Creare account per Amministratori di supporto e agenti immobiliari.
    \item Modificare e cancellare definitivamente immobili creati dagli agenti.
    \item Modificare la propria password.
\end{itemize}

\subsection*{Amministratore di supporto}
\textbf{Nota:} Non sono stati definiti requisiti specifici per questo ruolo. Si propone di assegnare le stesse funzionalità dell'Amministratore, con eventuali limitazioni future da specificare.

\subsection*{Agenti immobiliari}
\begin{itemize}
    \item Creare, modificare e cancellare immobili da loro creati.
    \item Visualizzare le proposte ricevute sugli immobili.
    \item Registrare proposte esterne.
    \item Accettare o rifiutare proposte.
    \item Modificare la propria password.
\end{itemize}

% punto a)
\newpage
\section{Glossario}

\textbf{Amministratore}: Ruolo all'interno del sistema che gode di accesso a tutti gli Endpoint, autorizzazioni  e metodi.\\
\textbf{Agente}: Ruolo all'interno del sistema che è autorizzato a creare, modificare ed eliminare gli Annunci per gli immobili gestiti dall'Agenzia.\\
\textbf{Agenzia}: Entità caratterizzata dal suo Amministratore e Agenti in  impiego. Alla sua creazione nella base dati verrà automaticamente creato un Amministratore.\\
\textbf{Annunci}: Entità caratterizzata dal un Immobile e il tipo di Contratto.\\

\newpage
% punto b)
\newpage
\section{Modellazione dei Casi d'Uso}
Dopo aver definito i requisiti funzionali e i vincoli del sistema, questa sezione illustra i casi
d'uso (use case), che descrivono in dettaglio le principali interazioni tra gli utenti e il sistema
per il raggiungimento degli obiettivi. Nella nostra analisi preliminare, abbiamo individuato quattro
attori principali che interagiscono con il sistema: Guest, Utente, Agente e Manager.\\
\begin{itemize}
    \item Guest: rappresenta l'utente non autenticato, che accede al sito senza effettuare il login. Questo attore ha accesso limitato, ma può comunque compiere alcune azioni di base, come
effettuare ricerche sugli Immobili e visualizzare informazioni pubbliche su di essi.
\item User: questo attore è un utente autenticato con un account registrato. Rispetto al Guest, ha funzionalità aggiuntive, come fare
offerte sugli Immobili, monitorare il proprio storico delle ricerche effettuate e ricevere notifiche in base alle preferenze che ha esibito. Queste notifiche possono esse ulteriormente personalizzata nelle impostazioni con dei filtri.
\item Agente: rappresenta l'utente che può creare e modificare gli Annunci e gli Immobili.
\item Manager: l'attore che gestisce l'agenzia Immobiliare nella sua interezza, avendo il potere di appuntare sia Agenti che altri Manager. Inoltre il Manager può curare il catalogo di Annunci, modificando o eliminando elementi dalla lista.
\end{itemize}
Per ciascun attore, sono stati definiti i casi d'uso relativi, illustrati nei diagrammi seguenti. Questi schemi mostrano le modalità principali d'interazione con il sistema e forniscono una
visione chiara, delineando le possibilità di interazione per ciascun tipo di utente. In questo
modo, è possibile comprendere come le funzionalità definite nei requisiti trovano applicazione
pratica all'interno delle azioni eseguibili da ciascun attore, evidenziando i passaggi chiave delle interazioni.
\newpage
\subsection*{Diagrammi Casi d'uso}

\begin{figure}[H]
\caption{Casi d'uso Guest}
\centering
\includegraphics[width=0.8\textwidth]{Immagini/Diagrammi Casi D'uso/UseCase-Utente Non Registrato.drawio.png}
\end{figure}

\begin{figure}[H]
\centering
\caption{Casi d'uso Utente}
\includegraphics[width=0.8\textwidth]{Immagini/Diagrammi Casi D'uso/UseCase-Utente registrato.drawio.png}
\end{figure}

\begin{figure}[H]
\centering
\caption{Casi d'uso Agente}
\includegraphics[width=0.8\textwidth]{Immagini/Diagrammi Casi D'uso/UseCase-Agente.drawio.png}
\end{figure}

\begin{figure}[H]
\centering
\caption{Casi d'uso Manager}
\includegraphics[width=0.8\textwidth]{Immagini/Diagrammi Casi D'uso/UseCase-Admin.drawio.png}
\end{figure}
\newpage

% punto c)
\section{Analisi Target Utenti
}
Quando si crea un software, è importante ricordare che non è necessario soddisfare ogni singolo utente possibile, ma solo la categoria di utenti che usufruiranno del nostra sistema più frequentemente.\\
Questa decisione ci permette di concentrare la nostra analisi e ridurre funzioni che, al momento del rilascio del software al committente, sono superflue.\\
Uno strumento utile per fare questa analisi è la 
Persona, un modello di utente ideale che può essere basato su ricerche di mercato oppure da input dai committenti stessi.\\
\subsection*{Personas Individuate}
Per creare Personas è stato adottato un approccio basato su ipotesi e ricerche secondarie.\\
Abbiamo così definito il contesto, identificando il pubblico target, i loro obiettivi e le situazioni in cui potrebbero utilizzare il prodotto. Si possono utilizzare archetipi comuni del dominio di riferimento per immaginare utenti tipo, discutendo possibili bisogni, obiettivi e preferenze degli utenti.\\
Abbiamo creato le Personas a partire da una template predefinita e usando Figma per il design e l'aggiornamento di esse in corso d'opera.
\\

\begin{figure}[H]
\centering
\caption{Proprietario Agenzia Immobiliare}
\includegraphics[width=1\textwidth]{{Immagini/Personas/Personas-Propretario di un agenzia Immobiliare.png}}
\end{figure}

\begin{figure}[H]
\centering
\caption{Agente Immobiliare}
\includegraphics[width=1\textwidth]{{Immagini/Personas/Personas-Agente Immobiliare.png}}
\end{figure}

\begin{figure}[H]
\centering
\caption{Manager di azienda tecnologica}
\includegraphics[width=1\textwidth]{Immagini/Personas/Personas-manager di azienda tecnologica.png}
\end{figure}

\begin{figure}[H]
\centering
\caption{Padre di famiglia}
\includegraphics[width=1\textwidth]{Immagini/Personas/Personas-Padre di famiglia.png}
\end{figure}
\newpage
\subsection*{Tratti Caratteristici}
Analizzando queste Personas possiamo notare dei tratti che possiamo usare nella nostra applicazione:
\begin{itemize}
    \item  L'Agente- Aldo Imparato: questa categoria di utenti è parte dello staff dell'Agenzia Immobiliare e quindi ha accesso al lato di amministrazione degli Immobili, e potrebbe volere una sezione nella sa pagina personale che gli mostra tutti gli Immobili a suo carico.
    \item  Il Padre di Famiglia - Giovanni Rossi: l'archetipo che incarna il Utente medio che ha vaghe idee su cosa vuole e si affida all'Agente Immobiliare. Dobbiamo quindi tener conto che le informazioni visibili a questo tipo di utenti deve essere chiara e concisa, come ad esempio le etichette che indicando punti di interesse come parchi e scuole.
    \item Il Manager di Azienda Tecnologica - Giorgia Esposito: l'utente che sta cercando qualcosa di ben definito, ma l'agenzia non offre Immobili con le caratteristiche cercate. Per questo tipo di persone offriremo un servizio di iscrizione ai tag di interesse per ricevere notifiche quando un immobile che rispecchia tali desideri viene messo in vendita.
    \item Il Proprietario dell'Agenzia Immobiliare - Marco Bianchi: gli Manager del sistema che gestiscono dipendenti e il catalogo di Immobili, questo tipo di utente deve gestire multipli dati personali e sensibili. Quando crea un Agente o Manager, le nuove informazioni verranno generate dal sistema e inviate tramite un servizio di posta elettronica.
\end{itemize}
\newpage
% punto d)
\clearpage
\section{Descrizione Requisiti}

\subsection{Requisiti non funzionali}

\subsection*{Sicurezza}

\begin{table}[H]
    \centering
    \renewcommand{\arraystretch}{1.3} % Aumenta leggermente l'altezza delle righe
    
    \begin{tabular}{|p{3cm}|p{10cm}|} 
        \hline
        \textbf{ID} & \textbf{Descrizione} \\  
        \hline
        NF-S1 &  Implementazione di protocolli standard di sicurezza (es. HTTPS per il trasporto sicuro
        dei dati). \\ 
        \hline
        NF-S2 &  Le credenziali devono essere salvate in modo sicuro, utilizzando tecniche di hashing sicuro. \\ 
        \hline
        NF-S3 &  Le richieste alle REST API devono essere autenticate utilizzando JWT (JSON Web Tokens), per garantire che solo gli utenti autenticati possano accedere alle risorse protette. \\ 
        \hline
    \end{tabular}
    
\end{table}
\subsection*{Interfaccia utente}

\begin{table}[H]
    \centering
    \renewcommand{\arraystretch}{1.3} % Aumenta leggermente l'altezza delle righe
    \begin{tabular}{|p{3cm}|p{10cm}|} 
        \hline
        \textbf{ID} & \textbf{Descrizione} \\  
        \hline
        NF-UI11 & L'interfaccia utente deve adattarsi automaticamente alla risoluzione dello schermo dell’utente, garantendo un’esperienza di utilizzo ottimale su dispositivi mobili e desktop. \\ 
        \hline
    \end{tabular}
\end{table}
\subsection*{Prestazioni}

\begin{table}[H]
    \centering
    \renewcommand{\arraystretch}{1.3} % Aumenta leggermente l'altezza delle righe
    \begin{tabular}{|p{3cm}|p{10cm}|} 
        \hline
        \textbf{ID} & \textbf{Descrizione} \\  
        \hline
        NF-P1 & La ricerca e la visualizzazione degli Immobili devono avere un tempo di risposta inferiore
        a 2 secondi, con un database contenente almeno 10.000 record. \\ 
        \hline
    \end{tabular}
\end{table}
\subsection*{Manutenibilità e scalabilità}

\begin{table}[H]
    \centering
    \renewcommand{\arraystretch}{1.3} % Aumenta leggermente l'altezza delle righe
    \begin{tabular}{|p{3cm}|p{10cm}|} 
        \hline
        \textbf{ID} & \textbf{Descrizione} \\  
        \hline
        NF-MS1 & Il sistema deve essere progettato per supportare l’aggiunta di nuove tipologie di Immobili
        e contratti (ad esempio, affitti brevi) senza modificare l’architettura esistente. \\ 
        \hline
    \end{tabular}
\end{table}

\newpage

\subsection{Requisiti di dominio}

\subsection*{Gestione dell'agenzia Immobiliare}

\begin{table}[H]
    \centering
    \renewcommand{\arraystretch}{1.3} % Aumenta leggermente l'altezza delle righe
    
    \begin{tabular}{|p{3cm}|p{10cm}|} 
        \hline
        \textbf{ID} & \textbf{Descrizione} \\  
        \hline
        D-A1 &  Un’agenzia Immobiliare è composta da un fondatore e da una lista di dipendenti, ciascuno
        dei quali ricopre il ruolo di Manager o Agente Immobiliare. \\ 
        \hline
        D-A2 &  La registrazione di un’agenzia Immobiliare comporta la generazione automatica di un
        Manager con il ruolo di fondatore. \\ 
        \hline
        D-A3 &  Alla creazione di un account Manager vengono assegnate credenziali predefinite
        che possono essere modificate in seguito. \\ 
        \hline
        D-A4 &   Le credenziali di un Manager includono uno username, costruito seguendo il formato [nome][cognome]@DIETI25.com, e una password scelta dall’utente. \\ 
        \hline
    \end{tabular}
    
\end{table}
\subsection*{Gestione degli utenti}

\begin{table}[H]
    \centering
    \renewcommand{\arraystretch}{1.3} % Aumenta leggermente l'altezza delle righe
    
    \begin{tabular}{|p{3cm}|p{10cm}|} 
        \hline
        \textbf{ID} & \textbf{Descrizione} \\  
        \hline
        D-U1 & Un guest può registrarsi al sistema fornendo un’email valida e una password, diventando così un utente. \\ 
        \hline
        D-U2 & Un guest può registrarsi o accedere tramite l’uso di API di terze parti. \\ 
        \hline
    \end{tabular}
    
\end{table}
\subsection*{Gestione degli Immobili}

\begin{table}[H]
    \centering
    \renewcommand{\arraystretch}{1.3} % Aumenta leggermente l'altezza delle righe
    
    \begin{tabular}{|p{3cm}|p{10cm}|} 
        \hline
        \textbf{ID} & \textbf{Descrizione} \\  
        \hline
        D-G1 &  Ogni immobile è descritto da una serie di dettagli obbligatori, tra cui: foto, descrizione, prezzo, dimensioni, indirizzo, numero di stanze, piano, presenza di ascensore, classe energetica e ulteriori servizi (es. portineria, climatizzazione). \\ 
        \hline
        D-G2 &  Ogni immobile è associato a una delle seguenti tipologie contrattuali: "vendita" o "affitto". \\ 
        \hline
        D-G3 &  Ogni immobile, al momento della creazione, ha associata una lista di punti di riferimento generata automaticamente utilizzando il servizio GEOAPIFY. \\ 
        \hline
    \end{tabular}
    
\end{table}
\subsection*{Ricerca e visualizzazione}

\begin{table}[H]
    \centering
    \renewcommand{\arraystretch}{1.3} % Aumenta leggermente l'altezza delle righe
    
    \begin{tabular}{|p{3cm}|p{10cm}|} 
        \hline
        \textbf{ID} & \textbf{Descrizione} \\  
        \hline
        D-R1 &  La ricerca degli immobili consente di effettuare una selezione geografica basata su un punto centrale e un raggio, garantendo una precisione pari o superiore al 95\%. \\ 
        \hline
        D-R2 &  La ricerca avanzata degli immobili permette di utilizzare parametri multipli, tra cui tipologia di inserzione, prezzo minimo e massimo, numero di stanze, classe energetica e area geografica tramite mappa interattiva. \\ 
        \hline
        D-R3 &  Gli immobili ricercati possono essere visualizzati su una mappa interattiva. \\ 
        \hline
    \end{tabular}
    
\end{table}
\subsection*{Gestione delle offerte}

\begin{table}[H]
    \centering
    \renewcommand{\arraystretch}{1.3} % Aumenta leggermente l'altezza delle righe
    
    \begin{tabular}{|p{3cm}|p{10cm}|} 
        \hline
        \textbf{ID} & \textbf{Descrizione} \\  
        \hline
        D-O1 & Le inserzioni degli Immobili possono ricevere offerte da parte degli utenti, composte da un prezzo proposto e dalle credenziali dell’offerente. Le offerte sono visibili a tutti gli utenti. \\ 
        \hline
    \end{tabular}
    
\end{table}
\subsection*{Gestione delle notifiche}

\begin{table}[H]
    \centering
    \renewcommand{\arraystretch}{1.3} % Aumenta leggermente l'altezza delle righe
    
    \begin{tabular}{|p{3cm}|p{10cm}|} 
        \hline
        \textbf{ID} & \textbf{Descrizione} \\  
        \hline
        D-N1 & Il sistema di notifiche consente di inviare avvisi personalizzati agli utenti, categorizzati in base a eventi come: nuove proprietà in linea con ricerche precedenti, conferme o rifiuti di proposte, e messaggi promozionali. \\ 
        \hline
        D-N2 & Le notifiche promozionali sono composte da un contenuto in formato RICH TEXT, definito dagli amministratori, e sono visibili esclusivamente agli utenti iscritti alla relativa newsletter. \\ 
        \hline
    \end{tabular}
    
\end{table}


% punto e. I)
\section{Descrizione Testuale strutturata del Sistema}
In questa sezione faremo un analisi più approfondita di alcuni dei casi d'uso campione del sistema, per esporre il nostro ragionamento per modellare e sviluppare il Software .\\
Questa descrizione ci servirà per pensare a che tipo di interazioni con l'Utente sono necessarie, i passaggi da effettuare  dal sistema in risposta alle richieste ricevute e 
I casi d'uso scelti per questo scopo sono la Creazione di un nuovo Annuncio di un Immobile e Modifica notifiche in Arrivo.
  
    \subsection*{Creazione Nuovo Annuncio per Immobile}
% Please add the following required packages to your document preamble:
% \usepackage{multirow}
% \usepackage[table,xcdraw]{xcolor}
% Beamer presentation requires \usepackage{colortbl} instead of \usepackage[table,xcdraw]{xcolor}
% \usepackage{longtable}
% Note: It may be necessary to compile the document several times to get a multi-page table to line up properly

\subsection{Cockburn: nuovo annuncio}

\begin{longtable}{|l|lll|}
\caption{Creazione nuovo Annuncio}
\label{tab:my-table}\\
\hline
\rowcolor[HTML]{B2C9AB} 
\textbf{Use Case 1}                                                                                                                                                                              & \multicolumn{3}{l|}{\cellcolor[HTML]{B2C9AB}\textbf{Creazione Nuovo Annuncio per Immobile}}                                                                                                                                                                                                                                                                                                                   \\ \hline
\endhead
%
\cellcolor[HTML]{B2C9AB}\textbf{Goal In Context}                                                                                                                                                 & \multicolumn{3}{l|}{\begin{tabular}[c]{@{}l@{}}Inserimento nel Catalogo dell'Agenzia Immobiliare\\ un nuovo Annuncio per Immobile.\end{tabular}}                                                                                                                                                                                                                                                              \\ \hline
\cellcolor[HTML]{B2C9AB}\textbf{Preconditions}                                                                                                                                                   & \multicolumn{3}{l|}{\begin{tabular}[c]{@{}l@{}}- Login effettuato con account con\\ ruolo Agente o Amministratore. \\ - Si trova nella pagina di Creazione Annunci.\end{tabular}}                                                                                                                                                                                                                             \\ \hline
\cellcolor[HTML]{B2C9AB}\textbf{\begin{tabular}[c]{@{}l@{}}Success End\\ Conditions\end{tabular}}                                                                                                & \multicolumn{3}{l|}{\begin{tabular}[c]{@{}l@{}}Aggiunta in Catalogo dell'Immobile e Salvataggio\\ nel Database.\end{tabular}}                                                                                                                                                                                                                                                                                 \\ \hline
\cellcolor[HTML]{B2C9AB}\textbf{Primary Actor}                                                                                                                                                   & \multicolumn{3}{l|}{Agente}                                                                                                                                                                                                                                                                                                                                                                                   \\ \hline
\cellcolor[HTML]{B2C9AB}\textbf{Trigger}                                                                                                                                                         & \multicolumn{3}{l|}{\begin{tabular}[c]{@{}l@{}}Agente preme il bottone Aggiungi \\ Immobile nella Schermata del Catalogo.\end{tabular}}                                                                                                                                                                                                                                                                       \\ \hline
\rowcolor[HTML]{B2C9AB} 
\textbf{Descrizione}                                                                                                                                                                             & \multicolumn{1}{l|}{\cellcolor[HTML]{B2C9AB}\textbf{Step n.}} & \multicolumn{1}{l|}{\cellcolor[HTML]{B2C9AB}\textbf{Agente}}                                                                                      & \textbf{Sistema}                                                                                                                                                                          \\ \hline
\cellcolor[HTML]{B2C9AB}{\color[HTML]{000000} }                                                                                                                                                  & \multicolumn{1}{l|}{1}                                        & \multicolumn{1}{l|}{}                                                                                                                             & \textit{\begin{tabular}[c]{@{}l@{}}Controlla che non ci siano\\ operazioni in sospeso\end{tabular}}                                                                                       \\ \cline{2-4} 
\cellcolor[HTML]{B2C9AB}{\color[HTML]{000000} }                                                                                                                                                  & \multicolumn{1}{l|}{2}                                        & \multicolumn{1}{l|}{}                                                                                                                             & \textit{\begin{tabular}[c]{@{}l@{}}Mostra la\\ pagina di creazione \\ Annunci.\end{tabular}}                                                                                              \\ \cline{2-4} 
\cellcolor[HTML]{B2C9AB}{\color[HTML]{000000} }                                                                                                                                                  & \multicolumn{1}{l|}{3}                                        & \multicolumn{1}{l|}{\begin{tabular}[c]{@{}l@{}}Compila 3 campi:\\ - Titolo Annuncio.\\ - Tipologia contratto.\\ Tipologia Immobile.\end{tabular}} &                                                                                                                                                                                           \\ \cline{2-4} 
\cellcolor[HTML]{B2C9AB}{\color[HTML]{000000} }                                                                                                                                                  & \multicolumn{1}{l|}{4}                                        & \multicolumn{1}{l|}{}                                                                                                                             & \textit{\begin{tabular}[c]{@{}l@{}}In base ai dati inseriti \\ mostra il form adatto \\ per la creazione.\end{tabular}}                                                                   \\ \cline{2-4} 
\cellcolor[HTML]{B2C9AB}{\color[HTML]{000000} }                                                                                                                                                  & \multicolumn{1}{l|}{5}                                        & \multicolumn{1}{l|}{\begin{tabular}[c]{@{}l@{}}Compila Fom\\ Annuncio\\ \\ \end{tabular}}                                                                                               & \textit{}                                                                                                                                                                                 \\ \cline{2-4} 
\cellcolor[HTML]{B2C9AB}{\color[HTML]{000000} }                                                                                                                                                  & \multicolumn{1}{l|}{6}                                        & \multicolumn{1}{l|}{Click Bottone Conferma.}                                                                                                      & \textit{}                                                                                                                                                                                 \\ \cline{2-4} 
\cellcolor[HTML]{B2C9AB}{\color[HTML]{000000} }                                                                                                                                                  & \multicolumn{1}{l|}{7}                                        & \multicolumn{1}{l|}{}                                                                                                                             & \textit{\begin{tabular}[c]{@{}l@{}}Mostra Riepilogo Dati\\ Inseriti.\end{tabular}}                                                                                                        \\ \cline{2-4} 
\cellcolor[HTML]{B2C9AB}{\color[HTML]{000000} }                                                                                                                                                  & \multicolumn{1}{l|}{8}                                        & \multicolumn{1}{l|}{Click Bottone Pubblica.}                                                                                                      & \textit{}                                                                                                                                                                                 \\ \cline{2-4} 
\cellcolor[HTML]{B2C9AB}{\color[HTML]{000000} }                                                                                                                                                  & \multicolumn{1}{l|}{9}                                        & \multicolumn{1}{l|}{}                                                                                                                             & \textit{Validazione Dei Dati}                                                                                                                                                             \\ \cline{2-4} 
\cellcolor[HTML]{B2C9AB}{\color[HTML]{000000} }                                                                                                                                                  & \multicolumn{1}{l|}{10}                                       & \multicolumn{1}{l|}{}                                                                                                                             & \textit{Salvataggio in Database}                                                                                                                                                          \\ \cline{2-4} 
\cellcolor[HTML]{B2C9AB}{\color[HTML]{000000} }                                                                                                                                                  & \multicolumn{1}{l|}{11}                                       & \multicolumn{1}{l|}{}                                                                                                                             & \textit{\begin{tabular}[c]{@{}l@{}}Mostra Popup\\ Operazione completata\\ con successo.\end{tabular}}                                                                                     \\ \cline{2-4} 
\multirow{-43}{*}{\cellcolor[HTML]{B2C9AB}{\color[HTML]{000000} \begin{tabular}[c]{@{}l@{}}Scenario \\ Principale\end{tabular}}}                                                                 & \multicolumn{1}{l|}{12}                                       & \multicolumn{1}{l|}{}                                                                                                                             & \textit{\begin{tabular}[c]{@{}l@{}}Use Case terminato con \\ Successo.\end{tabular}}                                                                                                      \\ \hline
\newpage
\rowcolor[HTML]{B2C9AB} 
\textbf{Extension}                                                                                                                                                                               & \multicolumn{1}{l|}{\cellcolor[HTML]{B2C9AB}\textbf{Step n.}} & \multicolumn{1}{l|}{\cellcolor[HTML]{B2C9AB}\textbf{Utente}}                                                                                      & \textbf{Sistema}                                                                                                                                                                          \\ \hline
\cellcolor[HTML]{B2C9AB}                                                                                                                                                                         & \multicolumn{1}{l|}{A.2}                                      & \multicolumn{1}{l|}{}                                                                                                                             & \textit{\begin{tabular}[c]{@{}l@{}}Vede che c'è un operazione\\ di creazione in sospeso\end{tabular}}                                                                                     \\ \cline{2-4} 
\cellcolor[HTML]{B2C9AB}                                                                                                                                                                         & \multicolumn{1}{l|}{A.3}                                      & \multicolumn{1}{l|}{}                                                                                                                             & \textit{\begin{tabular}[c]{@{}l@{}}Mostra Popup chiedendo se \\ vuole continuare \\ l'operazione o creare un \\ nuovo Annuncio\end{tabular}}                                              \\ \cline{2-4} 
\cellcolor[HTML]{B2C9AB}                                                                                                                                                                         & \multicolumn{1}{l|}{A.4}                                      & \multicolumn{1}{l|}{Click su Nuovo Annuncio}                                                                                                      & \textit{}                                                                                                                                                                                 \\ \cline{2-4} 
\cellcolor[HTML]{B2C9AB}                                                                                                                                                                         & \multicolumn{1}{l|}{A.5}                                      & \multicolumn{1}{l|}{}                                                                                                                             & \textit{\begin{tabular}[c]{@{}l@{}}Mostra Dialog per avvisare \\ che i dati dell Annuncio \\ in sospeso verranno persi\end{tabular}}                                                      \\ \cline{2-4} 
\cellcolor[HTML]{B2C9AB}                                                                                                                                                                         & \multicolumn{1}{l|}{A.6}                                      & \multicolumn{1}{l|}{Click Conferma}                                                                                                               & \textit{}                                                                                                                                                                                 \\ \cline{2-4} 
\multirow{-18}{*}{\cellcolor[HTML]{B2C9AB}\begin{tabular}[c]{@{}l@{}}Sistema riconosce\\ che c’è un Annuncio\\ in creazione sospeso.\\ Utente vuole creare\\ un Nuovo Annuncio\end{tabular}}      & \multicolumn{1}{l|}{A.7}                                      & \multicolumn{1}{l|}{}                                                                                                                             & \textit{\begin{tabular}[c]{@{}l@{}}Vai al passo 2 dello \\ Scenario Principale\end{tabular}}                                                                                              \\ \hline
\cellcolor[HTML]{B2C9AB}                                                                                                                                                                         & \multicolumn{1}{l|}{B.2}                                      & \multicolumn{1}{l|}{}                                                                                                                             & \textit{\begin{tabular}[c]{@{}l@{}}Vede che c'è un \\ operazione di\\ creazione in sospeso\end{tabular}}                                                                                  \\ \cline{2-4} 
\cellcolor[HTML]{B2C9AB}                                                                                                                                                                         & \multicolumn{1}{l|}{B.3}                                      & \multicolumn{1}{l|}{}                                                                                                                             & \textit{\begin{tabular}[c]{@{}l@{}}Mostra Popup chiedendo se \\ vuole continuare \\ l'operazione o creare un \\ nuovo Annuncio\end{tabular}}                                              \\ \cline{2-4} 
\cellcolor[HTML]{B2C9AB}                                                                                                                                                                         & \multicolumn{1}{l|}{B.4}                                      & \multicolumn{1}{l|}{\begin{tabular}[c]{@{}l@{}}Click su Continua\\ Operazione\\ \\ \end{tabular}}                                                                                                & \textit{}                                                                                                                                                                                 \\ \cline{2-4} 
\cellcolor[HTML]{B2C9AB}                                                                                                                                                                         & \multicolumn{1}{l|}{B.5}                                      & \multicolumn{1}{l|}{\textit{}}                                                                                                                    & \textit{\begin{tabular}[c]{@{}l@{}}Mostra la pagina\\ di creazione caricando i\\ dati inseriti nell’annuncio\\ sospeso\end{tabular}}                                                      \\ \cline{2-4} 
\multirow{-19}{*}{\cellcolor[HTML]{B2C9AB}\begin{tabular}[c]{@{}l@{}}Sistema riconosce\\ che c’è un Annuncio\\ in creazione sospeso.\\ Utente vuole continuare\\ operazione sospesa\end{tabular}} & \multicolumn{1}{l|}{B.6}                                      & \multicolumn{1}{l|}{}                                                                                                                             & \textit{\begin{tabular}[c]{@{}l@{}}Vai al passo 4 dello \\ Scenario Principale\end{tabular}}                                                                                              \\ \hline
\cellcolor[HTML]{B2C9AB}                                                                                                                                                                         & \multicolumn{1}{l|}{C.9}                                      & \multicolumn{1}{l|}{}                                                                                                                             & \textit{\begin{tabular}[c]{@{}l@{}}Mostra Messaggio di Errore \\ che avverte che l'operazione \\ non è andata buon fine e \\ che i dati sono stati \\ preservati in locale.\end{tabular}} \\ \cline{2-4} 
\cellcolor[HTML]{B2C9AB}                                                                                                                                                                         & \multicolumn{1}{l|}{C.10}                                     & \multicolumn{1}{l|}{}                                                                                                                             & \textit{Salva dati in locale.}                                                                                                                                                            \\ \cline{2-4} 
\multirow{-8}{*}{\cellcolor[HTML]{B2C9AB}\begin{tabular}[c]{@{}l@{}}Il Sistema è Offline \\ al momento della\\ Pubblicazione oppure\\ non è Collegato a\\ Internet\end{tabular}}                 & \multicolumn{1}{l|}{C.11}                                     & \multicolumn{1}{l|}{}                                                                                                                             & \textit{\begin{tabular}[c]{@{}l@{}}Chiude la schermata.\\ Use Case Terminato in \\ Fallimento.\end{tabular}}                                                                              \\ \hline
\cellcolor[HTML]{B2C9AB}                                                                                                                                                                         & \multicolumn{1}{l|}{D.10}                                     & \multicolumn{1}{l|}{}                                                                                                                             & \begin{tabular}[c]{@{}l@{}}Evidenzia campi del form\\ non riempiti/validi\end{tabular}                                                                                                    \\ \cline{2-4} 
\multirow{-4}{*}{\cellcolor[HTML]{B2C9AB}Dati Inseriti non validi}                                                                                                                               & \multicolumn{1}{l|}{D.11}                                     & \multicolumn{1}{l|}{}                                                                                                                             & \begin{tabular}[c]{@{}l@{}}Vai al passo 5 dello\\ Scenario Principale\end{tabular}                                                                                                        \\ \hline
\end{longtable}
\newpage
\subsection*{Modifica Stato Notifiche in arrivo}
% Please add the following required packages to your document preamble:
% \usepackage{multirow}
% \usepackage[table,xcdraw]{xcolor}
% Beamer presentation requires \usepackage{colortbl} instead of \usepackage[table,xcdraw]{xcolor}
% \usepackage{longtable}
% Note: It may be necessary to compile the document several times to get a multi-page table to line up properly

\subsection{Cockburn: Attivazione e Disattivazione categorie notifiche}

\begin{longtable}{|l|lll|}
\caption{}
\label{tab:my-table}\\
\hline
\rowcolor[HTML]{E1D5E7} 
\textbf{Use Case 2}                                                                                                                                                                 & \multicolumn{3}{l|}{\cellcolor[HTML]{E1D5E7}\textbf{Modifica notifiche in Arrivo}}                                                                                                                                                                                                                                                                                                                        \\ \hline
\endhead
%
\cellcolor[HTML]{E1D5E7}\textbf{Goal In Context}                                                                                                                                    & \multicolumn{3}{l|}{Cambia lo stato di notiche a scelta}                                                                                                                                                                                                                                                                                                                                                  \\ \hline
\cellcolor[HTML]{E1D5E7}\textbf{Preconditions}                                                                                                                                      & \multicolumn{3}{l|}{\begin{tabular}[c]{@{}l@{}}- Login effettuato con account con\\ ruolo Utente. \\ - Si trova nella pagina di Visualizazione notifiche\end{tabular}}                                                                                                                                                                                                                                    \\ \hline
\cellcolor[HTML]{E1D5E7}\textbf{\begin{tabular}[c]{@{}l@{}}Success End\\ Conditions\end{tabular}}                                                                                   & \multicolumn{3}{l|}{Lo stato di abilitazione di notifica viene modificato}                                                                                                                                                                                                                                                                                                                                \\ \hline
\cellcolor[HTML]{E1D5E7}\textbf{Primary Actor}                                                                                                                                      & \multicolumn{3}{l|}{Utente}                                                                                                                                                                                                                                                                                                                                                                               \\ \hline
\cellcolor[HTML]{E1D5E7}\textbf{Trigger}                                                                                                                                            & \multicolumn{3}{l|}{Utente preme il bottone Gestisci Notifiche}                                                                                                                                                                                                                                                                                                                                           \\ \hline
\rowcolor[HTML]{E1D5E7} 
\textbf{Descrizione}                                                                                                                                                                & \multicolumn{1}{l|}{\cellcolor[HTML]{E1D5E7}\textbf{Step n.}} & \multicolumn{1}{l|}{\cellcolor[HTML]{E1D5E7}\textbf{Utente}}                                                                            & \textbf{Sistema}                                                                                                                                                                                \\ \hline
\cellcolor[HTML]{E1D5E7}                                                                                                                                                            & \multicolumn{1}{l|}{1}                                        & \multicolumn{1}{l|}{}                                                                                                                   & \textit{\begin{tabular}[c]{@{}l@{}}Mostra pagina delle\\ notifiche\end{tabular}}                                                                                                                \\ \cline{2-4} 
\cellcolor[HTML]{E1D5E7}                                                                                                                                                            & \multicolumn{1}{l|}{2}                                        & \multicolumn{1}{l|}{\begin{tabular}[c]{@{}l@{}}Click Bottone per\\ modificare stato notifica\end{tabular}}                              &                                                                                                                                                                                                 \\ \cline{2-4} 
\cellcolor[HTML]{E1D5E7}                                                                                                                                                            & \multicolumn{1}{l|}{3}                                        & \multicolumn{1}{l|}{}                                                                                                                   & \textit{\begin{tabular}[c]{@{}l@{}}Mosta una Popup con \\ elenco di tutte le \\ categorie che possono \\ essere Modificate\end{tabular}}                                                        \\ \cline{2-4} 
\cellcolor[HTML]{E1D5E7}                                                                                                                                                            & \multicolumn{1}{l|}{4}                                        & \multicolumn{1}{l|}{\begin{tabular}[c]{@{}l@{}}Click su la casella/e \\ relativa/e alla categoria/e \\ da Modificare\end{tabular}}      & \textit{}                                                                                                                                                                                       \\ \cline{2-4} 
\cellcolor[HTML]{E1D5E7}                                                                                                                                                            & \multicolumn{1}{l|}{5}                                        & \multicolumn{1}{l|}{Click Bottone Conferma}                                                                                             & \textit{}                                                                                                                                                                                       \\ \cline{2-4} 
\cellcolor[HTML]{E1D5E7}                                                                                                                                                            & \multicolumn{1}{l|}{6}                                        & \multicolumn{1}{l|}{}                                                                                                                   & \textit{\begin{tabular}[c]{@{}l@{}}Mostra Dialog con \\ un messaggio che \\ avverte l'Utente \\ che non riceverà \\ più nuove Notifiche \\ relative alla \\ categoria selezionata\end{tabular}} \\ \cline{2-4} 
\cellcolor[HTML]{E1D5E7}                                                                                                                                                            & \multicolumn{1}{l|}{7}                                        & \multicolumn{1}{l|}{\begin{tabular}[c]{@{}l@{}}Click Bottone Conferma \\ nella Dialog\end{tabular}}                                     & \textit{}                                                                                                                                                                                       \\ \cline{2-4} 
\cellcolor[HTML]{E1D5E7}                                                                                                                                                            & \multicolumn{1}{l|}{8}                                        & \multicolumn{1}{l|}{}                                                                                                                   & \textit{\begin{tabular}[c]{@{}l@{}}Modifica Stato \\ Notifica in Database\end{tabular}}                                                                                                         \\ \cline{2-4} 
\cellcolor[HTML]{E1D5E7}                                                                                                                                                            & \multicolumn{1}{l|}{9}                                        & \multicolumn{1}{l|}{}                                                                                                                   & \textit{\begin{tabular}[c]{@{}l@{}}Feedback visivo del \\ cambio di visibilità\end{tabular}}                                                                                                    \\ \cline{2-4} 
\multirow{-50}{*}{\cellcolor[HTML]{E1D5E7}\textbf{\begin{tabular}[c]{@{}l@{}}Scenario \\ Principale\end{tabular}}}                                                                  & \multicolumn{1}{l|}{10}                                       & \multicolumn{1}{l|}{}                                                                                                                   & \textit{\begin{tabular}[c]{@{}l@{}}Use Case terminato \\ con Successo\end{tabular}}                                                                                                             \\ \hline
\newpage
\rowcolor[HTML]{E1D5E7} 
\textbf{Extension}                                                                                                                                                                  & \multicolumn{1}{l|}{\cellcolor[HTML]{E1D5E7}\textbf{Step n.}} & \multicolumn{1}{l|}{\cellcolor[HTML]{E1D5E7}\textbf{Utente}}                                                                            & \textbf{Sistema}                                                                                                                                                                                \\ \hline
\cellcolor[HTML]{E1D5E7}                                                                                                                                                            & \multicolumn{1}{l|}{A.2}                                      & \multicolumn{1}{l|}{click su notifica ricevuta}                                                                                         & \textit{}                                                                                                                                                                                       \\ \cline{2-4} 
\cellcolor[HTML]{E1D5E7}                                                                                                                                                            & \multicolumn{1}{l|}{A.3}                                      & \multicolumn{1}{l|}{}                                                                                                                   & \textit{\begin{tabular}[c]{@{}l@{}}Mostra il contenuto \\ della notifica\end{tabular}}                                                                                                          \\ \cline{2-4} 
\cellcolor[HTML]{E1D5E7}                                                                                                                                                            & \multicolumn{1}{l|}{A.4}                                      & \multicolumn{1}{l|}{\begin{tabular}[c]{@{}l@{}}Click Bottone Disattiva\\ Notifica\end{tabular}}                                         & \textit{}                                                                                                                                                                                       \\ \cline{2-4} 
\cellcolor[HTML]{E1D5E7}                                                                                                                                                            & \multicolumn{1}{l|}{A.5}                                      & \multicolumn{1}{l|}{}                                                                                                                   & \textit{\begin{tabular}[c]{@{}l@{}}Bottone\\ Disattiva Notifica \\ diventa Bottone \\ Attiva Notifica\end{tabular}}                                                                             \\ \cline{2-4} 

\multirow{-15}{*}{\cellcolor[HTML]{E1D5E7}\begin{tabular}[c]{@{}l@{}}Utente vuole \\ disabilitare una categoria \\ di notifiche a partire da \\ una notifica ricevuta\end{tabular}}    & \multicolumn{1}{l|}{A.6}                                      & \multicolumn{1}{l|}{}                                                                                                                   & \textit{\begin{tabular}[c]{@{}l@{}}Vai allo step 6 dello\\ Scenario Principale\end{tabular}}                                                                                                    \\ \hline
\cellcolor[HTML]{E1D5E7}                                                                                                                                                            & \multicolumn{1}{l|}{B.6}                                      & \multicolumn{1}{l|}{}                                                                                                                   & \textit{\begin{tabular}[c]{@{}l@{}}Il sistema nota che \\ non ci sono state \\ modifiche\end{tabular}}                                                                                          \\ \cline{2-4} 
\multirow{-4}{*}{\cellcolor[HTML]{E1D5E7}\begin{tabular}[c]{@{}l@{}}Utente non modifica \\ nessuna categoria \\ durante lo \\ Scenario Principale\end{tabular}}                     & \multicolumn{1}{l|}{B.7}                                      & \multicolumn{1}{l|}{}                                                                                                                   & \textit{\begin{tabular}[c]{@{}l@{}}Use case Terminato \\ con Successo\end{tabular}}                                                                                                             \\ \hline
\cellcolor[HTML]{E1D5E7}                                                                                                                                                            & \multicolumn{1}{l|}{C.2}                                      & \multicolumn{1}{l|}{\begin{tabular}[c]{@{}l@{}}Utente preme Bottone\\ Disabilita dalla lista delle\\ categorie Visibili\end{tabular}}   &                                                                                                                                                                                                 \\ \cline{2-4} 
\cellcolor[HTML]{E1D5E7}                                                                                                                                                            & \multicolumn{1}{l|}{C.3}                                      & \multicolumn{1}{l|}{}                                                                                                                   & \textit{\begin{tabular}[c]{@{}l@{}}Bottone Disabilita \\ diventa\\ Bottone Abilita\end{tabular}}                                                                                                \\ \cline{2-4} 
\multirow{-6}{*}{\cellcolor[HTML]{E1D5E7}\begin{tabular}[c]{@{}l@{}}Utente vuole disabilitare \\ categoria dalla lista \\ delle Categorie Attive\end{tabular}}                      & \multicolumn{1}{l|}{C.4}                                      & \multicolumn{1}{l|}{\textit{}}                                                                                                          & \textit{\begin{tabular}[c]{@{}l@{}}Vai al Passo 6 dello \\ Scenario Principale\end{tabular}}                                                                                                    \\ \hline
\cellcolor[HTML]{E1D5E7}                                                                                                                                                            & \multicolumn{1}{l|}{D.2}                                      & \multicolumn{1}{l|}{\begin{tabular}[c]{@{}l@{}}Utente Preme Bottone\\ Abilita dalla lista delle \\ Categorie disabilitate\end{tabular}} &                                                                                                                                                                                                 \\ \cline{2-4} 
\cellcolor[HTML]{E1D5E7}                                                                                                                                                            & \multicolumn{1}{l|}{D.3}                                      & \multicolumn{1}{l|}{}                                                                                                                   & \textit{\begin{tabular}[c]{@{}l@{}}Bottone Abilita \\ diventa\\ Bottone Disabilita\end{tabular}}                                                                                                \\ \cline{2-4} 
\multirow{-6}{*}{\cellcolor[HTML]{E1D5E7}\begin{tabular}[c]{@{}l@{}}Utente vuole abilitare\\ categoria dalla lista delle\\ Categorie Disabilitate\end{tabular}}                     & \multicolumn{1}{l|}{D.4}                                      & \multicolumn{1}{l|}{\textit{}}                                                                                                          & \textit{\begin{tabular}[c]{@{}l@{}}Vai al Passo 8 dello \\ Scenario Principale\end{tabular}}                                                                                                    \\ \hline
\cellcolor[HTML]{E1D5E7}                                                                                                                                                            & \multicolumn{1}{l|}{E.7}                                      & \multicolumn{1}{l|}{}                                                                                                                   & \textit{\begin{tabular}[c]{@{}l@{}}Mostra messaggio di \\ Errore il quale \\ avverte che \\ l'operazione non è \\ andata a buon fine.\end{tabular}}                                             \\ \cline{2-4} 
\cellcolor[HTML]{E1D5E7}                                                                                                                                                            & \multicolumn{1}{l|}{}                                         & \multicolumn{1}{l|}{}                                                                                                                   &                                                                                                                                                                                                 \\
\cellcolor[HTML]{E1D5E7}                                                                                                                                                            & \multicolumn{1}{l|}{}                                         & \multicolumn{1}{l|}{}                                                                                                                   &                                                                                                                                                                                                 \\
\multirow{-7}{*}{\cellcolor[HTML]{E1D5E7}\begin{tabular}[c]{@{}l@{}}Il sistema è Offline al \\ click del bottone \\ dell’Utente oppure \\ non è collegato a Internet.\end{tabular}} & \multicolumn{1}{l|}{\multirow{-3}{*}{E.8}}                    & \multicolumn{1}{l|}{\multirow{-3}{*}{}}                                                                                                 & \multirow{-3}{*}{\textit{\begin{tabular}[c]{@{}l@{}}Use Case Terminato \\ in Fallimento.\end{tabular}}}                                                                                         \\ \hline
\newpage
\cellcolor[HTML]{E1D5E7}                                                                                                                                                            & \multicolumn{1}{l|}{\cellcolor[HTML]{FFFFFF}F.2}              & \multicolumn{1}{l|}{Click su notifica ricevuta}                                                                                         & \textit{}                                                                                                                                                                                       \\ \cline{2-4} 
\cellcolor[HTML]{E1D5E7}                                                                                                                                                            & \multicolumn{1}{l|}{\cellcolor[HTML]{FFFFFF}F.3}              & \multicolumn{1}{l|}{}                                                                                                                   & \textit{\begin{tabular}[c]{@{}l@{}}Mostra il contenuto \\ della notifica\end{tabular}}                                                                                                          \\ \cline{2-4} 
\cellcolor[HTML]{E1D5E7}                                                                                                                                                            & \multicolumn{1}{l|}{\cellcolor[HTML]{FFFFFF}F.4}              & \multicolumn{1}{l|}{\begin{tabular}[c]{@{}l@{}}Click Bottone Attiva\\ Notifica\end{tabular}}                                            &                                                                                                                                                                                                 \\ \cline{2-4} 
\cellcolor[HTML]{E1D5E7}                                                                                                                                                            & \multicolumn{1}{l|}{F.5}                                      & \multicolumn{1}{l|}{}                                                                                                                   & \textit{\begin{tabular}[c]{@{}l@{}}Bottone \\ Attiva Notifiche\\ diventa Bottone \\ Disabilita Notifiche\end{tabular}}                                                                          \\ 

\cline{2-4} 
\multirow{-15}{*}{\cellcolor[HTML]{E1D5E7}\begin{tabular}[c]{@{}l@{}}Utente vuole abilitare \\ una categoria di \\ notifiche a partire \\ da una notica ricevuta\end{tabular}}       & \multicolumn{1}{l|}{\cellcolor[HTML]{FFFFFF}F.6}              & \multicolumn{1}{l|}{\cellcolor[HTML]{FFFFFF}\textbf{}}                                                                                  & \textit{\begin{tabular}[c]{@{}l@{}}Vai al Passo 8 dello \\ Scenario Principale\end{tabular}}                                                                                                    \\ \hline
\end{longtable}
% punto e.II)
\newpage
\section{Mock up}
\subsection*{Introduzione}
Dopo aver definito gli use case utilizzando il metodo Cockburn, siamo passati alla fase di progettazione visiva realizzando una serie di mockup interattivi in Figma. Questi mockup hanno lo scopo di rappresentare graficamente le interazioni dell'utente con il sistema, traducendo le specifiche funzionali in un'interfaccia visibile e navigabile. Ogni mockup è stato sviluppato tenendo conto delle diverse casistiche previste negli use case e nelle loro estensioni, in modo da garantire un’esperienza utente coerente e intuitiva.

Nei paragrafi seguenti verranno presentati i vari mockup, suddivisi per use case. Ogni sezione illustrerà le scelte progettuali effettuate, evidenziando come la UI si adatti alle diverse situazioni previste dai casi d'uso.

\subsection{Caso d'Uso: Aggiungi un Annuncio Immobiliare}

Per garantire un’esperienza utente ottimale, abbiamo progettato i mockup del caso d'uso \textit{Aggiungi un Annuncio Immobiliare} seguendo i principi della user experience (UX) e dell'usabilità. Il percorso utente è stato studiato attentamente per minimizzare il carico cognitivo e semplificare l’interazione, in linea con i modelli proposti da Nielsen e Norman \cite{nielsen1995,norman1988}.

\subsection*{Schermata Iniziale: Gestione Annunci Immobiliari}
L’azione di aggiungere un nuovo annuncio parte dalla schermata di Gestione Annunci Immobiliari, dove l’utente trova:
\begin{itemize}
    \item \textbf{Una lista degli Immobili a lui associati}, che consente una rapida contestualizzazione.
    \item \textbf{Un pulsante unico ed evidente}, etichettato “Aggiungi Annuncio Immobiliare”, che centralizza l’azione primaria.
\end{itemize}

\subsection*{Flusso di Navigazione e Segmentazione del Form}
Al click del pulsante, l’utente viene indirizzato a una schermata dedicata alla compilazione di un form articolato in più step. Questa segmentazione si basa sul principio del \textbf{chunking dell’informazione} \cite{miller1956}, riducendo il carico di memoria e rendendo il processo meno gravoso. In particolare:
\begin{itemize}
    \item \textbf{Step 1:} Richiede le informazioni basilari, quali il titolo, il tipo di contratto e il tipo di immobile.
    \item \textbf{Step 2 e successivi:} Raccolgono i dati principali e le caratteristiche secondarie dell’immobile, organizzati in modo logico e intuitivo.
\end{itemize}

\subsection*{Navigazione Sticky e Accessibilità degli Step}
I pulsanti di navigazione, posizionati in alto con comportamento \textit{sticky}, consentono all’utente di passare agevolmente da uno step all’altro, anche in presenza di schermate particolarmente lunghe. Tale scelta:
\begin{itemize}
    \item Riduce il tempo necessario per individuare i controlli di navigazione.
    \item Favorisce un’interazione continua e fluida, in linea con i principi di design centrato sull’utente e le best practice in ambito HCI (Human-Computer Interaction) \cite{shneiderman2004}.
\end{itemize}

\subsection*{Integrazione della Mappa Interattiva}
Per quanto riguarda l’inserimento dell’indirizzo:
\begin{itemize}
    \item \textbf{La mappa interattiva} permette di visualizzare immediatamente la posizione indicata, offrendo un feedback visivo diretto.
    \item L’utente ha la possibilità di regolare la posizione direttamente sulla mappa, migliorando la precisione del dato inserito \cite{wickens2008}.
\end{itemize}

\subsection*{Gestione delle Immagini e Feedback Visivo}
La sezione dedicata alle immagini è progettata per:
\begin{itemize}
    \item \textbf{Consentire l’aggiunta di foto} tramite pulsante dedicato o drag and drop, facilitando l’upload in maniera intuitiva.
    \item \textbf{Permettere l’inserimento di una descrizione} per ogni immagine, migliorando la contestualizzazione visiva dell’annuncio \cite{pieters2004}.
\end{itemize}

\subsection*{Anteprima e Conferma Finale}
Infine, viene presentato uno schema riepilogativo che funge da anteprima dell’annuncio così come apparirà agli utenti finali. Una volta verificata la correttezza dei dati:
\begin{itemize}
    \item L’utente può cliccare il pulsante \textbf{“Pubblica”}, che innesca un processo di caricamento e validazione.
    \item Al termine del processo, viene mostrato un messaggio di conferma che attesta il successo dell’operazione, riducendo l’ansia da incertezza e rafforzando la fiducia nel sistema \cite{nielsen1995}.
\end{itemize}

\newpage

\begin{figure}[ht]
    \centering
    \begin{tikzpicture}[node distance=1.5cm and 1cm, auto]
        % Nodo per immagine 1 con didascalia sotto
        \node (img1) {
            \begin{tabular}{c}
                \includegraphics[width=0.7\textwidth]{Immagini/Mockup/aggiungi annuncio/scenario principale/step1.png} \\
                Cockburn: step 3
            \end{tabular}
        };
        
        % Nodo per immagine 2 con didascalia sotto, posizionato a destra di img1
        \node (img2) [below=of img1] {
            \begin{tabular}{c}
                \includegraphics[width=0.7\textwidth]{Immagini/Mockup/aggiungi annuncio/scenario principale/step2.png} \\
                Cockburn: step4/5
            \end{tabular}
        };
        
        % Nodo per immagine 3 con didascalia sotto, posizionato sotto img2
        \node (img3) [below=of img2] {
            \begin{tabular}{c}
                \includegraphics[width=0.7\textwidth]{Immagini/Mockup/aggiungi annuncio/scenario principale/step3.png} \\
                Cockburn: step 4/5
            \end{tabular}
        };
        
        % Disegna le frecce
        \draw[->, thick] (img1) -- (img2);
        \draw[->, thick] (img2) -- (img3);
      
    \end{tikzpicture}
    \caption{Mockup: scenario principale della tabella di Cockburn del caso d'uso nuovo annuncio.}
    \label{fig:tikz_flow}
\end{figure}

\newpage



\begin{figure}[ht]
    \centering
    \begin{tikzpicture}[node distance=1.5cm and 1cm, auto]
        % Nodo per immagine 1 con didascalia sotto
        \node (img1) {
            \begin{tabular}{c}
                \includegraphics[width=\textwidth,keepaspectratio]{Immagini/Mockup/aggiungi annuncio/scenario principale/step4.png} \\
                Cockburn: step 4/5
            \end{tabular}
        };
        
        % Nodo per immagine 2 con didascalia sotto, posizionato a destra di img1
        \node (img2) [below=of img1] {
            \begin{tabular}{c}
                \includegraphics[width=\textwidth,keepaspectratio]{Immagini/Mockup/aggiungi annuncio/scenario principale/step5.png} \\
                Cockburn: step 6
            \end{tabular}
        };
        
        % Disegna le frecce
        \draw[->, thick] (img1) -- (img2);
      
    \end{tikzpicture}
    \caption{Mockup: scenario principale della tabella di Cockburn del caso d'uso nuovo annuncio.}
    \label{fig:mockup_scenario_principale_parte2_aggiungi_annuncio}
\end{figure}

\newpage

\input{Requisiti Del Software/Analisi dei Requisiti/Mockup/aggiungi annuncio/scenario principale Parte III}



\clearpage
\newpage

\subsubsection{Estensione A: Disattivazione Notifiche dalla Visualizzazione di una Notifica}

Per offrire un maggiore controllo sulla gestione delle notifiche senza interrompere l’esperienza utente, il sistema permette di disattivare una categoria direttamente dalla visualizzazione di una notifica specifica. Questa variante è progettata per garantire una modifica consapevole delle preferenze, evitando azioni impulsive che potrebbero compromettere la ricezione di informazioni rilevanti.

\vspace{0.5cm}
\subsubsection{Interfaccia e Comportamento del Bottone}
Alla fine del testo di ogni notifica, se la relativa categoria è attiva, è presente un pulsante neutro con la dicitura “Disattiva notifiche”. Accanto al pulsante, un testo in grigio informa l’utente della funzione del pulsante, evitando ambiguità. L’utilizzo di colori non accesi e di un design discreto segue il principio della \textbf{gerarchia visiva} \cite{pieters2004}, scoraggiando azioni impulsive che potrebbero portare alla perdita involontaria di notifiche future.

\vspace{0.5cm}
\subsubsection{Modifica dello Stato e Feedback Visivo}
Quando l’utente clicca sul pulsante, il testo del pulsante cambia colore, diventando rosso, e il messaggio a fianco si aggiorna per sottolineare che la categoria di notifiche è stata disattivata. Questo utilizza il principio della \textbf{salienza visiva} \cite{nielsen1995}, enfatizzando il cambiamento e rendendo immediatamente chiara la conseguenza dell’azione.

\vspace{0.5cm}
\subsubsection{Incentivo alla Riattivazione}
Una volta disattivata una categoria tramite questa modalità, viene visualizzato un secondo pulsante con una call-to-action mirata per incentivare la riattivazione delle notifiche. Il design e il posizionamento del pulsante sfruttano il \textbf{principio dell’affordance} \cite{norman1988}, rendendo chiaro che l’utente ha la possibilità di tornare indietro sulla sua decisione in modo semplice e immediato.
\newline
Questa estensione si integra perfettamente con il modello generale di gestione delle notifiche, garantendo un’interazione fluida e coerente con le esigenze dell’utente. Nel caso in cui l’utente scelga di riattivare la categoria delle notifiche direttamente da una notifica, si passa all’\textbf{Estensione F}, che approfondisce questa modalità di gestione a partire dalle notifiche disattivate.
\begin{figure}[ht]
    \centering
    \begin{tikzpicture}[node distance=1.5cm and 1cm, auto]
        % Nodo per immagine 1 con didascalia sotto
        \node (img1) {
            \begin{tabular}{c}
                \includegraphics[width=0.4\textwidth]{Immagini/Mockup/notifiche/estensione A/clickNotifica.png} \\
                Cockburn: Extension A.2/A.3
            \end{tabular}
        };
        
        % Nodo per immagine 2 con didascalia sotto, posizionato a destra di img1
        \node (img2) [below=of img1] {
            \begin{tabular}{c}
                \includegraphics[width=0.4\textwidth]{Immagini/Mockup/notifiche/estensione A/clickDisattiva.png} \\
                Cockburn: Extension A.4
            \end{tabular}
        };
        
        % Nodo per immagine 3 con didascalia sotto, posizionato sotto img2
        \node (img3) [below=of img2] {
            \begin{tabular}{c}
                \includegraphics[width=0.4\textwidth]{Immagini/Mockup/notifiche/estensione A/disattivato.png} \\
                Cockburn: extension A.5
            \end{tabular}
        };
        
        % Disegna le frecce
        \draw[->, thick] (img1) -- (img2);
        \draw[->, thick] (img2) -- (img3);
      
    \end{tikzpicture}
    \caption{Mockup: estensione A della tabella di Cockburn del caso d'uso disattiva/attiva categoria notifica}
    \label{fig:tikz_flow}
\end{figure}

\newpage



\clearpage
\newpage
\subsubsection{Estensione B: Ripristino di un Annuncio Precedente}
Se l’utente sceglie di \textbf{ripristinare l’annuncio precedente}, il sistema avvia un processo di caricamento per fornire un feedback visivo sulla ripresa dei dati. Sebbene i dati siano salvati localmente e il recupero sia immediato, un \textbf{indicatore di caricamento fittizio} viene mostrato per alcuni secondi prima di caricare la schermata.

Questa soluzione è basata sul principio della \textbf{coerenza con le aspettative dell’utente} \cite{shneiderman2004}. In un contesto digitale, un ripristino istantaneo potrebbe apparire innaturale e creare confusione. L’indicatore di caricamento:
\begin{itemize}
    \item Rafforza la percezione di un processo in corso, migliorando la trasparenza dell’operazione.
    \item Evita che l’utente si domandi se il recupero sia realmente avvenuto o se ci siano stati problemi tecnici.
    \item Contribuisce a una transizione più fluida tra stati dell’interfaccia.
\end{itemize}

Una volta completato il caricamento, il sistema presenta l’interfaccia con i dati precedentemente salvati, consentendo all’utente di riprendere il processo da dove era stato interrotto.


\begin{figure}[ht]
    \centering
    \begin{tikzpicture}[node distance=1.5cm and 1cm, auto]
        % Nodo per immagine 1 con didascalia sotto
        \node (img1) {
            \begin{tabular}{c}
                \includegraphics[width=0.7\textwidth]{Immagini/Mockup/aggiungi annuncio/estensione B/step1.png} \\
                click nuovo annuncio
            \end{tabular}
        };
        
        % Nodo per immagine 2 con didascalia sotto, posizionato a destra di img1
        \node (img2) [below=of img1] {
            \begin{tabular}{c}
                \includegraphics[width=0.7\textwidth]{Immagini/Mockup/aggiungi annuncio/estensione B/step2.png} \\
                Cockburn: extension B.2/B.3/B.4
            \end{tabular}
        };
        
        % Nodo per immagine 3 con didascalia sotto, posizionato sotto img2
        \node (img3) [below=of img2] {
            \begin{tabular}{c}
                \includegraphics[width=0.7\textwidth]{Immagini/Mockup/aggiungi annuncio/estensione B/step3.png} \\
                Cockburn: extension B.5
            \end{tabular}
        };
        
        % Disegna le frecce
        \draw[->, thick] (img1) -- (img2);
        \draw[->, thick] (img2) -- (img3);
      
    \end{tikzpicture}
    \caption{Mockup: estensione B della tabella di Cockburn del caso d'uso nuovo annuncio}
    \label{fig:mockup_estensione_B_aggiungi_annuncio}
\end{figure}

\newpage



\clearpage
\newpage

\subsubsection{Estensione C: Modifica Rapida dello Stato delle Notifiche dalla Barra Laterale}

Per migliorare l’accessibilità e rendere la gestione delle notifiche più immediata, il sistema offre un metodo alternativo per attivare e disattivare le notifiche direttamente dalla barra laterale. Questa estensione elimina la necessità di accedere alla schermata dedicata, offrendo un controllo contestuale più rapido ed efficace.

\vspace{0.5cm}
\subsubsection{Interfaccia e Interazione}
Ogni categoria di notifica presente nella barra laterale include un pulsante contestuale che consente di modificarne lo stato. Il pulsante ha due possibili stati:

\begin{itemize}
    \item \textbf{Attiva}, se la categoria è attualmente disabilitata.
    \item \textbf{Disattiva}, se la categoria è attualmente abilitata.
\end{itemize}

Quando l’utente interagisce con il pulsante:
\begin{itemize}
    \item Se clicca su \textbf{Disattiva}, appare un popup di conferma che informa sulle implicazioni della disattivazione, in linea con i principi di prevenzione degli errori di Nielsen \cite{nielsen1995}.
    \item Se l’utente conferma, il sistema esegue una transizione animata per spostare la categoria dalla sezione delle notifiche attive a quella delle notifiche disattivate.
    \item Il pulsante cambia stato, diventando \textbf{Attiva}, in modo da riflettere visivamente la modifica e garantire un feedback immediato.
\end{itemize}

\subsubsection{Feedback Visivo e UX Design}
Per rendere il cambiamento chiaro e intuitivo, il sistema implementa le seguenti tecniche di UX design:
\begin{itemize}
    \item \textbf{Popup di conferma}: viene visualizzato prima di procedere con la modifica, seguendo le euristiche di usabilità per la prevenzione degli errori \cite{nielsen1995}.
    \item \textbf{Animazione di transizione}: la categoria viene spostata visivamente tra le sezioni della barra laterale, applicando il principio della \textbf{gestalt della continuità} \cite{miller1956} per rendere il cambiamento più naturale.
    \item \textbf{Aggiornamento dello stato del pulsante}: il pulsante cambia dinamicamente per riflettere lo stato attuale della categoria, riducendo l’ambiguità e migliorando la prevedibilità dell’interazione.
\end{itemize}

Questa estensione garantisce un’esperienza utente più fluida e immediata, riducendo il numero di passaggi necessari per la gestione delle notifiche senza compromettere la chiarezza e il controllo dell’utente.


\begin{figure}[ht]
    \centering
    \begin{tikzpicture}[node distance=1.5cm and 1cm, auto]
        % Nodo per immagine 1 con didascalia sotto
        \node (img1) {
            \begin{tabular}{c}
                \includegraphics[width=0.4\textwidth]{Immagini/Mockup/notifiche/ESTENSIONE C/clickDisattiva.png} \\
                Cockburn: Extension C.2
            \end{tabular}
        };
        
        % Nodo per immagine 2 con didascalia sotto, posizionato a destra di img1
        \node (img2) [below=of img1] {
            \begin{tabular}{c}
                \includegraphics[width=.6\textwidth]{Immagini/Mockup/notifiche/ESTENSIONE C/clickCoonferma.png} \\
                Cockburn: step 6/7
            \end{tabular}
        };
        
        % Nodo per immagine 3 con didascalia sotto, posizionato sotto img2
        \node (img3) [below=of img2] {
            \begin{tabular}{c}
                \includegraphics[width=0.3\textwidth]{Immagini/Mockup/notifiche/ESTENSIONE C/disattivato.png} \\
                Cockburn: step 8/9/10
            \end{tabular}
        };
        
        % Disegna le frecce
        \draw[->, thick] (img1) -- (img2);
        \draw[->, thick] (img2) -- (img3);
      
    \end{tikzpicture}
    \caption{Mockup: estensione C della tabella di Cockburn del caso d'uso disattiva/attiva categoria notifica}
    \label{fig:mockup_estensione_C_disattiva_notifiche}
\end{figure}

\newpage



\clearpage
\newpage

\subsubsection{Estensione D: Modifica Rapida dello Stato delle Notifiche con Attivazione Immediata}

Questa variante rappresenta l’approccio duale dell’\textbf{Estensione C}, semplificando ulteriormente l’attivazione delle notifiche. L’obiettivo è ridurre i passaggi necessari per riattivare una categoria disattivata, mantenendo comunque un controllo chiaro sulla disattivazione.

\vspace{0.5cm}
\subsubsection{Interfaccia e Interazione}
Analogamente all’\textbf{Estensione C}, ogni categoria nella barra laterale dispone di un pulsante contestuale per modificarne lo stato. Il pulsante può assumere due stati:

\begin{itemize}
    \item \textbf{Attiva}, se la categoria è attualmente disabilitata.
    \item \textbf{Disattiva}, se la categoria è attualmente abilitata.
\end{itemize}

Le interazioni dell’utente variano a seconda dell’azione eseguita:

\begin{itemize}
    \item \textbf{Disattivazione di una categoria}:
    \begin{itemize}
        \item Al clic su \textbf{Disattiva}, appare un popup di conferma che informa l’utente sulle conseguenze della scelta, prevenendo azioni accidentali in linea con i principi di Nielsen \cite{nielsen1995}.
        \item Se confermata, la categoria viene spostata nella sezione delle notifiche disattivate tramite un’animazione di transizione.
        \item Il pulsante cambia stato, diventando \textbf{Attiva}, fornendo un feedback visivo chiaro sulla modifica.
    \end{itemize}
    
    \item \textbf{Attivazione di una categoria}:
    \begin{itemize}
        \item Al clic su \textbf{Attiva}, il sistema aggiorna immediatamente lo stato della notifica senza richiedere una conferma esplicita.
        \item La categoria viene spostata nella sezione delle notifiche attive con un’animazione fluida, applicando il principio della \textbf{gestalt della continuità} \cite{miller1956}.
        \item Il pulsante cambia stato in \textbf{Disattiva}, rendendo la modifica evidente e intuitiva.
    \end{itemize}
\end{itemize}

\subsubsection{Feedback Visivo e UX Design}
L’esperienza utente è ottimizzata tramite tecniche di design che garantiscono chiarezza e immediatezza:

\begin{itemize}
    \item \textbf{Popup di conferma per la disattivazione}: aiuta a prevenire errori e rende consapevole l’utente delle conseguenze della scelta \cite{nielsen1995}.
    \item \textbf{Animazione di transizione}: assicura una continuità visiva fluida nello spostamento delle categorie, migliorando la percezione del cambiamento \cite{miller1956}.
    \item \textbf{Aggiornamento immediato dello stato del pulsante}: il cambio di testo e colore riflette lo stato corrente della categoria, riducendo l’ambiguità e migliorando la prevedibilità dell’interazione.
\end{itemize}

Questa estensione semplifica l’attivazione delle notifiche, eliminando il passaggio della conferma e migliorando la fluidità dell’interazione, senza compromettere il controllo dell’utente sulla gestione delle proprie preferenze.

\begin{figure}[ht]
    \centering
    \begin{tikzpicture}[node distance=1.5cm and 1cm, auto]
        % Nodo per immagine 1 con didascalia sotto
        \node (img1) {
            \begin{tabular}{c}
                \includegraphics[width=0.6\textwidth]{Immagini/Mockup/notifiche/estensione D/clickAttiva.png} \\
                Cockburn: Extension D.2
            \end{tabular}
        };
        
        % Nodo per immagine 2 con didascalia sotto, posizionato a destra di img1
        \node (img2) [below=of img1] {
            \begin{tabular}{c}
                \includegraphics[width=0.6\textwidth]{Immagini/Mockup/notifiche/estensione D/attivato.png} \\
                Cockburn: step 8/9/10
            \end{tabular}
        };
        
        % Disegna le frecce
        \draw[->, thick] (img1) -- (img2);
      
    \end{tikzpicture}
    \caption{Mockup: estensione D della tabella di Cockburn del caso d'uso disattiva/attiva categoria notifica}
    \label{fig:tikz_flow}
\end{figure}

\newpage



\clearpage
\newpage

\subsection{Caso d'Uso: Attivazione e Disattivazione Notifiche}

Il sistema prevede un'interfaccia dedicata alla gestione delle preferenze di notifica, pensata per permettere all'utente di curare la propria esperienza d'uso dell'applicazione, ricevendo news solo sugli argomenti di interesse.\\
In questa sezione andremo a esaminare le scelte effettuate durante la progettazione, l'impatto sull'esperienza utente e i prototipi generati alla fine dell'analisi. 

\vspace{0.5cm}
\subsubsection{Tipologie di Notifiche e Controllo Utente}
Le notifiche sono suddivise in diverse categorie per permettere una personalizzazione granulare:
\begin{itemize}
    \item \textbf{Annunci di nuovi Immobili}: notifiche basate sulle ricerche dell’utente.
    \item \textbf{Risposte alle offerte}: aggiornamenti sulle interazioni con gli annunci pubblicati.
    \item \textbf{Messaggi promozionali}: comunicazioni di marketing e offerte esclusive.
\end{itemize}
L’utente può, in qualsiasi momento, disattivare le notifiche per una o più categorie, mantenendo un controllo totale sulla propria esperienza \cite{shneiderman2004}.

\vspace{0.5cm}
\subsubsection{Gestione delle Notifiche e Conferma delle Modifiche}
La gestione delle notifiche avviene principalmente attraverso la schermata delle notifiche, composta da:
\begin{itemize}
    \item \textbf{Lista delle notifiche ricevute}, ognuna cliccabile per visualizzare i dettagli.
    \item \textbf{Barra laterale con le categorie di notifiche}, suddivise in:
    \begin{itemize}
        \item \textbf{Categorie attive}, con notifiche attualmente abilitate.
        \item \textbf{Categorie disattivate}, che non inviano più notifiche.
    \end{itemize}
\end{itemize}

In cima alla barra laterale è presente un’icona che, se cliccata, apre una schermata popup intitolata “Attiva e Disattiva Notifiche”. All’interno, ogni categoria è rappresentata da un toggle switch che indica lo stato attuale delle notifiche.

\vspace{0.5cm}
\subsubsection{Feedback Visivo e Animazioni Intuitive}
Per garantire un’interazione chiara e immediata, il sistema utilizza diverse tecniche di UX design:
\begin{itemize}
    \item \textbf{Animazione di transizione}: quando un toggle viene modificato, la categoria si sposta visivamente tra la sezione attiva e quella disattiva, sfruttando il principio di \textbf{gestalt della continuità} \cite{miller1956} per rendere il cambiamento intuitivo.
    \item \textbf{Feedback visivo immediato}: l’utente percepisce immediatamente l’effetto dell’azione senza necessità di un testo esplicativo eccessivo.
\end{itemize}

\newpage
\subsubsection{Conferma e Implicazioni della Disattivazione}
Per evitare errori accidentali e garantire consapevolezza delle conseguenze, la disattivazione di una categoria di notifiche è accompagnata da:
\begin{itemize}
    \item Un popup di conferma che informa l’utente che, durante il periodo in cui le notifiche sono disattivate, le notifiche non potranno essere recuperate \cite{wickens2008}.
    \item Un ulteriore messaggio di avviso prima della conferma definitiva, in linea con le \textbf{heuristiche di usabilità di Nielsen} \cite{nielsen1995} per la prevenzione degli errori.
\end{itemize}

Solo dopo la conferma finale, il sistema applica le modifiche alle preferenze dell'utente, garantendo un'interazione consapevole e trasparente.\\
Nel prototipo questi comportamenti sono stati modellati con un pulsante, tuttavia nell'applicazione finale è stato deciso di sostituirlo con un menù contestuale.



\begin{figure}[ht]
    \centering
    \begin{tikzpicture}[node distance=1.5cm and 1cm, auto]
        % Nodo per immagine 1 con didascalia sotto
        \node (img1) {
            \begin{tabular}{c}
                \includegraphics[width=0.7\textwidth]{Immagini/Mockup/notifiche/scenario principale/Pagina Lista Notifiche.png}\\
                Cockburn: step 1/2/3
            \end{tabular}
        };
        
        % Nodo per immagine 2 con didascalia sotto, posizionato a destra di img1
        \node (img2) [below=of img1] {
            \begin{tabular}{c}
                \includegraphics[width=0.7\textwidth]{Immagini/Mockup/notifiche/scenario principale/perDisattivareRisposte.png} \\
                Cockburn: step 4
            \end{tabular}
        };
        
        % Nodo per immagine 3 con didascalia sotto, posizionato sotto img2
        \node (img3) [below=of img2] {
            \begin{tabular}{c}
                \includegraphics[width=0.7\textwidth]{Immagini/Mockup/notifiche/scenario principale/perDisattivareNueve.png} \\
                Cockburn: step 4
            \end{tabular}
        };

        
        % Disegna le frecce
        \draw[->, thick] (img1) -- (img2);
        \draw[->, thick] (img2) -- (img3);
        
    \end{tikzpicture}
    \caption{Mockup: scenario principale della tabella di Cockburn del caso d'uso disattiva/attiva categoria notifica}
    \label{fig:tikz_flow}
\end{figure}

\clearpage
\newpage

\input{Requisiti Del Software/Analisi dei Requisiti/Mockup/disattivazione notifiche/Scenario principale 2}

\clearpage
\newpage

\subsubsection{Estensione A: Disattivazione Notifiche dalla Visualizzazione di una Notifica}

Per offrire un maggiore controllo sulla gestione delle notifiche senza interrompere l’esperienza utente, il sistema permette di disattivare una categoria direttamente dalla visualizzazione di una notifica specifica. Questa variante è progettata per garantire una modifica consapevole delle preferenze, evitando azioni impulsive che potrebbero compromettere la ricezione di informazioni rilevanti.

\vspace{0.5cm}
\subsubsection{Interfaccia e Comportamento del Bottone}
Alla fine del testo di ogni notifica, se la relativa categoria è attiva, è presente un pulsante neutro con la dicitura “Disattiva notifiche”. Accanto al pulsante, un testo in grigio informa l’utente della funzione del pulsante, evitando ambiguità. L’utilizzo di colori non accesi e di un design discreto segue il principio della \textbf{gerarchia visiva} \cite{pieters2004}, scoraggiando azioni impulsive che potrebbero portare alla perdita involontaria di notifiche future.

\vspace{0.5cm}
\subsubsection{Modifica dello Stato e Feedback Visivo}
Quando l’utente clicca sul pulsante, il testo del pulsante cambia colore, diventando rosso, e il messaggio a fianco si aggiorna per sottolineare che la categoria di notifiche è stata disattivata. Questo utilizza il principio della \textbf{salienza visiva} \cite{nielsen1995}, enfatizzando il cambiamento e rendendo immediatamente chiara la conseguenza dell’azione.

\vspace{0.5cm}
\subsubsection{Incentivo alla Riattivazione}
Una volta disattivata una categoria tramite questa modalità, viene visualizzato un secondo pulsante con una call-to-action mirata per incentivare la riattivazione delle notifiche. Il design e il posizionamento del pulsante sfruttano il \textbf{principio dell’affordance} \cite{norman1988}, rendendo chiaro che l’utente ha la possibilità di tornare indietro sulla sua decisione in modo semplice e immediato.
\newline
Questa estensione si integra perfettamente con il modello generale di gestione delle notifiche, garantendo un’interazione fluida e coerente con le esigenze dell’utente. Nel caso in cui l’utente scelga di riattivare la categoria delle notifiche direttamente da una notifica, si passa all’\textbf{Estensione F}, che approfondisce questa modalità di gestione a partire dalle notifiche disattivate.
\begin{figure}[ht]
    \centering
    \begin{tikzpicture}[node distance=1.5cm and 1cm, auto]
        % Nodo per immagine 1 con didascalia sotto
        \node (img1) {
            \begin{tabular}{c}
                \includegraphics[width=0.4\textwidth]{Immagini/Mockup/notifiche/estensione A/clickNotifica.png} \\
                Cockburn: Extension A.2/A.3
            \end{tabular}
        };
        
        % Nodo per immagine 2 con didascalia sotto, posizionato a destra di img1
        \node (img2) [below=of img1] {
            \begin{tabular}{c}
                \includegraphics[width=0.4\textwidth]{Immagini/Mockup/notifiche/estensione A/clickDisattiva.png} \\
                Cockburn: Extension A.4
            \end{tabular}
        };
        
        % Nodo per immagine 3 con didascalia sotto, posizionato sotto img2
        \node (img3) [below=of img2] {
            \begin{tabular}{c}
                \includegraphics[width=0.4\textwidth]{Immagini/Mockup/notifiche/estensione A/disattivato.png} \\
                Cockburn: extension A.5
            \end{tabular}
        };
        
        % Disegna le frecce
        \draw[->, thick] (img1) -- (img2);
        \draw[->, thick] (img2) -- (img3);
      
    \end{tikzpicture}
    \caption{Mockup: estensione A della tabella di Cockburn del caso d'uso disattiva/attiva categoria notifica}
    \label{fig:tikz_flow}
\end{figure}

\newpage


\clearpage
\newpage
\subsubsection{Estensione B: Ripristino di un Annuncio Precedente}
Se l’utente sceglie di \textbf{ripristinare l’annuncio precedente}, il sistema avvia un processo di caricamento per fornire un feedback visivo sulla ripresa dei dati. Sebbene i dati siano salvati localmente e il recupero sia immediato, un \textbf{indicatore di caricamento fittizio} viene mostrato per alcuni secondi prima di caricare la schermata.

Questa soluzione è basata sul principio della \textbf{coerenza con le aspettative dell’utente} \cite{shneiderman2004}. In un contesto digitale, un ripristino istantaneo potrebbe apparire innaturale e creare confusione. L’indicatore di caricamento:
\begin{itemize}
    \item Rafforza la percezione di un processo in corso, migliorando la trasparenza dell’operazione.
    \item Evita che l’utente si domandi se il recupero sia realmente avvenuto o se ci siano stati problemi tecnici.
    \item Contribuisce a una transizione più fluida tra stati dell’interfaccia.
\end{itemize}

Una volta completato il caricamento, il sistema presenta l’interfaccia con i dati precedentemente salvati, consentendo all’utente di riprendere il processo da dove era stato interrotto.


\begin{figure}[ht]
    \centering
    \begin{tikzpicture}[node distance=1.5cm and 1cm, auto]
        % Nodo per immagine 1 con didascalia sotto
        \node (img1) {
            \begin{tabular}{c}
                \includegraphics[width=0.7\textwidth]{Immagini/Mockup/aggiungi annuncio/estensione B/step1.png} \\
                click nuovo annuncio
            \end{tabular}
        };
        
        % Nodo per immagine 2 con didascalia sotto, posizionato a destra di img1
        \node (img2) [below=of img1] {
            \begin{tabular}{c}
                \includegraphics[width=0.7\textwidth]{Immagini/Mockup/aggiungi annuncio/estensione B/step2.png} \\
                Cockburn: extension B.2/B.3/B.4
            \end{tabular}
        };
        
        % Nodo per immagine 3 con didascalia sotto, posizionato sotto img2
        \node (img3) [below=of img2] {
            \begin{tabular}{c}
                \includegraphics[width=0.7\textwidth]{Immagini/Mockup/aggiungi annuncio/estensione B/step3.png} \\
                Cockburn: extension B.5
            \end{tabular}
        };
        
        % Disegna le frecce
        \draw[->, thick] (img1) -- (img2);
        \draw[->, thick] (img2) -- (img3);
      
    \end{tikzpicture}
    \caption{Mockup: estensione B della tabella di Cockburn del caso d'uso nuovo annuncio}
    \label{fig:mockup_estensione_B_aggiungi_annuncio}
\end{figure}

\newpage



\input{Requisiti Del Software/Analisi dei Requisiti/Mockup/disattivazione notifiche/estensione c}

\clearpage
\newpage

\subsubsection{Estensione D: Modifica Rapida dello Stato delle Notifiche con Attivazione Immediata}

Questa variante rappresenta l’approccio duale dell’\textbf{Estensione C}, semplificando ulteriormente l’attivazione delle notifiche. L’obiettivo è ridurre i passaggi necessari per riattivare una categoria disattivata, mantenendo comunque un controllo chiaro sulla disattivazione.

\vspace{0.5cm}
\subsubsection{Interfaccia e Interazione}
Analogamente all’\textbf{Estensione C}, ogni categoria nella barra laterale dispone di un pulsante contestuale per modificarne lo stato. Il pulsante può assumere due stati:

\begin{itemize}
    \item \textbf{Attiva}, se la categoria è attualmente disabilitata.
    \item \textbf{Disattiva}, se la categoria è attualmente abilitata.
\end{itemize}

Le interazioni dell’utente variano a seconda dell’azione eseguita:

\begin{itemize}
    \item \textbf{Disattivazione di una categoria}:
    \begin{itemize}
        \item Al clic su \textbf{Disattiva}, appare un popup di conferma che informa l’utente sulle conseguenze della scelta, prevenendo azioni accidentali in linea con i principi di Nielsen \cite{nielsen1995}.
        \item Se confermata, la categoria viene spostata nella sezione delle notifiche disattivate tramite un’animazione di transizione.
        \item Il pulsante cambia stato, diventando \textbf{Attiva}, fornendo un feedback visivo chiaro sulla modifica.
    \end{itemize}
    
    \item \textbf{Attivazione di una categoria}:
    \begin{itemize}
        \item Al clic su \textbf{Attiva}, il sistema aggiorna immediatamente lo stato della notifica senza richiedere una conferma esplicita.
        \item La categoria viene spostata nella sezione delle notifiche attive con un’animazione fluida, applicando il principio della \textbf{gestalt della continuità} \cite{miller1956}.
        \item Il pulsante cambia stato in \textbf{Disattiva}, rendendo la modifica evidente e intuitiva.
    \end{itemize}
\end{itemize}

\subsubsection{Feedback Visivo e UX Design}
L’esperienza utente è ottimizzata tramite tecniche di design che garantiscono chiarezza e immediatezza:

\begin{itemize}
    \item \textbf{Popup di conferma per la disattivazione}: aiuta a prevenire errori e rende consapevole l’utente delle conseguenze della scelta \cite{nielsen1995}.
    \item \textbf{Animazione di transizione}: assicura una continuità visiva fluida nello spostamento delle categorie, migliorando la percezione del cambiamento \cite{miller1956}.
    \item \textbf{Aggiornamento immediato dello stato del pulsante}: il cambio di testo e colore riflette lo stato corrente della categoria, riducendo l’ambiguità e migliorando la prevedibilità dell’interazione.
\end{itemize}

Questa estensione semplifica l’attivazione delle notifiche, eliminando il passaggio della conferma e migliorando la fluidità dell’interazione, senza compromettere il controllo dell’utente sulla gestione delle proprie preferenze.

\begin{figure}[ht]
    \centering
    \begin{tikzpicture}[node distance=1.5cm and 1cm, auto]
        % Nodo per immagine 1 con didascalia sotto
        \node (img1) {
            \begin{tabular}{c}
                \includegraphics[width=0.6\textwidth]{Immagini/Mockup/notifiche/estensione D/clickAttiva.png} \\
                Cockburn: Extension D.2
            \end{tabular}
        };
        
        % Nodo per immagine 2 con didascalia sotto, posizionato a destra di img1
        \node (img2) [below=of img1] {
            \begin{tabular}{c}
                \includegraphics[width=0.6\textwidth]{Immagini/Mockup/notifiche/estensione D/attivato.png} \\
                Cockburn: step 8/9/10
            \end{tabular}
        };
        
        % Disegna le frecce
        \draw[->, thick] (img1) -- (img2);
      
    \end{tikzpicture}
    \caption{Mockup: estensione D della tabella di Cockburn del caso d'uso disattiva/attiva categoria notifica}
    \label{fig:tikz_flow}
\end{figure}

\newpage



\clearpage
\newpage

\subsubsection{Estensione E: Errore Durante la Modifica dello Stato delle Categorie di Notifica}

Nel caso in cui si verifichi un errore durante il tentativo di modifica dello stato di una categoria di notifica, viene visualizzato un messaggio di errore sotto forma di un popup informativo. Il messaggio informa l'utente che la modifica non è riuscita e lo invita a riprovare più tardi, senza salvare le modifiche effettuate.

\vspace{0.5cm}
\subsubsection{Gestione dell'Errore e Feedback Utente} Il popup di errore presenta i seguenti elementi chiave: \begin{itemize} \item \textbf{Messaggio chiaro e informativo}: il messaggio comunica all'utente che l'operazione non è andata a buon fine, senza entrare in dettagli tecnici, per ridurre il rischio di frustrazione. L'utente viene anche informato che l'operazione non è stata completata e che le modifiche non sono state salvate. \item \textbf{Pulsante “Ok”}: consente all'utente di chiudere il popup e tornare all'interfaccia principale. Il pulsante di conferma è chiaro e consente di riprendere l'interazione senza indugi. \end{itemize}

\subsubsection{Principi di Design Applicati} L'approccio di gestione dell'errore in questa estensione si fonda sui seguenti principi di UX e usabilità: \begin{itemize} \item \textbf{Visibilità dello stato del sistema} \cite{nielsen1995}: l'errore è comunicato all'utente attraverso un popup che evidenzia chiaramente che il sistema non è riuscito a completare l'operazione. \item \textbf{Prevenzione degli errori} \cite{nielsen1995}: sebbene l'errore non possa essere evitato completamente, il sistema offre un feedback immediato e comprensibile, impedendo all'utente di rimanere confuso o incerto sullo stato dell'operazione. \item \textbf{Semplicità e chiarezza} \cite{nielsen1995}: il messaggio di errore è semplice e diretto, senza sovraccaricare l'utente con informazioni tecniche. L'invito a riprovare più tardi mantiene il flusso di lavoro semplice e lineare. \item \textbf{Controllo dell'utente} \cite{norman1988}: l'utente ha il pieno controllo sulla gestione dell'errore, poiché il popup permette di chiudere facilmente l'interfaccia e riprendere l'attività, mantenendo un'esperienza utente fluida. \end{itemize}

Questa soluzione di gestione dell'errore è progettata per garantire un'esperienza utente chiara e senza frustrazioni, minimizzando il disagio derivante da errori tecnici imprevisti e offrendo un percorso semplice per riprendere l'interazione.
\begin{figure}[ht]
    \centering
    \begin{tikzpicture}[node distance=1.5cm and 1cm, auto]
        % Nodo per immagine 1 con didascalia sotto
        \node (img1) {
            \begin{tabular}{c}
                \includegraphics[width=0.6\textwidth]{Immagini/Mockup/notifiche/estensione E/click conferma.png} \\
                Cockburn: step 7
            \end{tabular}
        };
        
        % Nodo per immagine 2 con didascalia sotto, posizionato a destra di img1
        \node (img2) [below=of img1] {
            \begin{tabular}{c}
                \includegraphics[width=0.6\textwidth]{Immagini/Mockup/notifiche/estensione E/errore.png} \\
                Cockburn: Extension E.7/E.8
            \end{tabular}
        };
        
        % Disegna le frecce
        \draw[->, thick] (img1) -- (img2);
      
    \end{tikzpicture}
    \caption{Mockup: estensione E della tabella di Cockburn del caso d'uso disattiva/attiva categoria notifica}
    \label{fig:tikz_flow}
\end{figure}

\newpage



\clearpage
\newpage

\subsubsection{Estensione F: Riattivazione Notifiche dalle Notifiche Disattivate}

Questa estensione consente all’utente di riattivare una categoria di notifiche direttamente da una notifica precedentemente ricevuta e appartenente a una categoria disattivata. L’obiettivo è fornire un meccanismo immediato per ripristinare le notifiche quando l’utente si rende conto della loro utilità.

\vspace{0.5cm}
\subsubsection{Indicazione dello Stato e Call-to-Action}
Quando una notifica proviene da una categoria disattivata, il sistema mostra un messaggio di avviso evidenziato che informa l’utente che non riceverà più aggiornamenti simili. Il pulsante associato cambia stato e diventa un invito all’azione con il testo “Riattiva notifiche”. Questo sfrutta il \textbf{principio della reversibilità} \cite{shneiderman2004}, permettendo all’utente di annullare la decisione precedente senza difficoltà.

\vspace{0.5cm}
\subsubsection{Feedback Visivo e Conferma della Riattivazione}
Alla pressione del pulsante, il testo cambia colore in verde e il messaggio informativo si aggiorna, confermando che le notifiche per quella categoria sono state riattivate. Il sistema può fornire un’ulteriore conferma con un breve messaggio di notifica o una vibrazione del dispositivo per enfatizzare l’azione completata.

\vspace{0.5cm}
\subsubsection{Coerenza con il Modello di Gestione Notifiche}
Questa estensione rafforza la coerenza dell’interfaccia di gestione delle notifiche, mantenendo le scelte dell’utente sempre modificabili e promuovendo un’interazione trasparente e prevedibile. L’utente può così gestire le notifiche senza dover accedere necessariamente alla schermata delle impostazioni, riducendo il carico cognitivo e migliorando l’usabilità complessiva del sistema.\begin{figure}[H]
    \centering
    \begin{tikzpicture}[node distance=1.5cm and 1cm, auto]
        % Nodo per immagine 1 con didascalia sotto
        \node (img1) {
            \begin{tabular}{c}
                \includegraphics[width=0.4\textwidth]{Immagini/Mockup/notifiche/estensione F/clickNotifica.png} \\
                Cockburn: Extension F.2
            \end{tabular}
        };
        
        % Nodo per immagine 2 con didascalia sotto, posizionato a destra di img1
        \node (img2) [below=of img1] {
            \begin{tabular}{c}
                \includegraphics[width=0.4\textwidth]{Immagini/Mockup/notifiche/estensione F/clickAttiva.png} \\
                Cockburn: Extension F.3
            \end{tabular}
        };
        
        % Nodo per immagine 3 con didascalia sotto, posizionato sotto img2
        \node (img3) [below=of img2] {
            \begin{tabular}{c}
                \includegraphics[width=0.4\textwidth]{Immagini/Mockup/notifiche/estensione F/attivato.png} \\
                Cockburn: Extension F.4/F.5
            \end{tabular}
        };
        
        % Disegna le frecce
        \draw[->, thick] (img1) -- (img2);
        \draw[->, thick] (img2) -- (img3);
      
    \end{tikzpicture}
    \caption{Mockup: estensione F della tabella di Cockburn del caso d'uso disattiva/attiva categoria notifica}
    \label{fig:tikz_flow}
\end{figure}

\newpage



\clearpage
\newpage

\section{Test di usabilità}

I \textbf{test di usabilità} rappresentano una fase essenziale nel processo di sviluppo di un’interfaccia utente, consentendo di valutare l'efficacia, l’efficienza e la soddisfazione dell’utente nell’interazione con il sistema. In particolare, l'uso di mockup interattivi offre la possibilità di raccogliere feedback sulle scelte di design prima ancora della fase di sviluppo, riducendo i costi di eventuali revisioni e migliorando la qualità dell’esperienza utente.
\newline
L'obiettivo principale di questi test è identificare eventuali problemi di navigazione, ambiguità nelle interazioni o difficoltà nella comprensione delle funzionalità, al fine di ottimizzare l'interfaccia prima del rilascio definitivo. 

\vspace{0.5cm} % Aggiunge spazio prima della sezione

\textbf{1. Tempo medio per completare un Task}
\begin{equation}
T_{\text{medio}} = \frac{\sum_{i=1}^{n} T_i}{n}
\end{equation}
dove \( T_i \) è il tempo impiegato dall'utente \( i \) per completare il task e \( n \) è il numero totale di utenti.

\vspace{0.5cm} % Aggiunge spazio prima della sezione
\textbf{2. Tasso di completamento di un Task}
\begin{equation}
T_{\text{completamento}} = \frac{U_{\text{completati}}}{U_{\text{totali}}} \times 100
\end{equation}
dove \( U_{\text{completati}} \) è il numero di utenti che hanno completato il task e \( U_{\text{totali}} \) è il numero totale di utenti che hanno provato il task.

\vspace{0.5cm} % Aggiunge spazio prima della sezione
\textbf{3. Errore Medio per Task}
\begin{equation}
E_{\text{medio}} = \frac{\sum_{i=1}^{n} E_i}{n}
\end{equation}
dove \( E_i \) è il numero di errori commessi dall'utente \( i \) durante il task.

\vspace{0.5cm} % Aggiunge spazio prima della sezione
\textbf{4. Customer Satisfaction Score}
\begin{equation}
CSAT = \frac{\sum_{i=1}^{n} S_i}{n} \times 100
\end{equation}
dove \( S_i \) è il punteggio di soddisfazione dato dall'utente \( i \) e \( n \) è il numero totale di risposte raccolte.

\vspace{0.5cm} % Aggiunge spazio prima della sezione
\textbf{5. System Usability Scale (SUS)}
\newline
\newline
Il \textbf{System Usability Scale (SUS)} è un metodo standardizzato, introdotto da \textbf{John Brooke} nel 1986, utilizzato per misurare l’usabilità di un prodotto attraverso un questionario composto da \textbf{10 affermazioni}. Gli utenti rispondono utilizzando una \textbf{scala Likert a 5 punti}, esprimendo il loro grado di accordo o disaccordo. Il punteggio complessivo, che varia da 0 a 100, fornisce una misura quantitativa dell’usabilità percepita, consentendo di confrontare i risultati con benchmark consolidati.

\vspace{0.5cm} % Aggiunge spazio prima della sezione
\subsubsection{Questionario relativo al mockup "Creazione nuovo annuncio"}

Di seguito presentiamo il questionario utilizzato per valutare il mockup del processo di creazione di un annuncio immobiliare. Ogni domanda prevede cinque opzioni di risposta:

\begin{itemize}
    \item Per niente d’accordo
    \item Poco d’accordo
    \item Né d’accordo né in disaccordo
    \item Abbastanza d’accordo
    \item Molto d’accordo
\end{itemize}
Ai fini del calcolo del punteggio \textbf{SUS}, alle risposte viene assegnato un valore da \textbf{1} a \textbf{5}, dove la prima opzione corrisponde a 1 punto e l’ultima a 5 punti.

\vspace{0.5cm} % Aggiunge spazio prima della sezione

\begin{enumerate}
    \item \textbf{La compilazione dei campi divisi in step ha reso il processo di creazione più intuitivo e meno stancante.}
    \newline
    \texttt{Scopo}: Verificare se la suddivisione del form in più passaggi migliora l’esperienza utente, evitando un sovraccarico cognitivo e rendendo la compilazione più fluida.

    \item \textbf{Il bottone “Avanti” è posizionato in modo poco visibile e difficile da individuare.}
    \newline
    \texttt{Scopo}: Valutare la visibilità e l’intuitività del pulsante che consente di procedere nella compilazione, elemento essenziale per garantire una navigazione chiara e senza interruzioni.

    \item \textbf{I campi a scelta (non liberi) mi hanno aiutato a capire meglio il tipo di informazioni richieste e a velocizzare la compilazione.}
    \newline
    \texttt{Scopo}: Analizzare se l’uso di menu a tendina o opzioni predefinite aiuta a ridurre incertezze, errori e tempi di compilazione rispetto ai campi di testo libero.

    
    \item \textbf{Il numero di campi richiesti è eccessivo o insufficiente, rendendo il form poco bilanciato.}
    \newline
    \texttt{Scopo}: Ottenere un feedback sulla quantità di informazioni richieste, bilanciando completezza e semplicità d’uso, evitando di rendere il processo troppo lungo o complesso.

    \item \textbf{I colori utilizzati sono gradevoli e non affaticano la lettura.}
    \newline
    \texttt{Scopo}: Valutare se la combinazione di colori scelta favorisce una buona leggibilità e un’esperienza visiva piacevole, senza creare affaticamento visivo.

    \item \textbf{L’importazione e la gestione delle foto dell’annuncio risultano poco intuitive.}
    \newline
    \texttt{Scopo}: esaminare la semplicità e l’efficacia del meccanismo di caricamento delle immagini, una funzionalità chiave nella creazione di annunci immobiliari.

    \item \textbf{Trovo utile l’anteprima dell’annuncio prima di confermare la pubblicazione.}
    \newline
    \texttt{Scopo}: Capire se l’anteprima fornisce valore aggiunto agli utenti, permettendo loro di controllare e correggere eventuali errori prima della pubblicazione definitiva.

    \item \textbf{L’avviso in caso di un annuncio in sospeso è poco chiaro o inutile.}
    \newline
    \texttt{Scopo}: Testare la comprensibilità e l’efficacia del messaggio di avviso per evitare che l’utente perda dati o crei annunci duplicati.

    \item \textbf{In caso di errore (compilazione errata o parziale), i messaggi di errore sono posizionati bene e mi aiutano a capire rapidamente dove ho sbagliato.}
    \newline
    \texttt{Scopo}: Valutare la chiarezza dei messaggi di errore e il loro posizionamento, per garantire che l’utente possa correggere facilmente eventuali problemi.

    \item \textbf{Nel complesso, non sono soddisfatto dell’esperienza di creazione di un annuncio.}
    \newline
    \texttt{Scopo}: Raccogliere un’indicazione generale sul livello di soddisfazione dell’utente rispetto all’intero processo, fornendo una misura qualitativa dell’usabilità percepita.
    
\end{enumerate}

\vspace{0.5cm} % Aggiunge spazio prima della sezione
\subsubsection{Risultato SUS ottenuto attraverso i feedback raccolti}
Il System Usability Scale (SUS) viene calcolato seguendo questi passaggi:

\begin{enumerate}

    \item \textbf{Assegnazione punteggio}
    \newline 
    Ogni domanda viene valutata su una scala da 1 a 5:

    \begin{itemize}
        \item 1 = Per niente d’accordo
        \item 2 = Poco d’accordo
        \item 3 = Né d’accordo né in disaccordo
        \item 4 = Abbastanza d’accordo
        \item 5 = Molto d’accordo
    \end{itemize}
    Le 10 domande del questionario SUS sono di due tipi:
    \begin{itemize}
        \item \textbf{Domande dispari (1, 3, 5, 7, 9):} indicano usabilità positiva
        
        \item \textbf{Domande pari (2, 4, 6, 8, 10):} indicano usabilità negativa
    \end{itemize}

    \item \textbf{Calcolo del punteggio per ogni domanda}
    
    \begin{itemize}
        \item \textbf{Per le domande dispari:} Punteggio=(Risposta-1)
        
        \item \textbf{Per le domande pari:} Punteggio=(5-Risposta)
    \end{itemize}

    \item \textbf{Sommiamo tutti i punteggi ottenuti}
    \newline
    Otteniamo un punteggio complessivo che va da 0 a 40.

    \item \textbf{Moltiplichiamo per 2,5}: SUS=(somma dei punteggi)×2.5.
    \newline
    Il punteggio finale sarà compreso tra 0 e 100, ma non rappresenta una percentuale.
    
\end{enumerate}

\textbf{Interpretazione del punteggio SUS}

\begin{itemize}
    \item \textbf{Sopra 80} → Ottima usabilità
    \item \textbf{Tra 70 e 80} → Buona usabilità
    \item \textbf{Tra 50 e 70} → Accettabile, ma migliorabile
    \item \textbf{Sotto 50} → Problemi di usabilità significativi
\end{itemize}
Il questionario è stato compilato da quattro agenti immobiliari, i principali attori di questo caso d’uso. Ciascun agente proviene da un’agenzia immobiliare diversa e appartiene a una fascia d’età differente, al fine di garantire un campione più eterogeneo e realistico. 

\vspace{0.5cm}

\textbf{Utente 1:}
\begin{itemize}
    \item \textbf{Risposte a domande dispari}
    \begin{itemize}
        \item Domanda 1: molto d'accordo = 5-1 = 4
        \item Domanda 3: molto d'accordo = 5-1 = 4
        \item Domanda 5: molto d'accordo = 5-1 = 4
        \item Domanda 7: molto d'accordo = 5-1 = 4
        \item Domanda 9: molto d'accordo = 5-1 = 4
    \end{itemize}
    \item \textbf{Risposte a domande pari}
    \begin{itemize}
        \item Domanda 2: per niente d'accordo = 5-1 = 4
        \item Domanda 4: poco d'accordo = 5-2 = 3
        \item Domanda 6: poco d'accordo = 5-2 = 3
        \item Domanda 8: per niente d'accordo = 5-1 = 4
        \item Domanda 10:per niente d'accordo = 5-1 = 4
    \end{itemize}

    \item \textbf{Totale punteggio utente 1: } 38*2.5 = \textbf{95}
    
\end{itemize}
\textbf{Utente 2:}
\begin{itemize}
    \item \textbf{Risposte a domande dispari}
    \begin{itemize}
        \item Domanda 1: molto d'accordo = 5-1 = 4
        \item Domanda 3: molto d'accordo = 5-1 = 4
        \item Domanda 5: molto d'accordo = 5-1 = 4
        \item Domanda 7: abbastanza d'accordo = 4-1 = 3
        \item Domanda 9: molto d'accordo = 5-1 = 4
    \end{itemize}
    \item \textbf{Risposte a domande pari}
    \begin{itemize}
        \item Domanda 2: per niente d'accordo = 5-1 = 4
        \item Domanda 4: per niente d'accordo = 5-1 = 4
        \item Domanda 6: poco d'accordo = 5-2 = 3
        \item Domanda 8:né d'accordo né in disaccordo = 5-3 = 2
        \item Domanda 10:per niente d'accordo = 5-1 = 4
    \end{itemize}

    \item \textbf{Totale punteggio utente 2: } 36*2.5 = \textbf{90}
    
\end{itemize}
\textbf{Utente 3:}
\begin{itemize}
    \item \textbf{Risposte a domande dispari}
    \begin{itemize}
        \item Domanda 1:abbastanza d'accordo = 4-1 = 3
        \item Domanda 3: molto d'accordo = 5-1 = 4
        \item Domanda 5: molto d'accordo = 5-1 = 4
        \item Domanda 7: molto d'accordo = 5-1 = 4
        \item Domanda 9: molto d'accordo = 5-1 = 4
    \end{itemize}
    \item \textbf{Risposte a domande pari}
    \begin{itemize}
        \item Domanda 2: poco d'accordo = 5-2 = 3
        \item Domanda 4: poco d'accordo= 5-2 = 3
        \item Domanda 6: per niente d'accordo = 5-1 = 4
        \item Domanda 8: poco d'accordo = 5-2 = 3
        \item Domanda 10: per niente d'accordo = 5-1 = 4
    \end{itemize}

    \item \textbf{Totale punteggio utente 3: } 36*2.5 = \textbf{90}
    
\end{itemize}
\textbf{Utente 4:}
\begin{itemize}
    \item \textbf{Risposte a domande dispari}
    \begin{itemize}
        \item Domanda 1: molto d'accordo = 5-1 = 4
        \item Domanda 3: molto d'accordo = 5-1 = 4
        \item Domanda 5: né d'accordo né in dissacordo = 3-1 = 2
        \item Domanda 7: molto d'accordo = 5-1 = 4
        \item Domanda 9: molto d'accordo = 5-1 = 4
    \end{itemize}
    \item \textbf{Risposte a domande pari}
    \begin{itemize}
        \item Domanda 2: poco d'accordo = 5-2 = 3
        \item Domanda 4: né 'accordo né in disaccordo = 5-3 = 2
        \item Domanda 6:per niente d'accordo = 5-1 = 4
        \item Domanda 8: per niente d'accordo = 5-1 = 4
        \item Domanda 10:per niente d'accordo = 5-1 = 4
    \end{itemize}

    \item \textbf{Totale punteggio utente 4: } 35*2.5= \textbf{87.5}
    
\end{itemize}
Media punteggio SUS = 95+90+90+87.5/4 = \textbf{90.62}. Il punteggio è nettamente superiore a 80 il che indica un \textbf{ottima usabilità}.


