\subsection{Caso d'Uso: Attivazione e Disattivazione Notifiche}

Il sistema prevede un'interfaccia dedicata alla gestione delle preferenze di notifica, pensata per permettere all'utente di curare la propria esperienza d'uso dell'applicazione, ricevendo news solo sugli argomenti di interesse.\\
In questa sezione andremo a esaminare le scelte effettuate durante la progettazione, l'impatto sull'esperienza utente e i prototipi generati alla fine dell'analisi. 

\vspace{0.5cm}
\subsubsection{Tipologie di Notifiche e Controllo Utente}
Le notifiche sono suddivise in diverse categorie per permettere una personalizzazione granulare:
\begin{itemize}
    \item \textbf{Annunci di nuovi Immobili}: notifiche basate sulle ricerche dell’utente.
    \item \textbf{Risposte alle offerte}: aggiornamenti sulle interazioni con gli annunci pubblicati.
    \item \textbf{Messaggi promozionali}: comunicazioni di marketing e offerte esclusive.
\end{itemize}
L’utente può, in qualsiasi momento, disattivare le notifiche per una o più categorie, mantenendo un controllo totale sulla propria esperienza \cite{shneiderman2004}.

\vspace{0.5cm}
\subsubsection{Gestione delle Notifiche e Conferma delle Modifiche}
La gestione delle notifiche avviene principalmente attraverso la schermata delle notifiche, composta da:
\begin{itemize}
    \item \textbf{Lista delle notifiche ricevute}, ognuna cliccabile per visualizzare i dettagli.
    \item \textbf{Barra laterale con le categorie di notifiche}, suddivise in:
    \begin{itemize}
        \item \textbf{Categorie attive}, con notifiche attualmente abilitate.
        \item \textbf{Categorie disattivate}, che non inviano più notifiche.
    \end{itemize}
\end{itemize}

In cima alla barra laterale è presente un’icona che, se cliccata, apre una schermata popup intitolata “Attiva e Disattiva Notifiche”. All’interno, ogni categoria è rappresentata da un toggle switch che indica lo stato attuale delle notifiche.

\vspace{0.5cm}
\subsubsection{Feedback Visivo e Animazioni Intuitive}
Per garantire un’interazione chiara e immediata, il sistema utilizza diverse tecniche di UX design:
\begin{itemize}
    \item \textbf{Animazione di transizione}: quando un toggle viene modificato, la categoria si sposta visivamente tra la sezione attiva e quella disattiva, sfruttando il principio di \textbf{gestalt della continuità} \cite{miller1956} per rendere il cambiamento intuitivo.
    \item \textbf{Feedback visivo immediato}: l’utente percepisce immediatamente l’effetto dell’azione senza necessità di un testo esplicativo eccessivo.
\end{itemize}

\newpage
\subsubsection{Conferma e Implicazioni della Disattivazione}
Per evitare errori accidentali e garantire consapevolezza delle conseguenze, la disattivazione di una categoria di notifiche è accompagnata da:
\begin{itemize}
    \item Un popup di conferma che informa l’utente che, durante il periodo in cui le notifiche sono disattivate, le notifiche non potranno essere recuperate \cite{wickens2008}.
    \item Un ulteriore messaggio di avviso prima della conferma definitiva, in linea con le \textbf{heuristiche di usabilità di Nielsen} \cite{nielsen1995} per la prevenzione degli errori.
\end{itemize}

Solo dopo la conferma finale, il sistema applica le modifiche alle preferenze dell'utente, garantendo un'interazione consapevole e trasparente.\\
Nel prototipo questi comportamenti sono stati modellati con un pulsante, tuttavia nell'applicazione finale è stato deciso di sostituirlo con un menù contestuale.



\begin{figure}[ht]
    \centering
    \begin{tikzpicture}[node distance=1.5cm and 1cm, auto]
        % Nodo per immagine 1 con didascalia sotto
        \node (img1) {
            \begin{tabular}{c}
                \includegraphics[width=0.7\textwidth]{Immagini/Mockup/notifiche/scenario principale/Pagina Lista Notifiche.png}\\
                Cockburn: step 1/2/3
            \end{tabular}
        };
        
        % Nodo per immagine 2 con didascalia sotto, posizionato a destra di img1
        \node (img2) [below=of img1] {
            \begin{tabular}{c}
                \includegraphics[width=0.7\textwidth]{Immagini/Mockup/notifiche/scenario principale/perDisattivareRisposte.png} \\
                Cockburn: step 4
            \end{tabular}
        };
        
        % Nodo per immagine 3 con didascalia sotto, posizionato sotto img2
        \node (img3) [below=of img2] {
            \begin{tabular}{c}
                \includegraphics[width=0.7\textwidth]{Immagini/Mockup/notifiche/scenario principale/perDisattivareNueve.png} \\
                Cockburn: step 4
            \end{tabular}
        };

        
        % Disegna le frecce
        \draw[->, thick] (img1) -- (img2);
        \draw[->, thick] (img2) -- (img3);
        
    \end{tikzpicture}
    \caption{Mockup: scenario principale della tabella di Cockburn del caso d'uso disattiva/attiva categoria notifica}
    \label{fig:tikz_flow}
\end{figure}

\clearpage
\newpage

\begin{figure}[ht]
    \centering
    \begin{tikzpicture}[node distance=1.5cm and 1cm, auto]
      

         \node (img4){
            \begin{tabular}{c}
                \includegraphics[width=0.7\textwidth]{Immagini/Mockup/notifiche/scenario principale/clickConferma.png} \\
                Cockburn: step 5
            \end{tabular}
        };

        \node (img5) [below=of img4] {
            \begin{tabular}{c}
                \includegraphics[width=0.7\textwidth]{Immagini/Mockup/notifiche/scenario principale/allerAvvisoDisattivazione.png} \\
                Cockburn: step 6/7
            \end{tabular}
        };

        \node (img6) [below=of img5] {
            \begin{tabular}{c}
                \includegraphics[width=0.7\textwidth]{Immagini/Mockup/notifiche/scenario principale/ScenarioPrincipaleCompletato.png} \\
                Conckburn: step 8/9/10
            \end{tabular}
        };
        
        % Disegna le frecce
        \draw[->, thick] (img4) -- (img5);
        \draw[->, thick] (img5) -- (img6);
        
    \end{tikzpicture}
    \caption{Parte 2 mockup: scenario principale della tabella di Cockburn del caso d'uso disattiva/attiva categoria notifica}
    \label{fig:mockup_scenario_principale_parte2_disattiva_notifiche}
\end{figure}

\clearpage
\newpage

\subsubsection{Estensione A: Disattivazione Notifiche dalla Visualizzazione di una Notifica}

Per offrire un maggiore controllo sulla gestione delle notifiche senza interrompere l’esperienza utente, il sistema permette di disattivare una categoria direttamente dalla visualizzazione di una notifica specifica. Questa variante è progettata per garantire una modifica consapevole delle preferenze, evitando azioni impulsive che potrebbero compromettere la ricezione di informazioni rilevanti.

\vspace{0.5cm}
\subsubsection{Interfaccia e Comportamento del Bottone}
Alla fine del testo di ogni notifica, se la relativa categoria è attiva, è presente un pulsante neutro con la dicitura “Disattiva notifiche”. Accanto al pulsante, un testo in grigio informa l’utente della funzione del pulsante, evitando ambiguità. L’utilizzo di colori non accesi e di un design discreto segue il principio della \textbf{gerarchia visiva} \cite{pieters2004}, scoraggiando azioni impulsive che potrebbero portare alla perdita involontaria di notifiche future.

\vspace{0.5cm}
\subsubsection{Modifica dello Stato e Feedback Visivo}
Quando l’utente clicca sul pulsante, il testo del pulsante cambia colore, diventando rosso, e il messaggio a fianco si aggiorna per sottolineare che la categoria di notifiche è stata disattivata. Questo utilizza il principio della \textbf{salienza visiva} \cite{nielsen1995}, enfatizzando il cambiamento e rendendo immediatamente chiara la conseguenza dell’azione.

\vspace{0.5cm}
\subsubsection{Incentivo alla Riattivazione}
Una volta disattivata una categoria tramite questa modalità, viene visualizzato un secondo pulsante con una call-to-action mirata per incentivare la riattivazione delle notifiche. Il design e il posizionamento del pulsante sfruttano il \textbf{principio dell’affordance} \cite{norman1988}, rendendo chiaro che l’utente ha la possibilità di tornare indietro sulla sua decisione in modo semplice e immediato.
\newline
Questa estensione si integra perfettamente con il modello generale di gestione delle notifiche, garantendo un’interazione fluida e coerente con le esigenze dell’utente. Nel caso in cui l’utente scelga di riattivare la categoria delle notifiche direttamente da una notifica, si passa all’\textbf{Estensione F}, che approfondisce questa modalità di gestione a partire dalle notifiche disattivate.
\begin{figure}[ht]
    \centering
    \begin{tikzpicture}[node distance=1.5cm and 1cm, auto]
        % Nodo per immagine 1 con didascalia sotto
        \node (img1) {
            \begin{tabular}{c}
                \includegraphics[width=0.4\textwidth]{Immagini/Mockup/notifiche/estensione A/clickNotifica.png} \\
                Cockburn: Extension A.2/A.3
            \end{tabular}
        };
        
        % Nodo per immagine 2 con didascalia sotto, posizionato a destra di img1
        \node (img2) [below=of img1] {
            \begin{tabular}{c}
                \includegraphics[width=0.4\textwidth]{Immagini/Mockup/notifiche/estensione A/clickDisattiva.png} \\
                Cockburn: Extension A.4
            \end{tabular}
        };
        
        % Nodo per immagine 3 con didascalia sotto, posizionato sotto img2
        \node (img3) [below=of img2] {
            \begin{tabular}{c}
                \includegraphics[width=0.4\textwidth]{Immagini/Mockup/notifiche/estensione A/disattivato.png} \\
                Cockburn: extension A.5
            \end{tabular}
        };
        
        % Disegna le frecce
        \draw[->, thick] (img1) -- (img2);
        \draw[->, thick] (img2) -- (img3);
      
    \end{tikzpicture}
    \caption{Mockup: estensione A della tabella di Cockburn del caso d'uso disattiva/attiva categoria notifica}
    \label{fig:tikz_flow}
\end{figure}

\newpage


\clearpage
\newpage
\subsubsection{Estensione B: Ripristino di un Annuncio Precedente}
Se l’utente sceglie di \textbf{ripristinare l’annuncio precedente}, il sistema avvia un processo di caricamento per fornire un feedback visivo sulla ripresa dei dati. Sebbene i dati siano salvati localmente e il recupero sia immediato, un \textbf{indicatore di caricamento fittizio} viene mostrato per alcuni secondi prima di caricare la schermata.

Questa soluzione è basata sul principio della \textbf{coerenza con le aspettative dell’utente} \cite{shneiderman2004}. In un contesto digitale, un ripristino istantaneo potrebbe apparire innaturale e creare confusione. L’indicatore di caricamento:
\begin{itemize}
    \item Rafforza la percezione di un processo in corso, migliorando la trasparenza dell’operazione.
    \item Evita che l’utente si domandi se il recupero sia realmente avvenuto o se ci siano stati problemi tecnici.
    \item Contribuisce a una transizione più fluida tra stati dell’interfaccia.
\end{itemize}

Una volta completato il caricamento, il sistema presenta l’interfaccia con i dati precedentemente salvati, consentendo all’utente di riprendere il processo da dove era stato interrotto.


\begin{figure}[ht]
    \centering
    \begin{tikzpicture}[node distance=1.5cm and 1cm, auto]
        % Nodo per immagine 1 con didascalia sotto
        \node (img1) {
            \begin{tabular}{c}
                \includegraphics[width=0.7\textwidth]{Immagini/Mockup/aggiungi annuncio/estensione B/step1.png} \\
                click nuovo annuncio
            \end{tabular}
        };
        
        % Nodo per immagine 2 con didascalia sotto, posizionato a destra di img1
        \node (img2) [below=of img1] {
            \begin{tabular}{c}
                \includegraphics[width=0.7\textwidth]{Immagini/Mockup/aggiungi annuncio/estensione B/step2.png} \\
                Cockburn: extension B.2/B.3/B.4
            \end{tabular}
        };
        
        % Nodo per immagine 3 con didascalia sotto, posizionato sotto img2
        \node (img3) [below=of img2] {
            \begin{tabular}{c}
                \includegraphics[width=0.7\textwidth]{Immagini/Mockup/aggiungi annuncio/estensione B/step3.png} \\
                Cockburn: extension B.5
            \end{tabular}
        };
        
        % Disegna le frecce
        \draw[->, thick] (img1) -- (img2);
        \draw[->, thick] (img2) -- (img3);
      
    \end{tikzpicture}
    \caption{Mockup: estensione B della tabella di Cockburn del caso d'uso nuovo annuncio}
    \label{fig:mockup_estensione_B_aggiungi_annuncio}
\end{figure}

\newpage



\input{Requisiti Del Software/Analisi dei Requisiti/Mockup/disattivazione notifiche/estensione c}

\clearpage
\newpage

\subsubsection{Estensione D: Modifica Rapida dello Stato delle Notifiche con Attivazione Immediata}

Questa variante rappresenta l’approccio duale dell’\textbf{Estensione C}, semplificando ulteriormente l’attivazione delle notifiche. L’obiettivo è ridurre i passaggi necessari per riattivare una categoria disattivata, mantenendo comunque un controllo chiaro sulla disattivazione.

\vspace{0.5cm}
\subsubsection{Interfaccia e Interazione}
Analogamente all’\textbf{Estensione C}, ogni categoria nella barra laterale dispone di un pulsante contestuale per modificarne lo stato. Il pulsante può assumere due stati:

\begin{itemize}
    \item \textbf{Attiva}, se la categoria è attualmente disabilitata.
    \item \textbf{Disattiva}, se la categoria è attualmente abilitata.
\end{itemize}

Le interazioni dell’utente variano a seconda dell’azione eseguita:

\begin{itemize}
    \item \textbf{Disattivazione di una categoria}:
    \begin{itemize}
        \item Al clic su \textbf{Disattiva}, appare un popup di conferma che informa l’utente sulle conseguenze della scelta, prevenendo azioni accidentali in linea con i principi di Nielsen \cite{nielsen1995}.
        \item Se confermata, la categoria viene spostata nella sezione delle notifiche disattivate tramite un’animazione di transizione.
        \item Il pulsante cambia stato, diventando \textbf{Attiva}, fornendo un feedback visivo chiaro sulla modifica.
    \end{itemize}
    
    \item \textbf{Attivazione di una categoria}:
    \begin{itemize}
        \item Al clic su \textbf{Attiva}, il sistema aggiorna immediatamente lo stato della notifica senza richiedere una conferma esplicita.
        \item La categoria viene spostata nella sezione delle notifiche attive con un’animazione fluida, applicando il principio della \textbf{gestalt della continuità} \cite{miller1956}.
        \item Il pulsante cambia stato in \textbf{Disattiva}, rendendo la modifica evidente e intuitiva.
    \end{itemize}
\end{itemize}

\subsubsection{Feedback Visivo e UX Design}
L’esperienza utente è ottimizzata tramite tecniche di design che garantiscono chiarezza e immediatezza:

\begin{itemize}
    \item \textbf{Popup di conferma per la disattivazione}: aiuta a prevenire errori e rende consapevole l’utente delle conseguenze della scelta \cite{nielsen1995}.
    \item \textbf{Animazione di transizione}: assicura una continuità visiva fluida nello spostamento delle categorie, migliorando la percezione del cambiamento \cite{miller1956}.
    \item \textbf{Aggiornamento immediato dello stato del pulsante}: il cambio di testo e colore riflette lo stato corrente della categoria, riducendo l’ambiguità e migliorando la prevedibilità dell’interazione.
\end{itemize}

Questa estensione semplifica l’attivazione delle notifiche, eliminando il passaggio della conferma e migliorando la fluidità dell’interazione, senza compromettere il controllo dell’utente sulla gestione delle proprie preferenze.

\begin{figure}[ht]
    \centering
    \begin{tikzpicture}[node distance=1.5cm and 1cm, auto]
        % Nodo per immagine 1 con didascalia sotto
        \node (img1) {
            \begin{tabular}{c}
                \includegraphics[width=0.6\textwidth]{Immagini/Mockup/notifiche/estensione D/clickAttiva.png} \\
                Cockburn: Extension D.2
            \end{tabular}
        };
        
        % Nodo per immagine 2 con didascalia sotto, posizionato a destra di img1
        \node (img2) [below=of img1] {
            \begin{tabular}{c}
                \includegraphics[width=0.6\textwidth]{Immagini/Mockup/notifiche/estensione D/attivato.png} \\
                Cockburn: step 8/9/10
            \end{tabular}
        };
        
        % Disegna le frecce
        \draw[->, thick] (img1) -- (img2);
      
    \end{tikzpicture}
    \caption{Mockup: estensione D della tabella di Cockburn del caso d'uso disattiva/attiva categoria notifica}
    \label{fig:tikz_flow}
\end{figure}

\newpage



\clearpage
\newpage

\subsubsection{Estensione E: Errore Durante la Modifica dello Stato delle Categorie di Notifica}

Nel caso in cui si verifichi un errore durante il tentativo di modifica dello stato di una categoria di notifica, viene visualizzato un messaggio di errore sotto forma di un popup informativo. Il messaggio informa l'utente che la modifica non è riuscita e lo invita a riprovare più tardi, senza salvare le modifiche effettuate.

\vspace{0.5cm}
\subsubsection{Gestione dell'Errore e Feedback Utente} Il popup di errore presenta i seguenti elementi chiave: \begin{itemize} \item \textbf{Messaggio chiaro e informativo}: il messaggio comunica all'utente che l'operazione non è andata a buon fine, senza entrare in dettagli tecnici, per ridurre il rischio di frustrazione. L'utente viene anche informato che l'operazione non è stata completata e che le modifiche non sono state salvate. \item \textbf{Pulsante “Ok”}: consente all'utente di chiudere il popup e tornare all'interfaccia principale. Il pulsante di conferma è chiaro e consente di riprendere l'interazione senza indugi. \end{itemize}

\subsubsection{Principi di Design Applicati} L'approccio di gestione dell'errore in questa estensione si fonda sui seguenti principi di UX e usabilità: \begin{itemize} \item \textbf{Visibilità dello stato del sistema} \cite{nielsen1995}: l'errore è comunicato all'utente attraverso un popup che evidenzia chiaramente che il sistema non è riuscito a completare l'operazione. \item \textbf{Prevenzione degli errori} \cite{nielsen1995}: sebbene l'errore non possa essere evitato completamente, il sistema offre un feedback immediato e comprensibile, impedendo all'utente di rimanere confuso o incerto sullo stato dell'operazione. \item \textbf{Semplicità e chiarezza} \cite{nielsen1995}: il messaggio di errore è semplice e diretto, senza sovraccaricare l'utente con informazioni tecniche. L'invito a riprovare più tardi mantiene il flusso di lavoro semplice e lineare. \item \textbf{Controllo dell'utente} \cite{norman1988}: l'utente ha il pieno controllo sulla gestione dell'errore, poiché il popup permette di chiudere facilmente l'interfaccia e riprendere l'attività, mantenendo un'esperienza utente fluida. \end{itemize}

Questa soluzione di gestione dell'errore è progettata per garantire un'esperienza utente chiara e senza frustrazioni, minimizzando il disagio derivante da errori tecnici imprevisti e offrendo un percorso semplice per riprendere l'interazione.
\begin{figure}[ht]
    \centering
    \begin{tikzpicture}[node distance=1.5cm and 1cm, auto]
        % Nodo per immagine 1 con didascalia sotto
        \node (img1) {
            \begin{tabular}{c}
                \includegraphics[width=0.6\textwidth]{Immagini/Mockup/notifiche/estensione E/click conferma.png} \\
                Cockburn: step 7
            \end{tabular}
        };
        
        % Nodo per immagine 2 con didascalia sotto, posizionato a destra di img1
        \node (img2) [below=of img1] {
            \begin{tabular}{c}
                \includegraphics[width=0.6\textwidth]{Immagini/Mockup/notifiche/estensione E/errore.png} \\
                Cockburn: Extension E.7/E.8
            \end{tabular}
        };
        
        % Disegna le frecce
        \draw[->, thick] (img1) -- (img2);
      
    \end{tikzpicture}
    \caption{Mockup: estensione E della tabella di Cockburn del caso d'uso disattiva/attiva categoria notifica}
    \label{fig:tikz_flow}
\end{figure}

\newpage



\clearpage
\newpage

\subsubsection{Estensione F: Riattivazione Notifiche dalle Notifiche Disattivate}

Questa estensione consente all’utente di riattivare una categoria di notifiche direttamente da una notifica precedentemente ricevuta e appartenente a una categoria disattivata. L’obiettivo è fornire un meccanismo immediato per ripristinare le notifiche quando l’utente si rende conto della loro utilità.

\vspace{0.5cm}
\subsubsection{Indicazione dello Stato e Call-to-Action}
Quando una notifica proviene da una categoria disattivata, il sistema mostra un messaggio di avviso evidenziato che informa l’utente che non riceverà più aggiornamenti simili. Il pulsante associato cambia stato e diventa un invito all’azione con il testo “Riattiva notifiche”. Questo sfrutta il \textbf{principio della reversibilità} \cite{shneiderman2004}, permettendo all’utente di annullare la decisione precedente senza difficoltà.

\vspace{0.5cm}
\subsubsection{Feedback Visivo e Conferma della Riattivazione}
Alla pressione del pulsante, il testo cambia colore in verde e il messaggio informativo si aggiorna, confermando che le notifiche per quella categoria sono state riattivate. Il sistema può fornire un’ulteriore conferma con un breve messaggio di notifica o una vibrazione del dispositivo per enfatizzare l’azione completata.

\vspace{0.5cm}
\subsubsection{Coerenza con il Modello di Gestione Notifiche}
Questa estensione rafforza la coerenza dell’interfaccia di gestione delle notifiche, mantenendo le scelte dell’utente sempre modificabili e promuovendo un’interazione trasparente e prevedibile. L’utente può così gestire le notifiche senza dover accedere necessariamente alla schermata delle impostazioni, riducendo il carico cognitivo e migliorando l’usabilità complessiva del sistema.\begin{figure}[H]
    \centering
    \begin{tikzpicture}[node distance=1.5cm and 1cm, auto]
        % Nodo per immagine 1 con didascalia sotto
        \node (img1) {
            \begin{tabular}{c}
                \includegraphics[width=0.4\textwidth]{Immagini/Mockup/notifiche/estensione F/clickNotifica.png} \\
                Cockburn: Extension F.2
            \end{tabular}
        };
        
        % Nodo per immagine 2 con didascalia sotto, posizionato a destra di img1
        \node (img2) [below=of img1] {
            \begin{tabular}{c}
                \includegraphics[width=0.4\textwidth]{Immagini/Mockup/notifiche/estensione F/clickAttiva.png} \\
                Cockburn: Extension F.3
            \end{tabular}
        };
        
        % Nodo per immagine 3 con didascalia sotto, posizionato sotto img2
        \node (img3) [below=of img2] {
            \begin{tabular}{c}
                \includegraphics[width=0.4\textwidth]{Immagini/Mockup/notifiche/estensione F/attivato.png} \\
                Cockburn: Extension F.4/F.5
            \end{tabular}
        };
        
        % Disegna le frecce
        \draw[->, thick] (img1) -- (img2);
        \draw[->, thick] (img2) -- (img3);
      
    \end{tikzpicture}
    \caption{Mockup: estensione F della tabella di Cockburn del caso d'uso disattiva/attiva categoria notifica}
    \label{fig:tikz_flow}
\end{figure}

\newpage

