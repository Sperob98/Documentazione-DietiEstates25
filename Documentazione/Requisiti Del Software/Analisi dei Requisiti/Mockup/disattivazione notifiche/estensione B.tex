\clearpage
\newpage

\subsubsection{Estensione B: Conferma Senza Modifiche}

Nel caso in cui l'utente apra il popup "Attiva e Disattiva Notifiche" e prema il pulsante di conferma senza apportare alcuna modifica allo stato delle categorie di notifica, il sistema segue un comportamento mirato a garantire un'interazione fluida e priva di frizioni inutili.

\vspace{0.5 cm}
\subsubsection{Gestione dell'Interazione}

Per evitare interruzioni superflue, il sistema rileva automaticamente che non sono state effettuate modifiche e si limita a chiudere il popup senza richiedere ulteriori conferme. Questo approccio è in linea con le \textbf{euristiche di usabilità di Nielsen} \cite{nielsen1995}, in particolare il principio di \textbf{minimizzazione del carico cognitivo}, che riduce il numero di azioni necessarie per completare un’operazione priva di effetti.

\vspace{0.5cm}
\subsubsection{Feedback Visivo}

Sebbene non venga mostrato alcun messaggio di doppia conferma, il sistema utilizza piccoli accorgimenti di UX design per rendere l'interazione chiara:
\begin{itemize}
    \item \textbf{Animazione di chiusura fluida}: il popup si chiude con una breve transizione, migliorando la percezione di un'interazione naturale.
    \item \textbf{Assenza di notifiche di conferma}: poiché nessuna modifica è stata apportata, il sistema evita messaggi ridondanti che potrebbero confondere l’utente.
    \item \textbf{Chiarezza dello stato iniziale}: i toggle switch rimangono invariati, rafforzando la prevedibilit\`a del sistema.
\end{itemize}

Questo comportamento segue il principio della \textbf{coerenza e prevedibilit\`a} \cite{norman1988}, assicurando che il sistema reagisca in modo intuitivo alle azioni dell'utente senza introdurre complessità non necessarie.

