\subsection{Caso d'Uso: Fare una controproposta di un offerta}

TODO: Inserire descrizione

\begin{figure}[H]
	\centering
	\begin{tikzpicture}[node distance=1.5cm and 1cm, auto]
		% Nodo per immagine 1 con didascalia sotto
		\node (img1) {
			\begin{tabular}{c}
				\includegraphics[width=0.7\textwidth]{Immagini/Mockup/controproposte/scenario principale/clickControproposta.png} \\
				Cockburn: step 1/2/3/4
			\end{tabular}
		};
		
		% Nodo per immagine 2 con didascalia sotto, posizionato a destra di img1
		\node (img2) [below=of img1] {
			\begin{tabular}{c}
				\includegraphics[width=0.7\textwidth]{Immagini/Mockup/controproposte/scenario principale/ClickInviaControproposta.png} \\
				Cockburn: step 5/6/7
			\end{tabular}
		};
		
		% Nodo per immagine 3 con didascalia sotto, posizionato sotto img2
		\node (img3) [below=of img2] {
			\begin{tabular}{c}
				\includegraphics[width=0.7\textwidth]{Immagini/Mockup/controproposte/scenario principale/allertConfermaInvio.png} \\
				Cockburn: step 8/9/10
			\end{tabular}
		};
		
		% Disegna le frecce
		\draw[->, thick] (img1) -- (img2);
		\draw[->, thick] (img2) -- (img3);
		
	\end{tikzpicture}
	\caption{Mockup: scenario principale della tabella di Cockburn del caso d'uso: Fare una controproposta a un'offerta.}
	\label{fig:tikz_flow}
\end{figure}

\newpage

\begin{figure}[H]
	\centering
	\includegraphics[width=0.7\linewidth]{"Immagini/Mockup/controproposte/scenario principale/visualizzazioneControproposta"}
	\caption[Visualizzazione della controproposta]{}
	\label{fig:visualizzazionecontroproposta}
\end{figure}

\subsubsection{Estensione A: Prezzo della Controproposta Mancante}

Durante la compilazione della controproposta, l’agente potrebbe tentare di inviare il modulo senza aver specificato il prezzo proposto.  
Per prevenire errori di input e garantire l’integrità dei dati inviati al sistema, viene implementato un controllo di validazione lato client e lato server.

\subsubsection{Gestione del Messaggio di Errore}
Nel momento in cui l’agente clicca su \textbf{“Invia controproposta”} con il campo del prezzo vuoto, il sistema mostra un messaggio di errore direttamente sotto l’input della controproposta.  
Questo messaggio informa chiaramente l’utente dell’obbligo di compilare il campo prima di procedere.

L’interfaccia rimane attiva e consente all’agente di inserire il valore corretto e riprovare l’invio, tornando così al flusso principale dello scenario (Step 6).  

Questa scelta progettuale segue il principio della \textbf{visibilità dello stato del sistema} \cite{nielsen1995}, migliorando la chiarezza e riducendo la frustrazione dell’utente.

\begin{figure}[H]
	\centering
	\begin{tikzpicture}[node distance=1.5cm and 1cm, auto]
		% Nodo per immagine 2 con didascalia sotto, posizionato a destra di img1
		\node (img1) {
			\begin{tabular}{c}
				\includegraphics[width=0.7\textwidth]{Immagini/Mockup/controproposte/scenario principale/ClickInviaControproposta.png} \\
				Cockburn: step 6.A
			\end{tabular}
		};
		
		% Nodo per immagine 3 con didascalia sotto, posizionato sotto img2
		\node (img2) [below=of img1] {
			\begin{tabular}{c}
				\includegraphics[width=0.7\textwidth]{Immagini/Mockup/controproposte/Extensions A/MessaggioDiErrore.png} \\
				Cockburn: step 7.A
			\end{tabular}
		};
		
		% Disegna le frecce
		\draw[->, thick] (img1) -- (img2);
		
	\end{tikzpicture}
	\caption{Mockup: Extension A della tabella di Cockburn del caso d'uso: Fare una controproposta a un'offerta.}
	\label{fig:tikz_flow}
\end{figure}

\newpage




\newpage

\subsubsection{Estensione B: Prezzo Inferiore all’Offerta Originale}

Nel caso in cui l’agente inserisca un prezzo di controproposta inferiore a quello dell’offerta ricevuta, il sistema intercetta l’anomalia e informa immediatamente l’utente dell’errore.

\subsubsection{Validazione del Prezzo e Feedback all’Utente}
Dopo aver cliccato su \textbf{“Invia controproposta”}, viene mostrato un messaggio di errore sotto al campo di input, specificando che il prezzo inserito non può essere inferiore a quello proposto dall’acquirente.  
Il sistema non consente il salvataggio della controproposta fino alla correzione del valore inserito.

Questo comportamento applica il principio della \textbf{prevenzione degli errori}, garantendo coerenza logica nelle transazioni e riducendo la possibilità di offerte incoerenti o non valide.

\begin{figure}[H]
	\centering
	\begin{tikzpicture}[node distance=1.5cm and 1cm, auto]
		% Nodo per immagine 2 con didascalia sotto, posizionato a destra di img1
		\node (img1) {
			\begin{tabular}{c}
				\includegraphics[width=0.7\textwidth]{Immagini/Mockup/controproposte/scenario principale/ClickInviaControproposta.png} \\
				Cockburn: step 7.B
			\end{tabular}
		};
		
		% Nodo per immagine 3 con didascalia sotto, posizionato sotto img2
		\node (img2) [below=of img1] {
			\begin{tabular}{c}
				\includegraphics[width=0.7\textwidth]{Immagini/Mockup/controproposte/Extensions B/MessaggioDiErrore.png} \\
				Cockburn: step 8.B
			\end{tabular}
		};
		
		% Disegna le frecce
		\draw[->, thick] (img1) -- (img2);
		
	\end{tikzpicture}
	\caption{Mockup: Extension B della tabella di Cockburn del caso d'uso: Fare una controproposta a un'offerta.}
	\label{fig:tikz_flow}
\end{figure}

\newpage




\newpage

subsubsection{Estensione C: Errore di Salvataggio della Controproposta}

In alcune circostanze, l’invio della controproposta può fallire a causa di problemi di connessione o di errori del server.  
Il sistema gestisce questa eventualità mostrando all’utente un chiaro feedback visivo sull’esito dell’operazione.

\subsubsection{Gestione dell’Errore e Comunicazione all’Utente}
Dopo aver cliccato su \textbf{“Invia controproposta”}, viene mostrato un breve stato di caricamento.  
Se il salvataggio non va a buon fine, compare un pop-up di allerta che informa l’agente dell’errore e della mancata registrazione della controproposta.

L’utente può chiudere l’allerta e tentare nuovamente l’operazione una volta ristabilita la connessione.  
Questo comportamento supporta la \textbf{recuperabilità dall’errore}, assicurando che l’utente non perda il lavoro svolto e possa riprovare senza dover reinserire tutti i dati.

\begin{figure}[H]
	\centering
	\begin{tikzpicture}[node distance=1.5cm and 1cm, auto]
		% Nodo per immagine 2 con didascalia sotto, posizionato a destra di img1
		\node (img1) {
			\begin{tabular}{c}
				\includegraphics[width=0.7\textwidth]{Immagini/Mockup/controproposte/scenario principale/ClickInviaControproposta.png} \\
				Cockburn: step 7.C/8.C
			\end{tabular}
		};
		
		% Nodo per immagine 3 con didascalia sotto, posizionato sotto img2
		\node (img2) [below=of img1] {
			\begin{tabular}{c}
				\includegraphics[width=0.7\textwidth]{Immagini/Mockup/controproposte/Extensions C/MessaggioDiErrore.png} \\
				Cockburn: step 9.C
			\end{tabular}
		};
		
		% Disegna le frecce
		\draw[->, thick] (img1) -- (img2);
		
	\end{tikzpicture}
	\caption{Mockup: Extension C della tabella di Cockburn del caso d'uso: Fare una controproposta a un'offerta.}
	\label{fig:tikz_flow}
\end{figure}

\newpage


