

\subsection{Caso d'Uso: Registrazione di un Nuovo Agente}

Il caso d’uso «Registra nuovo agente» è stato progettato con l’obiettivo di garantire un’esperienza utente fluida e coerente con il resto dell’interfaccia, evitando interruzioni del contesto visivo. Per questo motivo, si è scelto di non introdurre una schermata dedicata, ma di utilizzare una finestra modale (popup), che consente all’utente di eseguire l’operazione senza abbandonare la vista corrente.

\vspace{0.5cm}
\subsubsection{Struttura e Flusso dell’Interazione}
L’interazione si articola in più fasi, guidando l’utente in modo progressivo e controllato:
\begin{itemize}
	\item \textbf{Popup iniziale}: contiene un form per l’inserimento dei dati anagrafici e professionali dell’agente, accompagnato da un unico pulsante in calce con etichetta «Registra nuovo utente».
	\item \textbf{Conferma dell’operazione}: al clic sul pulsante, viene mostrato un \textit{dialog} di doppia conferma per ridurre il rischio di azioni involontarie. I pulsanti di conferma e annullamento sono caratterizzati da colori neutri, in modo da distinguerli visivamente dall’azione primaria dell’applicazione.
	\item \textbf{Feedback di caricamento}: dopo la conferma, il sistema mostra una barra di avanzamento stilizzata con i colori istituzionali del sito. Anche nei casi in cui l’operazione risulti immediata, la barra può essere mantenuta visibile per una breve durata, al fine di fornire un riscontro percettivo chiaro e rassicurante \cite{nielsen1995}.
\end{itemize}

\vspace{0.5cm}
\subsubsection{Messaggio di Conferma e Gestione delle Credenziali}
Al completamento del processo di registrazione, il sistema visualizza un messaggio di successo che informa l’utente dell’avvenuta generazione delle credenziali di accesso.
Il pulsante contestuale assume l’etichetta \textbf{«Copia credenziali»}, permettendo di salvare in modo sicuro i dati generati.
Per prevenire la perdita delle credenziali, la chiusura del dialog è temporaneamente disabilitata fino a quando l’utente non conferma l’avvenuta copia.

\vspace{0.5cm}
\subsubsection{Visualizzazione e Sicurezza dei Dati}
In linea con i principi di \textbf{minimizzazione dei dati} e \textbf{sicurezza dell’informazione} \cite{wickens2008}, la visualizzazione completa delle credenziali non è obbligatoria:
il sistema privilegia la protezione dei dati sensibili, evitando di esporre informazioni in chiaro nell’interfaccia.
In alternativa, è possibile prevedere una versione mascherata delle credenziali o la sola visualizzazione di elementi non sensibili, mantenendo comunque la trasparenza informativa per l’utente.
\begin{figure}[H]
	\centering
	\begin{tikzpicture}[node distance=1.5cm and 1cm, auto]
		% Nodo per immagine 1 con didascalia sotto
		\node (img1) {
			\begin{tabular}{c}
				\includegraphics[width=0.7\textwidth]{Immagini/Mockup/nuovoAgente/scenario principale/clickNuovoDipendente.png} \\
				Cockburn: step 1
			\end{tabular}
		};
		
		% Nodo per immagine 2 con didascalia sotto, posizionato a destra di img1
		\node (img2) [below=of img1] {
			\begin{tabular}{c}
				\includegraphics[width=0.7\textwidth]{Immagini/Mockup/nuovoAgente/scenario principale/clickFormNuovoAgente.png} \\
				Cockburn: step 2/3/4
			\end{tabular}
		};
		
		% Nodo per immagine 3 con didascalia sotto, posizionato sotto img2
		\node (img3) [below=of img2] {
			\begin{tabular}{c}
				\includegraphics[width=0.7\textwidth]{Immagini/Mockup/nuovoAgente/scenario principale/ClickAllertConferma.png} \\
				Cockburn: step 5/6
			\end{tabular}
		};
		
		% Disegna le frecce
		\draw[->, thick] (img1) -- (img2);
		\draw[->, thick] (img2) -- (img3);
		
	\end{tikzpicture}
	\caption{Mockup: scenario principale della tabella di Cockburn del caso d'uso: Registra nuovo agente.}
	\label{fig:tikz_flow}
\end{figure}

\newpage

\begin{figure}[H]
	\centering
	\begin{tikzpicture}[node distance=1.5cm and 1cm, auto]
		% Nodo per immagine 1 con didascalia sotto
		\node (img1) {
			\begin{tabular}{c}
				\includegraphics[width=0.7\textwidth]{Immagini/Mockup/nuovoAgente/scenario principale/caricamentoRegistrazione.png} \\
				Cockburn: step 6/7/9
			\end{tabular}
		};
		
		% Nodo per immagine 2 con didascalia sotto, posizionato a destra di img1
		\node (img2) [below=of img1] {
			\begin{tabular}{c}
				\includegraphics[width=0.7\textwidth]{Immagini/Mockup/nuovoAgente/scenario principale/allertRegistrazioneEffettuata.png} \\
				Cockburn: step 10
			\end{tabular}
		};
		
		% Nodo per immagine 3 con didascalia sotto, posizionato sotto img2
		\node (img3) [below=of img2] {
			\begin{tabular}{c}
				\includegraphics[width=0.7\textwidth]{Immagini/Mockup/nuovoAgente/scenario principale/visualizzazioneNuovaRegistrazione.png} \\
				Cockburn: step 11
			\end{tabular}
		};
		
		% Disegna le frecce
		\draw[->, thick] (img1) -- (img2);
		\draw[->, thick] (img2) -- (img3);
		
	\end{tikzpicture}
	\caption{Mockup: scenario principale della tabella di Cockburn del caso d'uso: Registra nuovo agente.}
	\label{fig:tikz_flow}
\end{figure}

\newpage

\begin{figure}[H]
	\centering
	\begin{tikzpicture}[node distance=1.5cm and 1cm, auto]
		% Nodo per immagine 1 con didascalia sotto
		\node (img1) {
			\begin{tabular}{c}
				\includegraphics[width=0.7\textwidth]{Immagini/Mockup/nuovoAgente/extension A/nomeNonInserito.png} \\
				Cockburn: step 4.A
			\end{tabular}
		};
		
		% Nodo per immagine 2 con didascalia sotto, posizionato a destra di img1
		\node (img2) [below=of img1] {
			\begin{tabular}{c}
				\includegraphics[width=0.7\textwidth]{Immagini/Mockup/nuovoAgente/extension A/messaggioDiErrore.png} \\
				Cockburn: step 5.A
			\end{tabular}
		};
	
		
		% Disegna le frecce
		\draw[->, thick] (img1) -- (img2);
		
	\end{tikzpicture}
	\caption{Mockup: Extension A della tabella di Cockburn del caso d'uso: Registra nuovo agente.}
	\label{fig:tikz_flow}
\end{figure}




\subsubsection{Estensione D: Mancata Connessione al Sistema}

In alcune circostanze, il manager potrebbe non riuscire a completare la registrazione a causa di problemi di connessione o di un temporaneo malfunzionamento del server.
Il sistema gestisce questa evenienza fornendo un chiaro feedback sull’esito negativo dell’operazione.

\subsubsection{Gestione dell’Errore di Connessione}
Dopo aver cliccato su \textbf{“Conferma”}, il sistema mostra un breve stato di caricamento.
Se la connessione al server non riesce, viene visualizzato un pop-up di allerta che informa l’utente dell’impossibilità di completare la registrazione per motivi tecnici.

In questo caso, l’\textbf{use case è considerato fallito}, ma il manager può chiudere il messaggio e riprovare l’operazione una volta ristabilita la connessione.
Questa soluzione si basa sui principi di \textbf{visibilità dello stato del sistema} e di \textbf{recuperabilità dall’errore}, assicurando chiarezza e prevedibilità anche in situazioni di errore tecnico.
\begin{figure}[H]
	\centering
	\begin{tikzpicture}[node distance=1.5cm and 1cm, auto]
			% Nodo per immagine 3 con didascalia sotto, posizionato sotto img2
		\node (img1){
			\begin{tabular}{c}
				\includegraphics[width=0.7\textwidth]{Immagini/Mockup/nuovoAgente/scenario principale/ClickAllertConferma.png} \\
				Cockburn: step 6.D
			\end{tabular}
		};
		
		% Nodo per immagine 1 con didascalia sotto
		\node (img2)  [below=of img1] {
			\begin{tabular}{c}
				\includegraphics[width=0.7\textwidth]{Immagini/Mockup/nuovoAgente/scenario principale/caricamentoRegistrazione.png} \\
				Cockburn: step 7.D
			\end{tabular}
		};
		
		% Nodo per immagine 3 con didascalia sotto, posizionato sotto img2
		\node (img3) [below=of img2] {
			\begin{tabular}{c}
				\includegraphics[width=0.7\textwidth]{Immagini/Mockup/nuovoAgente/extension D/messaggioDiErrore.png} \\
				Cockburn: step 8.D
			\end{tabular}
		};
		
		% Disegna le frecce
		\draw[->, thick] (img1) -- (img2);
		\draw[->, thick] (img2) -- (img3);
		
	\end{tikzpicture}
	\caption{Mockup: Extension D della tabella di Cockburn del caso d'uso: Registra nuovo agente.}
	\label{fig:tikz_flow}
\end{figure}
